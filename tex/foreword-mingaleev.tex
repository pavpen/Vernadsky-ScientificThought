\ChapterByLine{Remarks to the Electronic Edition~\dots}{
V. I. Vernadsky Electronic Archive\\
\url{http://vernadsky.lib.ru}
}

The present electronic edition of V. I. Vendasky's book \rtitle{Scientific
Thought as a Planetary Phenomenon\footnoteRus{Научная мысль как планетное
явление}} was being prepared according to the
edition\fullcite{vernadsky1991thought} at the end of 1999.

The first four chapters were prepared by April, 2000, and added to the Maxim
Moshkov library (\url{http://lib.ru/FILOSOF/WERNADSKIJ/}).  These first
chapters were carefully proofread and, I hope, contain very few printing
errors.

The fifth and sixth chapters were proofread (also quite carefully, though not
as well as the first four) by the end of November, 2000.  They were published
on the server of the Electronic Archive (\url{http://vernadsky.lib.ru}), but
were not sent to the Moshkov library in the hope that the remaining four
chapters would be prepared sufficiently quickly.

Unfortunately, because of insufficient time, the work on the remaining chapters
kept dragging on and on, to the point that I decided to use the electronic
version of these chapters, which was prepared by the Russian Foundation for
Fundamental Research\footnoteRus{Росийским Фондом фундаментальных исследований}
from the edition \fullcite{vernadsky1997thought}.

However, comparing these two editions, it seemed to me, that the earlier one,
from 1991, was much closer to the original text of V. I. Vernadsky.  The 1997
edition is filled with slight editorial corrections, which, though nowhere (it
seems) distort Vernadsky's meaning, nevertheless, quite strongly change his
manner of exrpession, and that in such a way that at these places the mind is
often just tripped up, and it is at once apparent that Vladimir Ivanovich could
not have written in that manner.  It is, therefore, necessary to streighten out
chapters 7--10 according to the 1991 edition with time.  It is also necessary
to proofread all chapters once again, and correct any remaining errors.

I include the introductions of the editors of both editions at the begining of
this book, which tell about the history of the writing of Vladimir Ivanovich's
book, as well as about the history of its hard and quite controversial
publication.

For commercial use of the electronic edition of \rtitle{Scientific Thought as a
Planetary Phenomenon}, or (which would be just terrific ;-) for aid with its
proofreading, contact me at the address indicated on the
\url{http://vernadsky.lib.ru} server.

*Note:* The electronic edition is being prepared in the \LaTeX format; it is
necessary to update that version, and not the derived HTML version in the Maxim
Moshkov library when proofreading.

\begin{flushright}
Sergey Mingaleev\footnoteRus{Сергей Мингалеев}\\
October 16, 2001
\end{flushright}
