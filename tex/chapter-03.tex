\Chapter{%
The movement of scientific thought in the 20th century, and its significance in
the geological history of the biosphere.  Its main characteristics: explosion
of scientific work, change in the understanding of the fundamentals of reality,
ecumenicism, and efficient, social manifestation of science.}\label{ch:3}


\Section \label{sec:47}
What is presently occurring in the scientific movement can only be compared
with that scientific movement from the past of science, which was connected
with the birth of Greek philosophy and science in the 6th--7th c.~BC.

Unfortunately, so far we cannot clearly imagine that accumulation of scientific
knowledge which the ancient Greeks had amassed at the time when scientific
thought manifested itself in their environment, and when it, for the first
time, acquired a scientific-philosophical structure, outside of religious,
cosmogonic and poetical constructs---when the scientific method was created for
the first time in the Hellenic city civilization of the polis---logic and
theoretical mathematics applied to life, when the search for scientific truth
became a reality, as a goal for itself in the life of the individual in a
social environment.

The circumstances of this, as history has shown, momentous event in mankind's
life, and in the evolution of the biosphere are, to a large extent, mysterious
and the history of scientific knowledge is being clarified slowly, but
nevertheless ever deeper.  Clear is only a general sketch of the accumulation
of scientific knowledge of the Hellenic environment at that time, the
achievements of the thinkers of Hellenic science, who lived at the time, and
what they received from the previous generations of Hellenic civilization.  We
are slowly beginning to understand this.  This is on the one hand.

And on the other hand, the conceptions about what the Greeks received from
great civilizations preceding them---Asia Minor, Cretan, Chaldean
(Messopotamian), Ancient Egypt, India---are now starting to drastically
change.

Unfortunately, only a \emph{miniscule part} of Hellenic scientific literature
has reached us.  The major researchers have left no trace in the literature
accessible to us, or only fragmentary indications of their scientific work has
reached us.

True, a large part of the complete works of Plato has reached us, as well as a
significant part of Aristotle's scientific works, however, many of the latter's
works, fundamental from the standpoint of the scientific search, have been
lost.  Especially unfortunate, from this standpoint, is the loss of the works
of major scientists, in whose output scientific thought and the scientific
method entered the age of flourishing and synthesis of Hellenic
science---Alcmaeon (500~BC), Leucippus (430~BC), Democritus (420--370~BC),
Hippocrates of Chios (450--430~BC), Philolaus (5th century BC) and many others,
from whom only miniscule fragments, or nothing but names have remained.

The loss of the first attempts at histories of scientific work and thought,
which were written closest to the centuries of its manifestation, may be even
more unfortunate.  Partly distorted, and in an incomplete form, this work has
reached us in the form of nameless essentials, sometimes adapted and skewed in
the course of the many centuries after their publication.  But the originals of
Xenocrates' (397--314) history of Geometry, Eudemus of Rhodes' (circa~320)
history of science, Theophrastus' (372--288) historical books, and others have
been lost in the historical course of Greko-Roman civilization by the time of
our age---during the centuries closest to it, almost a thousand years ago.

In essence, the basic fund of Helenic science---what I call a \emph{scientific
apparatus}\footnote{
	\foreignlanguage{russian}{\fullcite[9--10]{vernadsky1939problemy2}
	(Problems of Biogeochemistry II)}
}---has reached us in miniscule fragments, passing, on top of it, through many
centuries, in the remains of Aristotle's and Theophrastus's works on the
history of natural sciences, as well as in the works of Greek mathematicians.
Nevertheless, it exerted tremendous influence on the Renaissance and on the
creation of Western European science in the 15th--17th centuries.  Our modern
science has been created, to a significant extent, relying on and starting from
this fund's achievements, developing the ideas and knowledge laid out in it.
Broken for centuries, that already during the time of the Roman Empire, the
threads were restored in the 17th century.


\Section % 48
The recent course of the history of science requires us to change
our conceptions of that pre-Hellenic heritage, from which Hellenic science
sprouted, as I already indicated (§42).

The Greeks have everywhere pointed to the great knowledge, which they had
received from Egypt, Chaldea, the East.  We must now admit that they were
correct.  Science had already existed before them---the science of the
``Chaldeans'', reaching back beyond millenia BC, is only now being uncovered
before us---in fragments, proving beyond any doubt its long unsuspected, until
our time, force (§42).

It is now becoming clear that we must attribute a much more real significance,
than has been recently done, to the numerous indications by ancient scientists
and writers of the fact that the creators of Hellenic science and philosophy
took into consideration, proceeded in their creative work from the achievements
of scientists and thinkers from Egypt, Chaldea, Arian and non-Arian
civilizations of the East.

Babylonian scientists worked together with Greek ones in the course of several
centuries.  At the same time, the new flourishing of Babylonian astronomy
occurred in the centuries closest to our age.  Gradually, in the course of
several generations, they merged into the Hellenic cultural environment and
equally suffered the unfavorable for science circumstances of that time (§40).
Undoubtedly, the knowledge received from the scientists of that time was used
by the Greeks during the period of this dialogue.

Undoubtedly, what was harnessed and used by them was very significant by that
time---especially if we consider the multimillenial experience and the
multimillenial tradition of seafaring, engineering, agriculture, irrigation
works, military art, government organization and everyday life.

For centuries Greek science worked in direct contact with Chaldean and Egyptian
science, was merging with them.  Though it is possible that creative thought in
Egyptian science died out during that time---this didn't happen with Chaldean
science (§42).

Hellenic science, in the age of its birth, is a direct continuation of the
intense creative thought of pre-Hellenic science.  This fact is acknowledged,
but still not assimilated, in the history of science.

The ``miracle'' of Hellenic civilization---a historical process, whose results
are clear, but whose course cannot be precisely traced---was a historical
process like others.  It had a solid basis in the past.  Only its result in its
achievement---the rate at which it was achieved---turned out to be singular in
time, and exceptional in its consequences in the noosphere.


\Section % 49

. . .
