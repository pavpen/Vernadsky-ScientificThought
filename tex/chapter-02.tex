\Chapter{%
Manifestation of the historical moment mankind is currently living through as a
geological process.  Evolution of the species of living matter and evolution of
the biosphere into the noosphere.  This evolution cannot be stopped by the
course of global human history.  Scientific thought and mankind's daily lives
as expressions of it.}\label{ch:2}

\Section % 14
We are not yet conscious of, we are not yet living the realization of the
full consequences of the astonishing, unprecedented times that mankind has
entered during the 20th century.

We are living at the threshold of an extremely important, fundamentally new
epoch in the existence of mankind, in mankind's history on our planet.

Mankind has, for the first time, encompassed the whole surface envelope of the
planet---the whole biosphere, all parts of the planet connected to life---with
human life, with human culture.

We are present at, and are actively participating in the creation of a new
\emph{geological factor} in the biosphere, unprecedented in its power and in
its unity.

It has been scientifically established for the last 20--30 thousand years, but
has been clearly manifested at an ever increasing rate only during the last
millenium.

The envelopment of the whole surface of the biosphere by a unified social
species of the animal kingdom---by \emph{mankind}---has been completed after
many hundreds of thousands of years of unstoppable, tempestuous striving for
it.  There is no corner on Earth inaccessible to mankind.  There is no limit to
our possible population growth.  Man, through scientific thought and through
his life, socially organized into states, and guided by technology, is creating
a new \emph{biogenic force} in the biosphere, which is guiding his population
growth and creating favorable conditions for his population in parts of the
biosphere, earlier impenetrable to human life, and even in places where there
was no life before.

Theoretically, we cannot foresee a limit to mankind's potential, if we only
take into account the effect of generations; every geological factor is fully
manifested in the biosphere only in the effect of generations of living beings,
only in geological time.  With the rapidly increasing precision of scientific
work---in this case, of the methodology of scientific observation,---we can now
clearly establish, and study the increase of this new, principally currently
emerging, geological force in historical time.

Mankind is a unified whole, and even if that is recognized by the vast
majority, this unity manifests itself in forms of human life, which actually
deepen and strengthen it without being noticed by man, impetuously, [as a
result of] an unconscious striving for it.  Human life, with all of its
variety, has become indivisible, unified.  An event, ocurring in a forsaken
corner on land or in the ocean, is reflected, and has consequences, major or
minor, in a multitude of other places, all over the Earth.  The telegraph,
telephone, radio, airplanes, aerostats\footnoteTransl{An aerostat is an object
that can stay stationary in air, bacause it is lighter than it, such as a
baloon or a dirigible.} encompass the globe.  Communication is ever easier and
faster.  Its organization increases, turbulently grows, every year.

We can clearly see that this is the beginning of a tempestuous movement, of a
natural phenomenon, which cannot be stopped by the accidents of human history.
Here the relation between historical processes and the paleontological history
of the manifestation of Homo sapiens is expressed, maybe for the first time.
That process---\emph{the complete colonization of the biosphere} by
mankind---arises from the course of the history of scientific thought, which is
inseparably connected with the speed of communication, with the achievements of
transportation technology, with the ability of thoughts to be communicated
\emph{instantaneously}, and to be discussed everywhere on the planet
simultaneously.

The fight, which is being carried out against this main historical current, is
forcing even its ideological opponents to obey it.  Government formations,
ideologically rejecting the equality and unity of all people, are attempting,
lacking no resources, to halt its impetuous manifestation; but it can hardly be
doubted that these utopian dreams would fail to last.  This transformation will
inevitably come to pass in the course of time, sooner or later, since the
creation of the noosphere out of the biosphere is a natural phenomenon,
fundamentally deeper and more powerful than human history.  It necessitates the
manifestation of mankind as a unified whole.  This is its inevitable
requirement.

Ours is a new stage in the history of the planet, which does not allow
comparison with past history without corrections.  It is so, because this stage
is creating fundamental \emph{novelty} in the history of the whole Earth, and
not just in the history of mankind.

Man has actually recognized for the first time that he is a citizen of the
\emph{planet} and that he can---must---think and act in a new aspect, not only
in the aspect of individual personalities, nuclear or extended families,
nations or their unions, but also in a \emph{planetary aspect}.  He, like
everything living, can think and act in a planetary aspect only in the region
of life---in \emph{the biosphere}, in a certain earth envelope, with which he
is inseparably and lawfully connected, and outside of which he cannot go.  His
existence is a function of it.  He carries it everywhere with himself.  And he
inevitably changes it lawfully and unceasingly.


\Section % 15
Simultaneously with mankind's complete envelopment of the surface of the
biosphere---with its complete colonization,---which is closely connected
with the achievements of scientfic thought, i.e. with the course of scientific
thought in time, a scientific generalization, which scientifically reveals the
character of the historical moment mankind is currently living through in a new
way, has been formed in \emph{geology}.

Mankind's geological role has been cast anew in the understanding of
geologists.  True, the recognition of the geological significance of our social
life has been expressed in a less clear form long ago, much earlier in the
history of scientific thought.  However, at the beginning of our century C.
Schuchert [1858--1942] in New
Haven,\footcite[80]{schuchert1933geology} and A.~P.\ Pavlov (1854--1929) in
Moscow\foreignlanguage{russian}{\footcite[с.~105 и
сл.]{pavlov1936geologicheskaya}} independently accounted, geologically anew,
for the long-known change which the emergence of human civilization introduces
into the environment, onto the face of the Earth.  They considered it possible
to take this manifestation of Homo sapiens as the basis for distinguishing
\emph{a new geological epoch,} along with the tectonic and orogenic data which
usually determine such divisions.

They correctly tried to split the Pleistocene Epoch, defining its end by the
beginning of the manifestation of mankind (during the recent hundred-somethng
thousand years---say a few decamyriads ago), and separating the latter in its
own geological epoch: \emph{psychozoic,} according to Schuchert;
\emph{anthropogenic,} according to A.~P.\ Pavlov.

Actually Ch.\ Schuchert and A.~P.\ Pavlov deepened and made more precise,
brought into the established in modern geology divisions of the history of the
Earth, a conculsion, which was made much before them, and which did not
contradict the empirical scientific work.  This conclusion was clearly
recognized by one of the creators of contemporary geology, L. Agassiz
(1807--1873), based on the paleontological history of \emph{life}.  He
established the special geological \emph{epoch of mankind} already in 1851.

However, Agassiz relied not on geological facts, but rather, to a great extent,
on the common religious conviction so strong during the age of natural science
before Darwin; he started from the special position of man in the
universe.\footnote{Agassiz expressed that idea in a polemical work directed
against Darwinism (\fullcite{agassiz1859essay}).  It is possible that this is
related to why the work did not reach, [despite] the many important reflections
in it, the influence it could have had.}

The geology in the middle of the 19th century, and the geology at the beginning
of the 20th century are incomparable in their power and scientific
justification, and the epoch of mankind of Agassiz is not scientifically
comparable with the epoch of Schuchert-Pavlov.

Already earlier, when geology was just being created and its basic concepts did
not yet exist, G.\ Buffon (1707--1788) notably expressed that same geological
epoch of mankind at the end of the 18th century.  He proceeded from the ideas
of the philosophy of the Enlightenment, advancing the significance of reason in
the conception of the universe.

The definite difference between these homonymous concepts is clear from the
fact that Agassiz assumed the geological age of the World to be the biblical
duration of the existence of the Earth---six--seven thousand years,---Buffon
thought about an age of more that 127 thousand years, Schuchert and Pavlov---of
more than a billion years.


\Section % 16
We have already met with similar conceptions in philosophy long ago.
Conceptions, which have been reached in another way---not by way of precise
scientific observation and experimentation, like that of C. Schuchert, A. P.
Pavlov, L. Agassiz (and J. Dana, who knew about the generalizations of
Agassiz), but by way of philosophical searches and intuition.

The philosophical worldview creates, in general, as well as in particular, that
environment, in which scientific thought takes place and develops.  To a
significant extent, it determines and gives rise to scientific thought, itself
being changed by its achievements.

The philosophers relied on free, it seemed to them, in their expression ideas,
on the searches of confused human thought, of human consciousness, which
wouldn't reconcile with reality.  However, man unavoidably built his ideal
world in the brutal framework of surrounding nature, the environment of his
life, the biosphere, with which he has a deep connection, independent of his
will, which he did not, and still does not, understand.

We find, in the history of philosophy, already many centuries before our age,
intuitions and constructs, which could be connected to scientific empirical
conclusions, if we translate the thoughts---intuitions---that have reached us
into the realm of real scientific facts of our time.  We lose their roots in
the past.  A few of the philosophical searches in India, many centuries
ago,---the philosophy of the Upanishads---can be interpreted in such a way, if
we translate them into the realm of 20th century science.\footnote{The
philosophy of The East, mainly of India, in connection with the new creative
work there, taking place under the influence of the introduction of Western
science in Indian culture, is of much greater interest for life sciences than
Western philosophy, which is deeply permeated---even in its materialistic
parts---by deep echoes of Judeo-Christian religious searches.}

Analogous conceptions existed in another, smaller, cultural area, partly
overlapping, but later, which was isolated from the Indian one for a
significant part of the time: in the circle of the Helenic Mediterranean
civilization.  We can trace the germs of these conceptions going back almost
two and a half thousand years ago.  The significance of science and scientists
for the government of the polis in political and social thought is clearly
manifested in Helenic thought, and is notably expressed in the concept of the
sate, [given by] Plato [427--347].

It cannot, it seems, be denied, but the condition of the sources, reaching us
in fragments, also does not allow us to confirm precisely, that after Aristotle
[384--322] these ideas were still alive during the Helenic age of Alexander the
Great [356--323], when, a few centuries after the destruction of the Persian
kingdom, a close exchange of ideas and knowledge between Helenic and Indian
civilization was established.  A connection between them and Chaldean
scientific thought, which went back a few millenia before Helenic and Indian
thought, was established at the same time.  The history of scientific work and
thought during this remarkable age is just beginning to come to light.

Better known is the influence of the Helenic political and social ideas.  We
can trace their historical influence exactly in the historical process of
modern science and of the civilization of the European West, which replaced the
theocratic ideological structure of the Middle Ages.  We can see their growth
in pactice, and with clarity only during the 16th--17th centuries, in the
conceptions and constructs of F. Bacon (1561--1626), who prominently advanced
the idea of the power of man over nature as the aim of modern science.

In the 18th century, in 1780, G.\ Buffon posed the manifestation of man's
control of nature \emph{as part of the history of the planet} not as an idea,
but as an observable natural phenomenon.  He relied on the hypothetical
reconstruction of the planet's past, connected with philosophical intuition and
theory, rather than on precisely observed facts---but he was looking for them.
His ideas were adopted by philosophical and political thought, and,
undoubtedly, exerted their influence on the course of scientific thought.
Geologists from the end of the 18th--beginning of the 19th century often relied
on them in their current scientific work.


\Section % 17
The scientific constructs of Schuchert and Pavlov and all the scientific
work which---to a significant degree unconsciously---preceded them are
essentially distinct from these philosophical constructs, which, however (this
can be established historically), undoubtedly influence the course of
geological thought, though unable to give it a firm basis.

It is clear from the generalizations of Schuchert and Pavlov that the main
influence of human thought as a geological factor is expressed in its
scientific manifestation: it mainly builds and guides the technical work of
mankind, which is transforming the biosphere.

Both of the indicated geologists were able to make their generalizations,
above all, because mankind was able to colonize the whole planet in their
time.  No organism except him, save for microscopic species and, possibly, a
few graminoids, has encompassed such an area in populating the planet.
However, mankind has accomplished this in a different way.  He thought
scientifically and transformed the biosphere through labor, adapted it to
himself and himself created the conditions for the manifestation of his
characteristic biogeochemical energy of reproduction.  Such population of the
whole planet became clear at the beginning of the 20th century, and it could
be considered a fact since about the first quarter of that century, which is
being confirmed every year in front of our eyes.  It became possible only
thanks to the drastic change of the conditions of life connected with the
emergence of a new ideology, with the drastic change in the tasks of
government life, with the scientific growth of technology, which were being
carried out at the very same time.

As J. Ortega y Gasset\footnote{\fullcite{ortegaygasset1932revolt-p19}}
correctly remarked, the 19th century in Europe, and over the whole world since
its second half, was a historical period when the significance of the vital
interests of the masses of population occupied first place in practice and
ideology in their consciousness and in the consciousness of government people
for the first time in wold history.  It was dramatically manifested in everyday
life for the first time.  A new ideology was based on the consciousness of the
population masses stepping onto the historical stage as a social force for the
first time.  It is beginning to encompass all mankind---every language without
exception---at a rapidly increasing rate.

It will show in its real significance only in the course of time.

The social-political ideological shift was dramatically manifested in the 20th
century mainly thanks to scientific work, thanks to the scientific
determination and clarification of the social tasks of mankind, and of the
form of his organization.


\Section % 18
The question of the better organization of life and of the means by which
it could be accomplished has been raised numerous times during the
multi-thousand-year historical tragedy full of blood, suffering, crime,
destitution, hardship, which we call world history.  Man has not accepted the
conditions of his life.

The exit from these searches has been resolved differently, and we can see
numerous (and how many have disappeared without trace!)
searches---philosophical, religious, artistic and scientific.  For millenia
they have been, and are being created in every corner where human society has
existed.

The world history of mankind has been lived and recreated for a significant
part of the human population, and the places and times full of suffering,
evil, slaughter, hunger, and destitution for the majority have been an
unsolvable mystery from a \emph{human} point of view of sensibility and
goodness.  In general, innumerable philosophical and religious attempts during
the course of millenia have not reached a unified explanation.

All solutions reached in such a way transfer and have transferred the question
in a different plane---from the domain of brutal reality, into the domain of
ideal constructs. Various forms of countless religious-philosophcal solutions,
which are indeed related to the notion of individual immortality, in one or
another form, in the literal meaning of the word, or in its future resurrection
in new conditions, where evil, suffering and disasters would not exist, or
where these would be distributed justly, have been found.  The notion of
metempsychosis, solving the question not from a personal standpoint, but from
the standpoint of all living matter, is the deepest.  It, having emerged a few
millenia ago, is still alive and vivid for many hundreds of millions of people
to this day.  And there is, perhaps, nothing it contradicts contemporary
scientific notions in.  The course of scientifc thought has nowhere run up
against the conclusions from this notion.

All of these notions---with all of their distance, sometimes, from precise
scientific knowledge---are a powerful social factor over the course of
millenia, strongly reflected in the process of the evolution of the biosphere
into the noosphere, far from being, however, decisive, or somehow distinguished
from other factors in its creation at the same time.  In the course of tens of
thousands of years, they have, in this aspect, sometimes played the main role,
have sometimes disappeared among others, have moved into the background, could
have been left neglected.


\Section % 19
Because this same historical process of world history is reflected in the
nature surrounding man in another way[;]\footnoteTransl{%
	The source Russian version has a full stop here.%
} it is possible and necessary to approach it purely scientifically, leaving
aside any notions which do not result from scientific facts.

Archeologists, geologists, and biologists are now having such an approach to
the study of world history, leaving without consideration of any of the
millenia-old notions of philosophy and religion, not taking them into account,
creating a new scientific understanding of the historical process of man's
life.  Geologists, deepening the study of the history of our planet, of the
Pleistocene, of the Ice Age, have collected a vast amount of scientific facts,
manifesting the reflection of the life of human societies---in the end, of
civilized mankind---on the geological processes of our planet, in fact, of the
biosphere.  Without its evaluation from the standpoint of good and evil,
without regard for the ethical or philosophical aspect, scientific work,
scientific thought is establishing a new fact of primary geological
significance in the history of the planet.  This fact consists of the detection
of a new\emph{ Psychozoic }or\emph{ Anthropogenic geological Age, } created by
the historical process.  In fact, it is defined planetologically by the
emergence of mankind.

None of the countless---geological, philosophical, or religious---notions of the
significance of mankind, and the significance of human history play any
[significant] role in this scientific generalization.  They can be left aside
without any concerns.  Science does not have to take them into account.


\Section % 20
Approaching the analysis of this scientific generalization, we should note that
its duration can be estimated as millions of years, while the historical
process of human societies encompasses a few decamyriads, hundreds of thousands
of years, of it.

It is necessary, most of all, to stress a few preconditions, which determine
this generalization.

First, is \emph{the unity and equality, in essence, in principle, of all
people}, of all races.  This is expressed biologically in the detection of all
people in the geological process \emph{as a unified whole} with respect to the
rest of the living population of the planet.

And this is despite the possibility, and, even, probability of the emergence of
the different human races from different species of the genus \lphr{Homo}.
This difference likely does not reach deeper, to the various animal
predecessors of the genus \lphr{Homo}.  We cannot, however, deny it.  Such
unity with respect to all other life has been, in general, maintained
throughout all of world history, even though it was absent, or almost absent at
times, and in places in special cases.  We are encountering such manifestations
still today, but the general tempestuous process is not changed by this.

The geological significance of mankind was manifested for the first time in
connection with this.  Apparently, already hundreds of millenia ago, when man
acquired control over fire and began making the first instruments, he laid the
foundation of his advantage over the higher animals, the fight with which
occupied a major part of his history, and was, theoretically, finally ended a
few centuries ago with the discovery of firearms.  Man must take special care
in the 20\textsuperscript{th}$\;$c.\ not to allow the extinction of all
animals---large mammals and reptiles,---which he would like to preserve because
of some or other considerations.  Many tens of millenia earlier, however, close
to his emergence, he was that force, new on our planet, which occupied an
important place along with other earlier species, in bringing the extinction of
species of large animals.  It is quite possible that he did not differ much
from numerous other gregarious predators at that time.


\Section % 21
Much more important, from a geological point of view, was another shift, slowly
taking place tens of thousands of years ago---the domestication of herd animals
and the cultivation of cultured plant races.  Man started changind the living
world around him, and creating for himself a new, previously non-existent
living nature by this means.  The great significance of this was manifested in
another way---in the fact that he saved himself from hunger in a new way, known
to only a limited degree among animals,---the conscious, creative safeguard
against hunger---and, consequently, created the possibility for his unlimited
reproduction.

At that time, perhaps, ten--twenty thousand years ago, thanks to this
possibility, the possibility for the formation of large settlements (towns and
villages), and, consequently, the possibility for the formation of government
structures, completely essentially different from those special forms which
arise from blood relations, was first established.  The idea of the unity of
mankind received here in reality, although, obviously, unconsciously, even
greater possibilities for its development.

Thanks to the discovery of fire, man was able to survive the Ice Age---those
great changes and variations of the climate and the state of the biosphere,
which are now being scientifically uncovered before us in the alterations with
the so-called interglacial periods---at least, three in number---in the
Northern Hemisphere.  He survived them, even though numerous other lage mammals
disappeared then from the face of the Earth.  It is possible that he aided
their extinction.

The Ice Age has not ended, and extends to the present time.  We are living in
an interglacial period---the warming is still continuing,---but man has adapted
to these conditions so well that he does not notice the Ice Age.  The
Scandinavian Glacier thawed in the location of St.\ Petersburg and Moscow a few
thousand years ago when man had already developed domestication and
agriculture.\footnoteEd{
	The time of the maximum of the last glaciation is determined today to
	be 18--20 thousand years ago by the method of carbon dating.  It did
	not reach Moscow, but only the Valdai Hills; the ice cover thawed about
	10--12 thousand years ago in the outskirts of Leningrad.
	\parenNoteAuth{Ed.}
}

Hundreds of thousands of generations passed in the history of mankind during
the Ice Age.

However, we can hardly doubt today that man (probably, not the genus Homo)
existed already much earlier---at latest, at the end of the Pliocene, a few
million years aro.  The Piltdown Man in Southern England at the end of the
Pliocene, morphologically different from contemporary man, already possessed
stone implements, and, obviously, unpreserved implements out of wood, and,
possibly, bone.  His brain apparatus was as developed as in contemporary
man.\footnoteEd{
	The skull from the Piltdown cave, constructed from fragmentary remains
	in 1912 by Charles Dawson, was fabricated either by him, or by other
	irresponsible anthropologists.  It is a skull of an entirely
	contemporary person with jaws of a hominid ape.
	(\cite{howell1965early})
	\parenNoteAuth{Ed.}
}
The Sinanthropus of Northern China, living, apparently, at the beginning of the
Post-Pliocene in an area where the glacier, apparently, did not reach,
controlled fire and possessed implements.\footnoteEd{
	Sinanthropus lived 350--400 thousand years ago, \ie\ in the middle of
	the Pleistocene, somewhat later than V.\ I.\ Vernadsky thought.
	However, his supposition that the genus Homo existed already ``a few
	million years ago'', turned out to be correct.  The famous excavations
	of Dr.\ L.\ Leakey in the Olduvai Gorge on the border of Kenya and
	Tanzania, widely covered in scientific and popular science journals,
	showed that primitive man in Eastern Africa, classified as the peculiar
	species of Homo habilis (handyman), undoubtedly lived 1,800--1,900
	thousand years ago.  The later discoveries of R.\ Leakey on the eastern
	shore of Lake Rudolph led to the wide-spread oppinion that man lived
	already 3 million years ago in Eastern Africa, although that number is
	not credible, since the fragmentary remains of the skull were found in
	scree, and it is not known what layer they originate from.  The
	contemporary species Homo sapiens (wise man) emerged 40--45 thousand
	years ago not in Africa, but in the fairly northern latitudes of
	Europe and Asia, probably not without the influence of, and adaptation
	to the extreme conditionns of the Ice Age.  (See
	\cite{ivanova1965geologicheskiy}; also in German:
	\cite{ivanova1972geologische}.)
	\parenNoteAuth{Ed.}
}

It is possible that A.\ P.\ Pavlov was quite right when he supposed that the
Ice Age, the first glaciation of the Northern Hemisphere, began at the end of
the Pliocene, and at that time a new organism, possessing an exceptional
central nervous system, which led, in the end, to the development of cognition,
and is now being manifested in the transition of the \emph{biosphere into the
noosphere,} emerged in the conditions approaching the severe ones of
glaciation.

Apparently, all morphologically different types of man, the different genera
and species were already communicating with each other, were distinct from the
general mass of living matter from the beginning, possessed creative work of a
drastically different character than that of surrounding life, and could
interbreed with each other.  \emph{The unity of mankind was developing
tempestuously} in this way.  Apparently, Osborn\footnote{\cite{osborn1910age}}
was right that man on the border between the Pliocene and the Post-Pliocene,
still lacking permanent settlements, possessed great mobility, traveled from
place to place, was recognizing and manifesting his strong
distinctness---strove toward independence from his surroundings [environment].


\Section % 22
In reality, this \emph{unity} of mankind, his \emph{distinction from everything
living}, this new form of \emph{power of the living organism} over the
biosphere, his greater \emph{independence} from \emph{his conditions} than that
of all other organisms is the main factor which, in the end, emerged in the
geologically evolutionary process of the noosphere's creation.  The unity of
human societies, their intercommunication and their power---the striving for
the manifestation of power---over the environment were manifested tempestuously
over the course of many generations, before they were detected, and were
recognized ideologically.

Of course, this was not a conscious phenomenon; it formed in the struggle
during clashes; there were mutual exterminations of people, times of
cannibalism and hunting of each other, but, as a general rule, these three
expressions in fact of the future idea of mankind's unity, his drastic
distinction from everything living, and the striving to master the environment
penetrate and create all of mankind's history, at least during the last ten
thousand years.  They have prepared the new contemporary striving to understand
them ideologically as a basis of human life.

We can trace its existence in the ideological aspect with scientific precision
in reality only over ten thousand years, maximum.  But then, we cannot reach
farther than four thousand years in written sources, since writing signs do not
reach much farther, and the system of alphabetic characters hardly goes beyond
three thousand years ago.  We can expect the most ancient sources of real
echoes of ideological constructs at most a thousand years before the discovery
of ideographic writing.  Consequently, we could hardly reach much earlier than
six thousand years ago in preserved legendary works, taking into account the
presently unusual oral ability to pass ideological constructs, formed by the
peculiar civilizations of that time, through generations.  The latest
archaeological discoveries reveal the unexpected fact that civilized city life,
the conditions of cultural city life customary in our everyday experience, the
peaceful trade and technology of life with achievements considered impossible
earlier, after oblivion and after millennia, were sometimes discovered anew;
they allow us to think that complex civilized city life has existed for a long
time---perhaps, millennia---before six thousand years ago.  All of these
achievements have been disseminated over the course of millennia by complex
pathways in all continents, including, apparently, the New World in some
period.  From mankind's point of view the New World was not new, and the
culture, even the scientific one, of its states at the end of the XV--beginning
of the XVI century---the time of its discovery by Western European
civilization---was not lower, but, in some respects, even higher than the
scientific knowledge of Western Europeans.  It suffered defeat only as a
consequence of the fact that military technology, firearms were unknown in
America, but became common in the life of Western Europeans a few decades
before the discovery of America.

The picture of the multi-millennial history of the material interaction of
civilizations, separate historical centers through Eurasia, parts of Africa,
from the Atlantic to the Pacific and Indian Ocean, at times---with
multi-century breaks---spreading through the oceans, is being clarified.  It is
exceptionally common that cultural centers were located in few places.  The
most ancient are: the Chaldean interfluvial area established by Breasted, the
Nile plains, Egypt, and pre-Arian Northern India.  They were all in
multi-millennial contact.  Not much later, no more than three thousand years
ago, the Northern Chinese center emerges.  But the scientific research here has
been going on for only the last three--four years, and has been obstructed by
the savage Japanese invasion.  There may be unexpected finds here.  Apparently,
a temporary center existed on the shore of the Pacific Ocean---in Korea, or
China---and on the shore of the Indian Ocean---in Annam,---whose role is still
completely unclear, and great discoveries are possible.


\Section % 23
A deep transformation of thought in the areas of religion, art, and philosophy
occurred `simultaneously,' say, two and a half thousand years ago in various
cultural centers: in Iran, in China, in Aryan India, in the Hellenic
Mediterranean (current Italy), great creators of religous systems
emerged---Zoroaster, Pythagoras, Confucius, Buddha, Lao-tzu, Mahavira,---who
have encompassed in their influence, continuing to this day, millions of
people.

\emph{The idea of the unity of mankind,} of people as brothers, left the limits
of individuals approaching it in their intuitions or inspirations for the first
time, and became an engine of the life and everyday activity of the masses of
peoples, and a goal of national education.  It has not left the field of
mankind's history since then, but is, nevertheless, still far from its
realization.  Slowly, with intermissions of many centuries, the conditions
allowing its realization, its actual incorporation into life, are being
created.

It is important and characteristic that these ideas belonged to the framework
of those everyday real phenomena, which have been created unconsiously in daily
life, outside man's will.  In them was manifested the influence of the
individual, an influence thanks to which, by organizing popular masses, this
idea can affect the surrounding biosphere and manifest itself tempestuously in
it.

Earlier, it was manifested in poetically inspired work, from which emanated
religion, philosophy, and science, all of which are social manifestations of
it.  The leading religous ideas, evidently, preceded the philosophical
intuitions and generalizations by many centuries, if not millenia.

The biosphere of the XX century is turning into the noosphere, which is being
created primarily by the growth of science, the scientific understanding and
the social labor of mankind based on it.  I shall return to the analysis of the
noosphere below, in the further exposition.  At present, it is necessary to
emphasize the unbreakable connection between its creation and the growth of
scientific thought, which is the first necessary precondition for this
creation.  The noosphere can be created only under this condition.


\Section % 24
And in our own time, since the beginning of the XX c., an exceptional
phenomenon is being observed in the course of scientific thought.  Its rate is
emerging as completely unusual, unprecedented in the course of many centuries.
Eleven years ago I equated it to an explosion---\emph{an explosion of
scientific creative work.}\footnote{
	\cite{vernadsky1927mysli}.  A report read at the first assembly of the
	Committee on the History of Science\footnoteRus{
		Комиссия по истории знаний
	},
	October 14, 1926.
}
I can still affirm this now, even more strongly and definitively.

We, in the XX c., are living through a time in the course of scientific
knowledge, in the course of scientific work in the history of mankind, whose
equal in significance we can find only in its remote past.

Unfortunately, the conditions of the history of scientific knowledge do not
allow us currently to make basic logical conclusions precisely and definitely
from this empirical case.  We can only affirm it as a fact and express it in a
geological aspect.

The history of scientific knowledge is a history of the creation of a new main
geological factor in the biosphere---its state of
organization,\footnoteTransl{
	\ie of being (constantly) organized.
}
identified tempestuously in the recent millenia.  It is not random, but lawful,
as the paleontological process is lawful in the course of time.

The history of scientific thought is still unwritten, and we are just barely
beginning---with great labor and big problems---to detect forgotten and
consciously unrecognized by mankind facts in it---we are beginning to look for
the major empirical generalizations characterizing it.

We are still unable to scientifically \emph{understand} this large-scale
phenomenon of great scientific and social importance.  To scientifically
\emph{understand} means to establish the phenomenon in the framework of
scientific reality---of the cosmos.  \emph{We} must currently simultaneously
\emph{strive to scientifically understand} it and use its study for
reaching the main milestones of \emph{the history of scientific thought}---one
of mankind's most vitally important scientific disciplines.

We are living through a fundamental turning point in the scientific worldview,
occurring during the life of the presently living generations, we are living
through the creation of vast new areas of knowledge, expanding the
scientifically encompassed cosmos from the end of the past century, both in its
spatial, and in its temporal extent, beyond recognition, we are living through
a shift in the scientific methodology, taking place with a speed which we would
look for in vain in the preserved chronicles and in the records of world
science.  New methodologies of scientific work, and new areas of knowledge, new
sciences, discovering before us millions of scientific facts and millions of
scientific phenomena, whose existence we did not suspect only yesterday, are
being created at an ever increasing rate.  The individual scientist can follow
the course of scientific knowledge only laboriously, and incompletely, as never
before.

Science is being reformed before our eyes.

But, even more importantly, the impact of science, constantly increasing, on
our life, on our living and dead---inert---environment is being revealed, it
seems to me, with a stunning clarity.  Science and the scientific thought
creating it are identifynig their other, foreign to us, planetary character, in
this \emph{XX-c.\ growth of science we are living through, in this social
phenomenon} of mankind's history of deep significance.  Science is being
revealed to us a new way in it.

We can study this phenomenon we are living through---study it
scientficially---from two points of view.  On the one hand, as one of the main
phenomena of the history of scientific thought, and on the other, as a
manifestation of the structure of the biosphere, revealing to us new major
characteristics of its state of organization.\footnoteTransl{
	or state of being (constantly) organized.
}
The close and unbreakable connection between these manifestations has never
stood before mankind with such clarity.

We live in an era when this aspect of the course of scientific thought is being
identified before us with an exceptional clarity---the course of the history of
scientific thougth is emerging before us as a natural process in the history of
the biosphere.

The historical process---the manifestation of the global history of
mankind---is being revealed before us in one---but a main one---of its
consequences as a natural phenomenon of immense geological significance.

This was not taken into account in the history of scientific thought as an
inseparable from it, main characteristic.


\Section % 25
So far the history of mankind and the history of its spiritual manifestations
has been studied as a self-contained phenomenon, manifested freely and without
lawfulness on the Earth's surface, in its environment, as something foreign to
it.  The social forces manifested in it are considered as largely independent
from the enviorment in which mankind's history is occurring.

Even though many different attempts to connect mankind's spiritual
manifestations and mankind's history in general with the environment in which
they are taking place exist, it is always missed that, first, this
environment---the biosphere---has a completely definite structure, determining
\emph{everything, without exception, occuring in it, }unable to be
fundamentally disrupted by the processes occurring in it, that it has, as a
natural phenomenon, its lawful transformations in space-time.

The explosion in scientific work is occurring, and plays a part in creating, to
a certain degree creates the transformation of the biosphere into the
noosphere.  Nevertheless, man oneself, both in one's individual, and in one's
social manifestation, is most closely lawfully, materially-energetically
connected with the biosphere; this connection never ceases, as long as one
exists, and has no significant difference from that of other biospheric
phenomena.


\Section % 26
Let us consider the following scientific-empirical generalizations.

\begin{enumerate}
  \item Man, as he is observed in nature, like all living organisms, like all
  living matter, is a definite \emph{function of the biosphere }in its definite
  space-time.

  \item Man, in all of his manifestations, comprises a definite, lawful part of
  the structure of the biosphere.

  \item The ``explosion'' of scientific thought in the XX century \emph{has
  been prepared by the whole foregoing biosphere }and has deep roots in its
  structure---it cannot cease and turn back.  It can only be slowed down in its
  rate.  The noosphere---the biosphere transformed by scientific thought,
  prepared by processes occuring for hundreds of millions, perhaps billions, of
  years, which created Homo sapiens faber---\emph{is not a short-term and
  passing geological phenomenon. }Processes prepared for many billions of years
  could not be passing, and could not cease.  It follows, thence, that the
  biosphere will unavoidably transform, in one way or another,---sooner or
  later---into the noosphere, \ie\ that events necessary therefore, but not
  contradicting this process, will occur in the histories of peoples populating
  it.
\end{enumerate}

The civilization of ``cultural mankind''---as far as it is a form of
organization of a new geological force emerging in the biosphere---\emph{cannot
be broken off and destroyed, }since it is a great natural phenomenon,
corresponding historically, or rather geologically, to the state of
organization developed in the biosphere.  Forming the noosphere, it is
connected with all of its roots to that earthly envelope, which earlier did not
exist to any comparable degree in mankind's history.


\Section % 27
All of the past historical experience and the events of the moment we are
currently living through seem to contradict this.

I cannot continue without pausing, even shortly, here.  It seems to me that the
creation of the noosphere by human thought and labor, which has now begun,
changes all the circumstances of mankind's history, does not allow us to simply
compare the past with the present, as was admissible earlier.

Everyone is familiar with numerous, not only extensive interruptions in the
growth of scientific thought, but also the loss and destruction of scientific
achievements, reached over many earlier centuries.  We can see times of sharply
expressed regression, which encompassed great territories and physically
destroyed whole civilizations that did not contain in themselves unavoidable
reasons for it.  The processes connected with the destruction of the
Greco-Roman civilization held back mankind's scientific work for many
centuries, and much that was achieved before was lost for long, and, often,
forever.  We can see the same in the ancient civilizations of India and the Far
East.

The fears and alarm of such a forceful breakdown in our time, after the world
war of 1914--1918, one of the greatest manifestations of mankind's barbarism,
that have spread among a wide circle of thinkers, therefore, seem
understandable and unavoidable.  Government forces, as we can now clearly see,
did not prove to be up to the situation after the war's dying out, and we have
been living through the consequences of the unstable situation over the last 20
years, connected with a deep moral breaking point---a consequence of the world
slaughter house, the pointless death of more than ten million people over four
years, and countless losses of peoples' labor.  Twenty years after the end of
the war, we face today the danger of a new---even more barbaric and more
pointless---war.  Now, not only in fact, but also ideologically, the method of
war is the extermination of not only the armed participants, but also the
peaceful population, including old men, old women, and children.  That which
had remained in the past like an ideal, and was morally unacceptable has now
become a cruel reality.


\Section % 28
As a result of the war of 1914--1918, leading to the breakdown of the most
powerful nations with centuries-old traditions, nations which were the least
democratic in their centuries-old ideals, the least free---the anchors of the
old traditions in Europe---a fundamental reconsideration of values occurred.
The idea of the ``equality'' of all people, expressed in the specific framework
of Christian religion, lied at the foundation of these nations.  It was the
basis of Christian morals.  However, reality never corresponded to this basic
principle of Christianity (and, ever more, of Islam), even though it was loudly
proclaimed everywhere in the Christian nations, and was---at face value---the
basis of national morals.  Something completely different occured in reality,
and the Christian nations of the white races carried out, practically, all of
the colonial politics in the course of centuries, acknowledging equality in
words, they mercilessly opressed, exterminated, and exploited peoples and
nations of the non-white races.  The war of 1914--1918 stirred up the whole
world, and uncovered the radical contradiction between words and deeds in front
of all, raised the power and significance of the non-white races.

This did not concern the moral significance of Islam and Buddhism, since in
them---in the actual politics of nations where they were preached---there did
not exist the contradiction which was in Christian nations.  These religions
observed the equality of all people of the same religion in national activity.

. . .

\Section % 29

. . .

\Section % 30

. . .

\Section % 31

. . .

\Section % 32

. . .

\Section % 33

. . .

\Section % 34
Science, therefore, is by no means a logical construct, a truth-seeking
apparatus.  Scientific truth can never be known by logic, but rather only by
living.  \emph{Action }is a characteristic of scientific thought.  Scientific
thought---scientific work---scientific knowledge occurs in the thick of life,
from which it is inseparable, and by its very existence gives rise to its own
active manifestations, which themselves are not only means of disseminating
scientific knowledge, but also create the countless forms of its detection,
give rise to countless major and minor sources of the growth of scientific
knowledge.

The human individual, even in the time of our state of organization of science,
is, thus, far from always the creator of scientific ideas and scientific
knowledge; the research scientist, living a life of purely scientific work, of
large or small extent, is \emph{one} of the creators of scientific knowledge.
Individual people, connected to scientifically important, but often foreign to
science, considerations, revealing scientific facts and scientific
generalizations, sometimes fundamental and decisive, hypotheses and theories
widely used in science, come forward accidentally, \ie\ by the means of
everyday life, out of the thick of life along with the scientist.

Such scientific work and scientific searches, proceeding from actions outside
the scientific, consciously organized work of mankind, is the active-scientific
manifestation of the living of the human cognitive environment at a given time,
a manifestation of life's scientific environment.  The part of the scientific
structure of the new scientific thought, introduced in science in this way, is
by its mass, and by its importance for the outcome of history comparable, it
seems to me, to what is introduced into science by the scientists consciously
working on it, to what is revealed by the consciously organized scientific
work.  Without the simultaneous existence of scientific organization and a
scientific environment, this ubiquitous form of mankind's scientific work,
tempestuously unconscious, disappears and is forgotten to a large extent, as
this occurred in the regions of Mediterranean civilization over the course of
long centuries in the Christianized Roman Empire, in Persian, Arabic, Berber,
Germanic, Slavic, and Celtic societies of Western Europe, in connection with
the national breakdown of the government formations existing in them during the
$4^\mathrm{th}$--$12^\mathrm{th}$ c.\ AD, and, often, later.  Science loses its
achievements in the course of time, and tempestuously comes back to them.

The history of science, and the history of mankind, reveals such events at
every step.  The flourishing of Hellenic science left aside, and did not make
use of, used late (after millennia) such achievements of everyday Chaldean
science, as, for example, Babylonian algebra.


\Section % 35
This means---the introduction of scientific discoveries, foreign to \emph{the
scientific searches of the individual personality}, to which life gives rise
everywhere, and their incorporation into the organized manifestation of the
scientific work of scientists, the scientific apparatus of a given
time,---however, is not the only means by which the living environment impacts
science.

This, in and of itself collective, \altStylePhr{and,} from a scientific point
of view, unconscious work,\footnote{%
	Uncoscious in the sense that the scientific result, or phenomenon of
	life, which creates the scientifically important or necessary fact (or
	generalization), did not have \emph{that goal} at its creation or
	manifestation.
} \emph{in the course of historical time} and through the
changes occurring in this way, creates the new and important, which can be
registered and can become the result of scientific achievements of primary
importance, as, for example, were the circumnavigation of the Earth, the
discovery of America, the fall of the Persian Kingdom (destroyed by Alexander
the Great), as well as the Chinese kingdoms and Central Asian cultural centers,
the defeat of Genghis Khan, the victory of the Christian church and religion,
the emergence of Mohammedanism and its religious-political identification, as
well as other major and minor events of political life.

No less, but, often, rather more powerful have been those changes, which have
occurred in economic life, in agriculture, or in individual manifestations of
success in everyday life, like, for example, the introduction of the camel
(dromedary) in the desert and semidesert areas of Northern Africa,\footnote{%
	\cite[p.~178]{julien1931histoire}.  See
	\cite{gsell1926memoires};\footnoteTransl{%
		According to Wikipedia[!] the Académie des Inscriptions et
		Belles-Lettres was founded by Jean-Baptiste Colbert, and Jean
		Sylvain Bailly was its member.
	} \cite[p.~181]{gautier1927siecles} for the significance of this
	phenomenon.
} and the discovery of printing in the Rhenish countries in Europe.\footnote{%
	We must never forget that the printing press was discovered in Korea a
	few centuries before Coster and Gutenberg, and was widely used in the
	Chinese kingdom.  There, however, the factor which gave it a living
	power did not exist: active scientific work was lacking in Korea and
	China at the time.
}

Along with these tempestuous phenomena, whose consequences for scientific
thought were not considered at their creation by mankind, to an equal, and
sometimes, perhaps, greater degree, scientific thought itself---the scientific
discoveries of individual thinkers and scientists, which change mankind's
world-view, like Copernicus, Newton, Linnaeus, Darwin, Pasteur, P.\ Curie---is
acting in the biosphere.  In some cases this was done consciously, in
others---unexpectedly for the scientist oneself, as occurred before our
eyes with A.\ Becquerel [1852--1908], discovering radioactivity in
1896,\footnote{%
	Becquerel himself thought that he took up Uranium only because it was
	studied by his father and grandfather (\autoref{sec:55}).
} or with H.\ Ørsted [1777--1851], detecting electromagnetism,\footnote{%
	Ørsted discovered electromagnetism in 1820.
	(\cite{oersted1920discovery}.)
} or with L.\ Galvani [1737--1798], discovering the galvanic
current.\footnote{%
	The phenomenon discovered by Galvani was correctly explained by Volta.
	Galvani's explanation was incorrect, but ``galvanism,'' with
	incalculable consequences before the study of electricity, was
	discovered by him.  (See \cite{alibert1801eloge} about him.)
}

Maxwell, Lavoisier, Ampere, Faraday, Darwin, Dokuchaev, Mendeleev and many
others encompassed great scientific revelations, worked creatively to bring
them into being in full consciousness of their fundamental significance for
life, but unexpected for their contemporaries.\footnote{%
	It is interesting that the significance of these discoveries in their
	application to life was admitted decades after the deaths of Maxwell,
	Lavoisier, Faraday, Mendeleev, \altStylePhr{and} Ampere.
}

Their thought---consciously for them---influenced the thick of life; here the
applied creations arising in this way, in a new form, unexpectedly and
unsurmisedly for their contemporaries, often after the deaths of their
creators, were reflected anew in scientific work, overturned mankind's everyday
life, \altStylePhr{and} created new, unexpected sources of scientific
knowledge.

Along with them, in the same way, through the thick of life, through the
environment, inventors, among them, often, people with little scientific
literacy---from all social classes and circles, often people having no
connection with or interest in the search for scientific truth,---are creating
a new, analogous cycle of scientific problems.\footnote{%
	R.\ Arkwright\dots\ [Arkwright, Richard (1732--1792)---English
	mechanic, inventor of the spinning frame. \noteAuth{Ed.}]; Zénobe
	Théophile Gramme\dots\ [Gramme (1826--1901)---Belgian electrical
	engineer, one of the inventors of the dynamo. \noteAuth{Ed.}]
}


\Section % 36
From everything said so far we can see that \emph{it is possible to make}
conclusions of great scientific significance, namely:
\begin{enumerate}
  \item The course of scientific work is that force by which man changes the
  	biosphere in which he lives.
  \item This manifestation of the biosphere's changing is an inevitable,
  	concomitant phenomenon to the growth of scientific thought.
  \item This change of the biosphere occurs independently of human will,
  	tempestuously, as a naturally-occurring phenomenon.
  \item And since the environment of life is an organized envelope of the
  	planet---the biosphere,---the introduction, in the course of its
	geologically long existence, of a new factor of change---the scientific
	work of mankind---in it is the natural process of the transition of the
	biosphere into a new phase, into a new state---into the noosphere.
  \item We can see this more clearly in the historical moment we are living
  	through than could be seen earlier.  ``Nature's law'' is being revealed
	before us now.  New sciences---geochemistry and biogeochemistry---are
	making the expression of a few important characteristics of the process
	mathematically possible.
\end{enumerate}


\Section % 37

. . .

\Section % 38

. . .

\Section % 39

. . .

\Section % 40

. . .

\Section % 41

. . .

\Section % 42

. . .

\Section % 43

. . .

\Section % 44

. . .

\Section % 45

. . .

\Section % 46

. . .
