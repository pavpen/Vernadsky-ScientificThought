\Chapter{%
Manifestation of the historical moment mankind is currently living through as a
geological process.  Evolution of the species of living matter and evolution of
the biosphere into the noosphere.  This evolution cannot be stopped by the
course of global human history.  Scientific thought and mankind's daily lives
as expressions of it.}

\Section % 14
We are not yet conscious of, we are not yet living the realization of the
full consequences of the astonishing, unprecedented times that mankind has
entered during the 20th century.

We are living at the threshold of an extremely important, fundamentally new
epoch in the existence of mankind, in mankind's history on our planet.

Mankind has, for the first time, encompassed the whole surface envelope of the
planet---the whole biosphere, all parts of the planet connected to life---with
human life, with human culture.

We are present at, and are actively participating in the creation of a new
\emph{geological factor} in the biosphere, unprecedented in its power and in
its unity.

It has been scientifically established for the last 20--30 thousand years, but
has been clearly manifested at an ever increasing rate only during the last
millenium.

The envelopment of the whole surface of the biosphere by a unified social
species of the animal kingdom---by \emph{mankind}---has been completed after
many hundreds of thousands of years of unstoppable, tempestuous striving for
it.  There is no corner on Earth inaccessible to mankind.  There is no limit to
our possible population growth.  Man, through scientific thought and through
his life, socially organized into states, and guided by technology, is creating
a new \emph{biogenic force} in the biosphere, which is guiding his population
growth and creating favorable conditions for his population in parts of the
biosphere, earlier impenetrable to human life, and even in places where there
was no life before.

Theoretically, we cannot foresee a limit to mankind's potential, if we only
take into account the effect of generations; every geological factor is fully
manifested in the biosphere only in the effect of generations of living beings,
only in geological time.  With the rapidly increasing precision of scientific
work---in this case, of the methodology of scientific observation,---we can now
clearly establish, and study the increase of this new, principally currently
emerging, geological force in historical time.

Mankind is a unified whole, and even if that is recognized by the vast
majority, this unity manifests itself in forms of human life, which actually
deepen and strengthen it without being noticed by man, impetuously, [as a
result of] an unconscious striving for it.  Human life, with all of its
variety, has become indivisible, unified.  An event, ocurring in a forsaken
corner on land or in the ocean, is reflected, and has consequences, major or
minor, in a multitude of other places, all over the Earth.  The telegraph,
telephone, radio, airplanes, aerostats\footnoteTransl{An aerostat is an object
that can stay stationary in air, bacause it is lighter than it, such as a
baloon or a dirigible.} encompass the globe.  Communication is ever easier and
faster.  Its organization increases, turbulently grows, every year.

We can clearly see that this is the beginning of a tempestuous movement, of a
natural phenomenon, which cannot be stopped by the accidents of human history.
Here the relation between historical processes and the paleontological history
of the manifestation of Homo sapiens is expressed, maybe for the first time.
That process---\emph{the complete colonization of the biosphere} by
mankind---arises from the course of the history of scientific thought, which is
inseparably connected with the speed of communication, with the achievements of
transportation technology, with the ability of thoughts to be communicated
\emph{instantaneously}, and to be discussed everywhere on the planet
simultaneously.

The fight, which is being carried out against this main historical current, is
forcing even its ideological opponents to obey it.  Government formations,
ideologically rejecting the equality and unity of all people, are attempting,
lacking no resources, to halt its impetuous manifestation; but it can hardly be
doubted that these utopian dreams would fail to last.  This transformation will
inevitably come to pass in the course of time, sooner or later, since the
creation of the noosphere out of the biosphere is a natural phenomenon,
fundamentally deeper and more powerful than human history.  It necessitates the
manifestation of mankind as a unified whole.  This is its inevitable
requirement.

Ours is a new stage in the history of the planet, which does not allow
comparison with past history without corrections.  It is so, because this stage
is creating fundamental \emph{novelty} in the history of the whole Earth, and
not just in the history of mankind.

Man has actually recognized for the first time that he is a citizen of the
\emph{planet} and that he can---must---think and act in a new aspect, not only
in the aspect of individual personalities, nuclear or extended families,
nations or their unions, but also in a \emph{planetary aspect}.  He, like
everything living, can think and act in a planetary aspect only in the region
of life---in \emph{the biosphere}, in a certain earth envelope, with which he
is inseparably and lawfully connected, and outside of which he cannot go.  His
existence is a function of it.  He carries it everywhere with himself.  And he
inevitably changes it lawfully and unceasingly.


\Section % 15
Simultaneously with mankind's complete envelopment of the surface of the
biosphere---with its complete colonization,---which is closely connected
with the achievements of scientfic thought, i.e. with the course of scientific
thought in time, a scientific generalization, which scientifically reveals the
character of the historical moment mankind is currently living through in a new
way, has been formed in \emph{geology}.

Mankind's geological role has been cast anew in the understanding of
geologists.  True, the recognition of the geological significance of our social
life has been expressed in a less clear form long ago, much earlier in the
history of scientific thought.  However, at the beginning of our century C.
Schuchert [1858--1942] in New
Haven,\footcite[80]{schuchert1933geology} and A.~P.\ Pavlov (1854--1929) in
Moscow\foreignlanguage{russian}{\footcite[с.~105 и
сл.]{pavlov1936geologicheskaya}} independently accounted, geologically anew,
for the long-known change which the emergence of human civilization introduces
into the environment, onto the face of the Earth.  They considered it possible
to take this manifestation of Homo sapiens as the basis for distinguishing
\emph{a new geological epoch,} along with the tectonic and orogenic data which
usually determine such divisions.

They correctly tried to split the Pleistocene Epoch, defining its end by the
beginning of the manifestation of mankind (during the recent hundred-somethng
thousand years---say a few decamyriads ago), and separating the latter in its
own geological epoch: \emph{psychozoic,} according to Schuchert;
\emph{anthropogenic,} according to A.~P.\ Pavlov.

Actually Ch.\ Schuchert and A.~P.\ Pavlov deepened and made more precise,
brought into the established in modern geology divisions of the history of the
Earth, a conculsion, which was made much before them, and which did not
contradict the empirical scientific work.  This conclusion was clearly
recognized by one of the creators of contemporary geology, L. Agassiz
(1807--1873), based on the paleontological history of \emph{life}.  He
established the special geological \emph{epoch of mankind} already in 1851.

However, Agassiz relied not on geological facts, but rather, to a great extent,
on the common religious conviction so strong during the age of natural science
before Darwin; he started from the special position of man in the
universe.\footnote{Agassiz expressed that idea in a polemical work directed
against Darwinism (\fullcite{agassiz1859essay}).  It is possible that this is
related to why the work did not reach, [despite] the many important reflections
in it, the influence it could have had.}

The geology in the middle of the 19th century, and the geology at the beginning
of the 20th century are incomparable in their power and scientific
justification, and the epoch of mankind of Agassiz is not scientifically
comparable with the epoch of Schuchert-Pavlov.

Already earlier, when geology was just being created and its basic concepts did
not yet exist, G.\ Buffon (1707--1788) notably expressed that same geological
epoch of mankind at the end of the 18th century.  He proceeded from the ideas
of the philosophy of the Enlightenment, advancing the significance of reason in
the conception of the universe.

The definite difference between these homonymous concepts is clear from the
fact that Agassiz assumed the geological age of the World to be the biblical
duration of the existence of the Earth---six--seven thousand years,---Buffon
thought about an age of more that 127 thousand years, Schuchert and Pavlov---of
more than a billion years.


\Section % 16
We have already met with similar conceptions in philosophy long ago.
Conceptions, which have been reached in another way---not by way of precise
scientific observation and experimentation, like that of C. Schuchert, A. P.
Pavlov, L. Agassiz (and J. Dana, who knew about the generalizations of
Agassiz), but by way of philosophical searches and intuition.

The philosophical worldview creates, in general, as well as in particular, that
environment, in which scientific thought takes place and develops.  To a
significant extent, it determines and gives rise to scientific thought, itself
being changed by its achievements.

The philosophers relied on free, it seemed to them, in their expression ideas,
on the searches of confused human thought, of human consciousness, which
wouldn't reconcile with reality.  However, man unavoidably built his ideal
world in the brutal framework of surrounding nature, the environment of his
life, the biosphere, with which he has a deep connection, independent of his
will, which he did not, and still does not, understand.

We find, in the history of philosophy, already many centuries before our age,
intuitions and constructs, which could be connected to scientific empirical
conclusions, if we translate the thoughts---intuitions---that have reached us
into the realm of real scientific facts of our time.  We lose their roots in
the past.  A few of the philosophical searches in India, many centuries
ago,---the philosophy of the Upanishads---can be interpreted in such a way, if
we translate them into the realm of 20th century science.\footnote{The
philosophy of The East, mainly of India, in connection with the new creative
work there, taking place under the influence of the introduction of Western
science in Indian culture, is of much greater interest for life sciences than
Western philosophy, which is deeply permeated---even in its materialistic
parts---by deep echoes of Judeo-Christian religious searches.}

Analogous conceptions existed in another, smaller, cultural area, partly
overlapping, but later, which was isolated from the Indian one for a
significant part of the time: in the circle of the Helenic Mediterranean
civilization.  We can trace the germs of these conceptions going back almost
two and a half thousand years ago.  The significance of science and scientists
for the government of the polis in political and social thought is clearly
manifested in Helenic thought, and is notably expressed in the concept of the
sate, [given by] Plato [427--347].

It cannot, it seems, be denied, but the condition of the sources, reaching us
in fragments, also does not allow us to confirm precisely, that after Aristotle
[384--322] these ideas were still alive during the Helenic age of Alexander the
Great [356--323], when, a few centuries after the destruction of the Persian
kingdom, a close exchange of ideas and knowledge between Helenic and Indian
civilization was established.  A connection between them and Chaldean
scientific thought, which went back a few millenia before Helenic and Indian
thought, was established at the same time.  The history of scientific work and
thought during this remarkable age is just beginning to come to light.

Better known is the influence of the Helenic political and social ideas.  We
can trace their historical influence exactly in the historical process of
modern science and of the civilization of the European West, which replaced the
theocratic ideological structure of the Middle Ages.  We can see their growth
in pactice, and with clarity only during the 16th--17th centuries, in the
conceptions and constructs of F. Bacon (1561--1626), who prominently advanced
the idea of the power of man over nature as the aim of modern science.

In the 18th century, in 1780, G.\ Buffon posed the manifestation of man's
control of nature \emph{as part of the history of the planet} not as an idea,
but as an observable natural phenomenon.  He relied on the hypothetical
reconstruction of the planet's past, connected with philosophical intuition and
theory, rather than on precisely observed facts---but he was looking for them.
His ideas were adopted by philosophical and political thought, and,
undoubtedly, exerted their influence on the course of scientific thought.
Geologists from the end of the 18th--beginning of the 19th century often relied
on them in their current scientific work.


\Section % 17
The scientific constructs of Schuchert and Pavlov and all the scientific
work which---to a significant degree unconsciously---preceded them are
essentially distinct from these philosophical constructs, which, however (this
can be established historically), undoubtedly influence the course of
geological thought, though unable to give it a firm basis.

It is clear from the generalizations of Schuchert and Pavlov that the main
influence of human thought as a geological factor is expressed in its
scientific manifestation: it mainly builds and guides the technical work of
mankind, which is transforming the biosphere.

Both of the indicated geologists were able to make their generalizations,
above all, because mankind was able to colonize the whole planet in their
time.  No organism except him, save for microscopic species and, possibly, a
few graminoids, has encompassed such an area in populating the planet.
However, mankind has accomplished this in a different way.  He thought
scientifically and transformed the biosphere through labor, adapted it to
himself and himself created the conditions for the manifestation of his
characteristic biogeochemical energy of reproduction.  Such population of the
whole planet became clear at the beginning of the 20th century, and it could
be considered a fact since about the first quarter of that century, which is
being confirmed every year in front of our eyes.  It became possible only
thanks to the drastic change of the conditions of life connected with the
emergence of a new ideology, with the drastic change in the tasks of
government life, with the scientific growth of technology, which were being
carried out at the very same time.

As J. Ortega y Gasset\footnote{\fullcite{ortegaygasset1932revolt-p19}}
correctly remarked, the 19th century in Europe, and over the whole world since
its second half, was a historical period when the significance of the vital
interests of the masses of population occupied first place in practice and
ideology in their consciousness and in the consciousness of government people
for the first time in wold history.  It was dramatically manifested in everyday
life for the first time.  A new ideology was based on the consciousness of the
population masses stepping onto the historical stage as a social force for the
first time.  It is beginning to encompass all mankind---every language without
exception---at a rapidly increasing rate.

It will show in its real significance only in the course of time.

The social-political ideological shift was dramatically manifested in the 20th
century mainly thanks to scientific work, thanks to the scientific
determination and clarification of the social tasks of mankind, and of the
form of his organization.


\Section % 18
The question of the better organization of life and of the means by which
it could be accomplished has been raised numerous times during the
multi-thousand-year historical tragedy full of blood, suffering, crime,
destitution, hardship, which we call world history.  Man did not accept the
conditions of his life.

The exit from these searches has been resolved differently, and we can see
numerous (and how many have disappeared without trace!)
searches---philosophical, religious, artistic and scientific.  For millenia
they have been, and are being created in every corner where human society has
existed.

The world history of mankind has been lived and recreated for a significant
part of the human population, and the places and times full of suffering,
evil, slaughter, hunger, and destitution for the majority have been an
unsolvable mystery from a \emph{human} point of view of sensibility and
goodness.  In general, innumerable philosophical and religious attempts during
the course of millenia have not reached a unified explanation.

All solutions reached in such a way transfer and have transferred the question
in a different plane---from the domain of heard reality into the domain of
ideal constructs.

. . .


\Section % 19

. . .

\Section % 20

. . .

\Section % 21

. . .

\Section % 22

. . .

\Section % 23

. . .

\Section % 24

. . .

\Section % 25

. . .

\Section % 26

. . .

\Section % 27

. . .

\Section % 28

. . .

\Section % 29

. . .

\Section % 30

. . .

\Section % 31

. . .

\Section % 32

. . .

\Section % 33

. . .

\Section % 34

. . .

\Section % 35

. . .

\Section % 36

. . .

\Section % 37

. . .

\Section % 38

. . .

\Section % 39

. . .

\Section % 40

. . .

\Section % 41

. . .

\Section % 42

. . .

\Section % 43

. . .

\Section % 44

. . .

\Section % 45

. . .

\Section % 46

. . .
