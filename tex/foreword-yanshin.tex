\ChapterByLine{Preface and Remarks by A.~L.\ Yanshin~\dots}{%
A.~L.\ Yanshin\\
{\rm\normalsize Chairman of the Committe of the Academy of Sciences of the USSR
for the Exploitation of the Scientific Heritage of Academician V. I.
Vernadsky}\footnotemarkTransl
\\[1ex]
F.~T.\ Yanshina\\
{\rm\normalsize Director-founder of the museum home of Academician V.~I.\ 
Vernadsky}\\[4ex]
\begin{center}
	\textit{The electronic version of the preface and the remarks was
	prepared from the edition in the book
	\fullcite{vernadsky1991thought}.}
\end{center}
}%
\footnotetextTransl{\foreignlanguage{russian}{Комиссия по разработке научного
наследия академика В.~И.\ Вернадского}}%


\section*{Preface}

The name of Vladimir Ivanovich Vernadsky has become widely known in our
country.  There is nobody with even the slightest degree of education, who
hasn't read, if not Vernadsky's works, then, at least, numerous newspaper and
magazine articles about him and his work.

There is a Vernadsky Avenue in Moscow.  One of the largest institutes at the
Academy of Sciences of the USSR, the Institute of Geochemistry and Analytical
Chemistry,\footnoteRus{Институт геохимии и аналитической химии} bears his name.
There is a Committee for the Exploitation of the Scientific Heritage of
Academician V.~I.\ Vernadsky, which publishes its own circular, at the
Presidium of the Academy of Sciences of the USSR.  Branches of that Committee
work in Leningrad and in Kiev.  There have been grants under Vernadsky's name
established at Moscow, Leningrad, Kiev, and Simferopol University.  National
scientific centers for the study of the work of this prominent thinker and for
its application to the solution of contemporary problems exist in Odessa,
Rostov-na-Don, Erevan, Simferopol, Ivanov, and in other cities in the USSR, and
abroad---in Prague, Oldenburg and Berlin.\footnote{
	Also named after V. I.  Vernadsky are: the National Geological
	Museum,\footnoteRus{Государственный геологический музей} the National
	Public University of Biospheric Studies,\footnoteRus{Всесоюзный
	народный университет биосферных знаний} the Central Scientific Library
	of the AS UkrSSR,\footnoteRus{Центральная научная библиотека АН УССР}
	the Student Sociological Center ``Noosphere'',\footnoteRus{Студенческий
	социологический центр ``Ноосфера''} the peak in the basin of
	Podkamennaya Tunguska River, the crater on the dark side of the Moon,
	the peninsula in Eastern Antarctica near the Sea of
	Astronauts,\footnoteRus{Море Космонавтов} the forest on the island of
	Paramushir (Kuril Island), the subglacial forests in Eastern
	Antarctica, the underwater volcano in the Atlantic Ocean, the mine in
	the region of Lake Baikal, the mineral Vernadit,\footnoteRus{вернадит,
	$Mn^{4+}, Fe^{3+}, Ca, NaS(O,OH)_{2n}\cdot H_2O$} the diatomaceous
	algae, research vessel ``Academician Vernadsky'' of AS UkrSSR, the
	steamboat ``Geologist Vernadsy''\footnoteRus{Геолог Вернадский} of the
	Kama River Shipping company,\footnoteRus{Камское речное пароходство}
	the Vernadsky village near Simferopol, the Vernadsky railway station on
	the Kazan line, the subway stop ``Vernadsky Avenue'' in Moscow, the
	Biosphere Museum at the Leningrad branch of the Institute of the
	History of Natural Science and Technology of the AS USSR.  A V.~I.\ 
	Vernadsky monument has been erected in Kiev, a memorial plate is in
	place on the old building of Moscow State University M.~V.\ 
	Lomonosov,\footnoteRus{МГУ им.\ М.~В.\ Ломоносова} on Vernadsky Avenue
	in Moscow, on the building of Leningrad State
	University,\footnoteRus{Ленинградского государственного университета}
	as well as on the building of the Kiev State University T. G.
	Shevchenko.\footnoteRus{Киевского государственного университета им. Т.
	Г.  Шевченко}  V.~I.\ Vernadsky grants are awarded for exceptional
	scientific work in the areas of mineralogy, geochemistry and
	astrochemistry by the Academy of Sciences of the USSR and by the
	Academy of Sciences of the UkrSSR.  A golden medal named after him has
	been established by the Academy of Sciences of the USSR.}

V.~I.\ Vernadsky's 125th birthday was celebrated in March 1988 in our country,
as well as abroad (in Prague and in Berlin).

The celebrations srpead very widely.  An exhibition dedicated to his work was
opened on January 15, 1988 at the Exhibition of the Achievements of the
National Economy.\footnoteRus{ВДНХ, from выставка достижений народного
хозяйства}  Scientific symposia on different directions of V.~I.\ Vernadsky's
research took place successively in Leningrad, Kiev, and Moscow with the
participation of foreign scientists from March 3 to 11.  A commemorative
conference took place in Balshoy Theatre\footnoteRus{Большой театр} in Moscow
on his birthaday, March 12, with the participtation of public organizations.
Separate conferences and scientific sessions took place during the same days in
Ivanov, Odessa, Simferopol, Rostov-na-Don, Yerevan, Baku,
Almaty,\footnoteRus{Алма-Ате} Novosibirsk, Irkutsk, and in many other
scientific centers of the nation.  The proposal to create an International
V.~I.\ Vernadsky\footnoteRus{Международного фонда В.~И.\ Вернадского} Fund for
subsidizing the translation of his works in other languages, finding materials
about him in foreign archives, and the invitation of scientists from foreign
nations to the USSR for reports and lectures on the contemporary development of
scientific problems noted by V.~I.\ Vernadsky was accepted.\footnote{See the
information at the end of the book.}  Articles about him, and his multifaceted
scientific work have appeared in almost all Soviet and international newspapers
and magazines.

Publishing house \rtitle{Nauka}\footnoteRus{Наука} released 4 volumes of works
by V.~I.\ Vernadsky, as well as his \tciteo{vernadsky1988pisma}, including the
book \tciteo{vernadsky1988filosofskie}, in which the work \rtitle{Scientific
Thought as a Planetary Phenomenon} was republished as a first part, now
published with reconstructions of all those passages, abridged for its first
edition in 1977, according to the archived original manuscript, before the very
anniversary in February, 1988.  The book was released in a 20,000 run.  The
whole run was bought out during the very first days after its appearance in the
bookstores' windows.  A barrage of letters requesting the release of an
additional run of \tciteo{vernadsky1988filosofskie}, or at least of its first
part, was received at the Scientific-publishing council of the Academy of
Sciences of the USSR.\footnoteRus{Научно-издательский совет АН СССР}

The appearance of \tciteo{vernadsky1988filosofskie} in 1988 found a broad
positive response from the press.  For example, the article
\tciteo{lukyanov1988neizvestny} was published in the journal
\ctciteo{lukyanov1988neizvestny} from September 29, 1988, in which the author
F.~Lukyanov\footnoteRus{Ф. Лукьянов} wrote: \begin{quotation}
  The name of academician Vladimir Ivanovich Vernadsky (1863--1945) cannot be
  called unknown to the Soviet reader.  However, he is still known among us in
  his homeland mainly as a scientist-naturalist, a historian of science, and is
  almost unknown as a thinker, a philosopher, even though his philosophical
  heritage has become a recognized phenomenon of European and world scientific
  thought long ago.
  
  The just-released book by V. I. Vernadsky from publishing house \rtitle{Наука}
  (\rtitle{Science}) \tciteo{vernadsky1988filosofskie} finally presents him
  to\footnoteTransl{omitting `и'} our reading public as a philosopher and
  thinker.  This book is, in essence, the first realization of a complete,
  unabridged publication of the essential works of the Russian thinker, above
  all the fundamental work \rtitle{Scientific Thought as a Planetary
  Phenomenon}, written in the period between the 1880s and 1940s, which has
  either completely disappeared, or has long ago become a bibliographic rarity.
\end{quotation}

During the preparation of the present edition of V.~I.\ Vernadsky's
% \tciteo: Add English translation of the title.
\tciteo{vernadsky1991thought} for publication its text was compared
with the manuscript of S.~N.\ Zhidovinov\footnoteRus{С. Н. Жидовиновым} once
again, with the help of collaborators from the Archive of the Academy of
Sciences of the USSR,\footnoteRus{Архива АН СССР} which enabled the correction
of some small inaccuracies, unnoticed in the previous editions, as well as the
restoration of the author's style, orthography, and punctuation where possible.

What does the book offered to the reader's attention present?  It is necessary
to shortly pause for a look at the development of V.~I.\ Vernadsky's ideas,
which have found their fullest reflection in this work, to answer this
question.
\medskip

It follows from letters to his wife Natalya Egorovna,\footnoteRus{Наталье
Егоровне} and to a few scientists, as well as from preserved diaries of
Vladimir Ivanovich, that his attention was attracted by the ever-increasing
technological might of mankind, which became comparable in its scale to the
most formiddable geological processes, already in his early years, i.~e.\
already at the end of the last century.  This activity irreversibly changes the
face of the whole Earth, of all of its nature in the physical-geographical and
chemical aspect.  (V.~I.\ Vernadsky didn't yet use the term `biosphere' at that
time.)

Such thoughts occurred not only to V.~I.\ Vernadsky.  He mentions his
predecessors and contemporaries in this aspect in his later works with his
characteristic courtesy.\footnoteRus{щепетильностью}  The American geologist
Charles Schuchert proposed viewing the contemporary epoch as the beginning of a
new, psychozoic age of the history of the Earth, emphasizing the significance
of the psychological activity of mankind as a geological factor with this name,
in 1933.\footcite[80]{schuchert1933geology}  Our Russian scientist A.~P.\
Pavlov, who invited V.~I.\ Vernadsky to teach mineralogy at Moscow
University in 1890, also thought that a new geological period in the Earth's
history began with the appearance of man on it, which he proposed to call
anthropogenic (from the Greek word
`anthropos'---man).\footcite{pavlov1922lednikovye}  There were also other
statements of similar character at the end of the past, and the beginning of
the present century.

However, V.~I.\ Vernadsky, not satisfied with general statements, began
dilligent labor on a quantitative estimate of the scale of human activity.
V.~I.\ Vernadsky noted the minerals and new chemical compounds formed as a
result of mankind's industrial activity, and gave the first estimates of the
total volume and mass of such `technogenic' minerals already in his
\rtitle{Mineralogy} courses,\footnoteTransl{
	See \refsmartcites{vernadsky1891mineralogy-1,
	vernadsky1891mineralogy, vernadsky1898mineralogy,
	vernadsky1899mineralogy, vernadsky1900mineralogy,
	vernadsky1906mineralogy, vernadsky1908mineralogy,
	vernadsky1910mineralogy-v1, vernadsky1910mineralogy-v2}.
} which were being republished, with additions every time, during the years of
his work at Moscow University (between 1891 and 1912).

He started publishing his \rtitle{Опыт описательной минералогии} (\rtitle{Essay
on Descriptive Mineralogy}),\footnoteTransl{
	See \refsmartcites{vernadsky1909opyt-v2, vernadsky1910opyt-v3,
	vernadsky1912opyt-v4, vernadsky1914opyt-v5}.
} subsequently encompassing all native elements, including gases, as well as
their sulphuric\footnoteRus{сернистые} and selenious\footnoteRus{селенистые}
compounds, in 1908.  In these installments, which were later collected in the
$2^\mathrm{nd}$ and $3^\mathrm{rd}$ volume of \rtitle{V.~I.\ Vernadsky's
Selected Works} (1955 and 1959),\footnoteTransl{
	\Ie\ \refsmartcites{vernadsky1955sochineniya-v2,
	vernadsky1959sochineniya-v3}.
} he includes, within the description of almost every mineral, or its groups, a
separate section titled ``Mankind's Work'',\footnoteRus{Труд человека} or
``Mankind's Activity'',\footnoteRus{Деятельность человека} in which he gives
numbers for their global extraction and refining, and communicates information
about the direct and indirect influence of human activity on the formation and
distribution of one or another mineral or chemical compound (for example,
hydrogen sulphide).

V.~I.\ Vernadsky published \rtitle{The History of Natural
Waters},\footnoteRus{\rtitle{Историю природных вод}} which he himself viewed as
the second volume of the \rtitle{History of Minerals of the Earth's
Crust},\footnoteRus{\rtitle{Истории минералов земной коры}} in two books in
1933, and 1934.\footnoteTransl{
	\Ie\ \refsmartcites{vernadsky1933istoriya-v2p1-1,
	vernadsky1934istoriya-v2p1-2}.  The third book of the series, published
	in 1936, is \refcite{vernadsky1936istoriya-v2p1-3}.
}  He dedicates quite a few pages to the conscious, and unconscious influence
of mankind on the geographical distribution, and on the composition of all
waters on the Earth in this work.  Vernadsky had concluded even then that
\begin{quote}
  the pristine rivers are quickly disappearing, or have disappeared, and have
  been replaced by a new type of formation, new waters, which had not existed
  earlier.  A transformation of the natural waters, and the simultaneous
  creation of new cultural rivers, lakes, reservoirs, coastal sea formations,
  soil solutions is going on on the vast territory of Eurasia, and in the last
  century also in America and in Australia---in the whole biosphere.
  
  This process reaches inward, changes the mode of the interstitial
  % FIXME: stratisphere?
  waters\footnoteRus{режим пластовых вод} of the biosphere and stratisphere.
  The transformation of vadose water---ground
  water\footnoteRus{верховодок---вод грунтовых} has been going on for millenia,
  then started the transformation of interstitial artesian
  waters\footnoteRus{вод пластовых напорных} by boring and ore mining.  Now its
  effect reaches more than two kilometers below the Earth's surface.
  
  The old species of surface, interstitial waters, soil waters, and
  springs\footnoteRus{старые виды поверхностных, пластовых вод, вод почв и
  источников} are disappearing and changing throughout the whole biosphere, new
  cultural waters are emerging.\footcite{vernadsky1960sochineniya-istoriya}
\end{quote}

Parallel to the study of the influence of mankind's activity on the changing of
Earth's nature, V.~I.\ Vernadsky began developing the study of the
biosphere---that envelope of the Earth, in which `living matter' is
concentrated---already in 1914--1916.  He didn't like the unnecessary coinage
of words, the creation of new terms, but had magnificent knowledge of all world
scientific literature, and employed its terminology extensively.  Such was the
case with the term `biosphere'.  It was first used by the French scientist
Jean-Baptiste Lamarck\footnoteRus{
	Жаном Батистом Ламарком, a.k.a.\ Жан Батист Пьер Антуан де Моне Ламарк
	(Jean-Baptiste Pierre Antoine de Monet, Chevalier de Lamarck).
} in a work on hydrogeology to refer to the complex of living organisms
inhabiting the globe, already in 1804.\footnoteTransl{
	From what was available on the Internet, it seems that the reference is
	to Lamarck's 1802 \btcite{lamarck1802hydrogeologie} where, in the last
	but one paragraph of the foreword, Lamarck seems to say that a good
	physics of the Earth requires studying three aspects of it, which share
	the same physical body: the atmosphere (meteorology), the Earth's crust
	(hydrogeology), and that of living bodies (biology).  The name
	`biosphere', however, does not seem to be used there.
}  The Austrian geologist Eduard Sueß,\footnoteRus{Эдуард Зюсс} and the German
scientist Johannes Walther\footnoteRus{Иоган Вальтер} used it at the end of the
$19^\mathrm{th}$ c., again with a meaning similar to Lamarck's concept.  V.~I.\ 
Vernadsky introduced a completely different, far deeper meaning to this term.
He introduced the term `living matter' for the complex of living organisms
inhabiting the Earth, but called biosphere that environment in which this
living matter is located, \ie\ the whole water envelope of the Earth, since
living organisms exist at even the greatest depths of the World Ocean, the
lower part of the atmosphere, where insects, birds, and people fly, as well as
the top part of the solid envelope of the Earth---the lithosphere, where living
bacteria can be encountered in underground waters at depths on the order of
$2\,\mathrm{km}$, and man has now penetrated to even greater depths, exceeding
$3\,\mathrm{km}$, with his shafts in the regions of gold deposits in India,
South Africa, and Brazil.  There is a `film of life',\footnoteRus{пленка жизни}
where the concentration of living matter is maximum, in the biosphere.  This is
the land surface, the soils, and the top layers of the World Ocean's waters.
The amount of living matter in the biosphere rapidly diminishes with distance
above and below it.

V.~I.\ Verndasky estimated the total amount of living matter in the
contemporary biosphere of the Earth, established the magnitude of the energy
locked up in it, carefully studied the process of absorption of solar energy
with the aid of chlorophyll in green plants on land, and algae in the World
Ocean, traced its transformation, and its influence on the generation of many
`vadose'\footnoteRus{``вадозных''} minerals, characteristic only of the
biosphere, clarified the character of solar energy's entry into the depths of
the Earth with the deposits of organic matter created by it, and analyzed all
transformations which occur in living, bioinert, and inert, as he called them,
matter of this most important envelope of the Earth for mankind.

V.~I.\ Vernadsky presented the results of his studies in numerous articles, in
the book \btcite{vernadsky1926biosfera}, which was first published in 1926, and
has been subsequently reprinted a few times, and in the fundamental work
\btcite{vernadsky1965himicheskoe}, which was first published after the author's
death, in 1965.  Many articles, brochures, and books dedicated to V.~I.\
Vernadsky's teachings about the biosphere, to its detailed presentation, to
commentary on it, and, unfortunately, only partially, to its development, have
appeared in the Soviet and in the foreign press in connection with the
increased attention to the tasks of the preservation of
nature\footnoteRus{охраны природы} during the past decade.  There is,
therefore, no need to delve into it in the foreword to the present book.  It
is, however, important to emphasize that V.~I.\ Vernadsky viewed human activity
as a process imposed on the biosphere, foreign to it by its nature from the
beginning.  We can suppose that the technogenic character of this human
activity, interfering much with the naturally occurring course of natural
phenomena,\footnoteRus{естественный ход природных процессов} contradicting
them, prompted him to have such thoughts.

We can judge of the [view of the]\footnoteTransl{
	Interpolated to express the implied meaning. [---Pav]
} `imposed',\footnoteRus{``наложенном''} foreign character of mankind's
industrial activity from numerous statements of V.~I.\ Vernadsky even in his
works from the beginning of the thirties.  So, he wrote in the mentioned
\rtitle{History of Natural Waters} about technogenic solid minerals, and
waters: ``These new chemical compounds---`artificial', \ie\ created with the
participation of the will and the consciousness of man, can, for now, be put
aside in the study of the history of natural
bodies''.\footcite[87]{vernadsky1960sochineniya-istoriya}\fncomma
\footnoteTransl{
	Here's what \refcite[§133]{vernadsky1933istoriya-v2p1-1} actually says:
	
	\smallskip
	\begingroup \leftskip 3em\rightskip 1em
	  \textbf{§133. }We are now living at only the very beginning of the
	  Psychozoic Age.  It is impossible to encompass its results
	  completely.
	  
	  We are still in a transitional period.  However, we cannot leave
	  without attention an on-rushing force changing the history and
	  composition of natural waters.
	  
	  Problems of the quantification---in all varieties---of the change of
	  natural waters by human culture have come to the fore, it being
	  necessary to reconstruct the character of those waters, which existed
	  a hundred thousand, and more, years ago, as well as in the previous
	  geological periods.  These problems have hardly been touched upon,
	  but they can be encompassed by scientific thought, and it is
	  necessary to strive toward their resolution, and have them in mind in
	  studying the history of natural waters.
	  
	  We find ourselves in the same condition, which we run into in other
	  branches of mineralogy,---with the emergence of new natural
	  compounds created by culture, changing the history of natural bodies
	  of the same, or of similar composition.  I already regarded that in
	  the history of metals (I, §267 etc.); and the same is reflected in
	  the history of natural gases ($CO_2, SO_3, H_2S$ and so forth).
	  These new types of chemical compounds---`artificial', \ie\ created
	  with the participation of the will and consciousness of man, can, for
	  now, be left aside in studying the history of natural bodies.
	  
	  But this is, obviously, a temporary solution to the problem.  We must
	  never fail to take into account the products of human work in the
	  history of numerous minerals, for example, carbon dioxide.  We must
	  neither fail to take them into account in the history of natural
	  waters.  However, on the other hand, it is impossible to include it
	  completely in our present considerations.  A host of waters connected
	  with engineering are constantly and quickly changing,---are
	  temporary, transitional phenomena.  Many of the new waters are
	  negligible in mass---are rare, quickly disappearing `minerals'.
	  
	  I will consider these new waters---a creation of cultural life---only
	  in so far as this is necessary for understanding the essential main
	  features of the history of natural waters.  This, however, is, of
	  course, only a temporary solution of the question---this is the
	  intrinsic\footnoteRus{исконный} path of the naturalist, seeking the
	  important, but not a logical sequence, in the complex phenomenon,
	  real at a given moment.
	  \par
	\endgroup
	[---Pav]
}

However, V.~I.\ Vernadsky arrived at the unavoidable conclusion about the
evolution of the Earth's biosphere, about the quantitative and the qualitative
change of its main component part---living matter, about the stages of the
biosphere's evolution in the last decade of his life.  Such a course of
thoughts brought him to the conclusion that the emergence of man, and the
impact of his activity on the surrounding natural environment is not an
accident, is not an `imposed' process on the natural course of events, but is,
rather, a definite, lawful stage of the evolution of the biosphere.  This stage
has to lead to the condition that the Earth's biosphere must transition into a
new state, which he proposed to call `noosphere' (from the Greek word
`noos'\footnoteRus{``ноос''}---mind), under the influence of scientific
thought, and the collective labor of unified mankind, directed toward the
satisfaction of all of its material and spiritual necessities.  V.~I.\
Vernadsky did not invent this term, nor the term `biosphere' himself.  He
lectured on biogeochemistry and the development of the biosphere at Collège de
France from 1922 to 1926 during his long foreign assignment, and the French
mathematician Édouard Le Roy, student of these lectures, published an article
about them in 1927, in which he employed, for the first time, the term
`noosphere', used by other French scientists and by V.~I.\ Vernadsky further
on.

The work \rtitle{Scientific Thought as a Planetary Phenomenon}, judging from
the diaries of V.~I.\ Vernadsky and from his letters, was written mainly in
1937--1938, \ie\ during the most tragic years of our history.  V.~I.\ Vernadsky
was far from indifferent to the events of those days.  His friends and students
were repressed.  Trying to prove their innocence, and the erroneousness of
their arrests, he wrote letters to J.~V.\ Stalin, N.~I.\ Ezhov, and
L.~P.\ Beria.  His diaries from these years are filled with heavy words.  But
the book, written by him for the future generations, was permeated by optimism,
faith in the triumph of human reason.

It is hard to completely characterize the content of the book.  It is
significantly broader than the book's title, though the idea of the global
significance of scientific thought permeates it from beginning to end, and
connects all of its parts.  Essentially, this book is an introduction to the
teaching of the noosphere.  Many places in it are dedicated to the analysis of
the conception of this term.  Along with that, the role of mankind in the
development of the biosphere is painted with the broad strokes of a great
painter, the concept of living matter and its state of organization, of the
evolution of the biosphere and the inevitability of its gradual transformation
into the noosphere, of the conditions neccessary for such a transition, of the
basic stages of development of human culture and its further destiny, of
biogeochemistry as a main scientific area of the study of the biosphere, of the
fundamental differences between the living and the inert matter of this evelope
of the Earth is given.

The work \rtitle{Scientific Thought as a Planetary Phenomenon} occupies a
special place among the works of V.~I.\ Vernadsky.  It is distinguished by an
unusual breadth of the range of questions considered in it, and by the specific
character of the main problems examined.  The breadth of the views of their
author about things, and the significance of the scale on which he poses
questions have always been inherent to V.~I.\ Vernadsky's works.  However,
these qualities of the scientist have been brought to a most prominent and
powerful expression in the work being published.  Nature, human society,
scientific thought are examined in their indissoluble unity, and the reality
surrounding us is painted in a truly universal\footnoteRus{вселенской, \ie,
also, `ecumenical' (see \autoref{ch:3}).} vastness.

\rtitle{Scientific Thought as a Planetary Phenomenon}---this is an apex of
V.~I.\ Vernadsky's work, a grandiose, in its intention, summary of his
meditations on the destiny of scientific knowledge, on the relationship between
science and philosophy, on the future of mankind.  It can be characterized as
an impressive, though unfinished, synthesis of the ideas being developed by the
scientist in the last period of his life.

Deep thoughts about the evolution of mankind on geological and socio-historical
scales of time are contained in the book.  It must be admitted that this is the
first attempt in world literature to generalize the evolution of our planet as
a single cosmic, geological, biogenic, and anthropogenic process.  The leading
transformative role of science and the socially organized labor of mankind in
the present and future of the planet is revealed in the work.  Scientific
thought, science is viewed and analyzed as the most important force of
transformation and evolution of the planet.

We must not fail to note that the book offered to the reader's attention has
also a deeply philosophical content.  V.\ I.\ Vernadsky was not simply
intrested in philosophy, but studied the works of philosophers of various
schools and currents thoroughly since his teenage years.  He considered the
collection and generalization of scientific facts as inseparable from the
philosophical understanding of the reached scientific conclusions, which is
especially distincly evident from his diaries and his correspondence.

Already in 1902, beginning his work on the history of the development of human
culture, he wrote to his wife Natalya Egorovna: ``I view the significance of
philosophy in the development of human knowledge entirely differently from the
majority of naturalists, and ascribe an enormous, fruitful signicance to it.
It seems to me that these are two sides of the same process---completely
unavoidable and inseparable sides.  They are separated only in our minds.  Were
one of them to die away, the living growth of the other would cease\dots{}
Philosophy always concludes \emph{germs, }sometimes even anticipates whole
areas of the future development of science, and, only thanks to the
simultaneous work of the human mind in this area as well, correct criticism of
the unavoidably over-simplified notions of science is produced.  Such
significance of philosophy, as the roots and vital atmosphere of scientific
endeavor, can be precisely and clearly traced in the history of the development
of scientific thought.''\footcite{vernadsky1988trudy-p21}

V.~I.~Vernadsky stayed true to the principle presented in this letter his
whole life.  Statements with similar meaning can be found in numerous other
letters and works of his, especially in the many publications on the history of
scientific knowledge.  All of them are permeated by philosophical
conceptualizations of the presented material.

However, we, it seems, unexpectedly encounter different statements of
V.~I.~Vernadsky's, which separate philosophy from scientific knowledge, and
even mention it alongside religion, in works, letters, and diaries from the
30s.  In order to understand this, it is necessary to take into account that in
the given case we are talking about the dominant in those years philosophy of
vulgar dialectical materialism, ordering not only the representatives of the
social sciences, but also natural scientists what conclusions and inferences to
make, in order for them to completely correspond to the philosophical ``laws''.
V.~I.~Vernadsky could not accept such a philosophy, for which he was
criticized by A.\ M.\ Deborin, who accused him of
idealism.\footnote{
	\cite{deborin1932problema}
}
V.~I.~Vernadsky responded to this criticism with great dignity, even though it
painfully wounded his pride.\footnote{
	\cite{vernadsky1933povodu}
}
He always thought that an unbiased collection of as many facts about the topic
of investigation as possible, their subsequent objective generalization, and,
only afterward, a philosophical understanding must be the basis for every
investigation.  By the way, V.~I.~Vernadsky regarded Karl Marx as a scientist
with great respect precisely because a great amount of thoroughly and
conscientiously collected material lay at the foundation of the
\btcite{marx2010capital-v1}\nocite{marx2007capital-v2,marx2010capital-v3}.

The development of V.\ I.\ Vernadsky's philosophical views is reviewed more in
depth in the article ``From the Editorial Board'' in the mentioned book
\tciteo{vernadsky1988filosofskie}.  Wide-ranging commentaries on the work
\rtitle{Scientific Thought as a Planetary Phenomenon} were published in that
book, as well as in the form of appendices, articles by B.~M.~Kedrov,
I.~V.~Kuznetsov, S.~R.~Mikulinskiy, and A.~L.~Yanshin written at various times,
in which the questions of V.~I.~Vernadsky's worldview and his teaching about
the gradual transition of the biosphere into the noosphere are reviewed from
different points of view.

We have significantly reduced the commentary in the present edition, which is
indented for the widest possible circle of readers, transferring the necessary
part of it to footnotes.  The majority of editorial comments have been updated.

V.~I.\ Vernadsky's articles \btcite{vernadsky1902nauchnom} and ''A Few Words
about the Noosphere'', as well as fragments (six sections) from the manuscript
\rtitle{Scientific Thought as a Planetary Phenomenon}, which, as was proper,
were not included in the text by the author himself.  The reader may be
interested in the content of these fragments, and obtaining the the small-run
journal \refparencite{voprosyIstorii1988n1} is fairly difficult.  Therefore, we
decided to reproduce them in the present edition, after comparison with the
author's original, and the correction of errata and modifications in the
journal's version.

The article \btcite{vernadsky1902nauchnom} was first published in the journal
\ctciteo{vernadsky1902nauchnom} in 1902, and was re-published a few more times
subsequently in various collections with little changes during Vernadsky's
life.\footnote{
	We are re-publishing it according to the edition
	\fullcite{vernadsky1988nauchnom}.
}

This is V.~I.~Vernadsky's first philosophical work, important for the
formulation of those views which permeate his further work.  He announces the
existence of a reality, independent of our consciousness, the conception of
which is the scientific world-view, changing according to the discovery of new
facts, new natural phenomena.  It is interesting that when P.~I.~Novgorodtsev,
having become acquainted with the manuscript of this article by Vernadsky,
proposed to publish it in the collection \btcite{novgorodtsev1902problemy}
Vladimir Vernadsky refused, stating that he was not an idealist, but a
realist.\footnote{
	\fullcite{mochalov1982vladimir}
}

The second article, ``Some Words about the Noosphere,'' can be viewed as a
direct continuation and development of the views presented in the work
\rtitle{Scientific Thought as a Planetary Phenomenon}.  This article was
published in 1944.\footnote{
	\fullcite{vernadsky1944neskolko}
}
The conditions, ensuring the transition of the biosphere into the noosphere,
are clearly formulated in it, and the article concludes with the author's
expressed firm confidence in the victory over fascism, since ``the ideals of
our democracy are in unison with the tempestuous geological process, with
nature's laws, correspond to the noosphere.''

This last of V.~I.~Vernadsky's publications in his lifetime (the article was
already written in the heat of the war in 1943) is infused with optimism.  The
genius scientist was convinced that human reason will triumph, and not only
that fascism will be defeated, but also that everything which still interferes
with the transformation of the biosphere into the noosphere will be eliminated.

Now his forecasts have started to come true.

\section*{Remarks}

In Vernadsky's manuscript \btcite{vernadsky1991thought}, written in 1938 and
kept in archive of the Academy of Sciences of the USSR, after section 150,
whose text was redacted from the 1977 edition of the book
\btcite{vernadsky1977razmyshlenia}, and restored in full in the book
\btcite{vernadsky1988filosofskie}, published in 1988, six more sections follow,
about which I.~I.~Mochalov and K.~P.~Florenskiy wrote as follows in their
commentaries to the 1977 edition:

. . .
