\ChapterByLine{Preface and remarks by A.~L.\ Yanshin~\dots}{%
A.~L.\ Yanshin\\
{\rm\normalsize Chairman of the Committe of the Academy of Sciences of the USSR
for the Exploitation of the Scientific Heritage of Academician V. I.
Vernadsky}\footnotemarkTransl
\\[1ex]
F.~T.\ Yanshina\\
{\rm\normalsize Director-founder of the museum home of Academician V.~I.\ 
Vernadsky}\\[4ex]
\begin{center}
	\textit{The electronic version of the preface and the remarks was
	prepared from the edition in the book
	\fullcite{vernadsky1991thought}.}
\end{center}
}%
\footnotetextTransl{\foreignlanguage{russian}{Комиссия по разработке научного
наследия академика В.~И.\ Вернадского}}%


\section*{Preface}

The name of Vladimir Ivanovich Vernadsky has become widely known in our
country.  There is nobody with even the slightest degree of education, who
hasn't read, if not Vernadsky's works, then, at least, numerous newspaper and
magazine articles about him and his work.

There is a Vernadsky Avenue in Moscow.  One of the largest institutes at the
Academy of Sciences of the USSR, the Institute of Geochemistry and Analytical
Chemistry,\footnoteRus{Институт геохимии и аналитической химии} bears his name.
There is a Committee for the Exploitation of the Scientific Heritage of
Academic V.~I.\ Vernadsky, which publishes its own circular, at the Presidium
of the Academy of Sciences of the USSR.  Branches of that Committee work in
Leningrad and in Kiev.  There have been grants under Vernadsky's name
established at Moscow, Leningrad, Kiev, and Simferopol University.  National
scientific centers for the study of the work of this prominent thinker and for
its application to the solution of contemporary problems exist in Odessa,
Rostov-na-Don, Erevan, Simferopol, Ivanov, and in other cities in the USSR, and
abroad---in Prague, Oldenburg and Berlin.\footnote{
	Also named after V. I.  Vernadsky are: the National Geological
	Museum,\footnoteRus{Государственный геологический музей} the National
	Public University of Biospheric Studies,\footnoteRus{Всесоюзный
	народный университет биосферных знаний} the Central Scientific Library
	of the AS UkrSSR,\footnoteRus{Центральная научная библиотека АН УССР}
	the Student Sociological Center ``Noosphere'',\footnoteRus{Студенческий
	социологический центр ``Ноосфера''} the peak in the basin of
	Podkamennaya Tunguska River, the crater on the dark side of the Moon,
	the peninsula in Eastern Antarctica near the Sea of
	Astronauts,\footnoteRus{Море Космонавтов} the forest on the island of
	Paramushir (Kuril Island), the subglacial forests in Eastern
	Antarctica, the underwater volcano in the Atlantic Ocean, the mine in
	the region of Lake Baikal, the mineral Vernadit,\footnoteRus{вернадит,
	$Mn^{4+}, Fe^{3+}, Ca, NaS(O,OH)_{2n}\cdot H_2O$} the diatomaceous
	algae, research vessel ``Academician Vernadsky'' of AS UkrSSR, the
	steamboat ``Geologist Vernadsy''\footnoteRus{Геолог Вернадский} of the
	Kama River Shipping company,\footnoteRus{Камское речное пароходство}
	the Vernadsky village near Simferopol, the Vernadsky railway station on
	the Kazan line, the subway stop ``Vernadsky Avenue'' in Moscow, the
	Biosphere Museum at the Leningrad branch of the Institute of the
	History of Natural Science and Technology of the AS USSR.  A V.~I.\ 
	Vernadsky monument has been erected in Kiev, a memorial plate is in
	place on the old building of Moscow State University M.~V.\ 
	Lomonosov,\footnoteRus{МГУ им.\ М.~В.\ Ломоносова} on Vernadsky Avenue
	in Moscow, on the building of Leningrad State
	University,\footnoteRus{Ленинградского государственного университета}
	as well as on the building of the Kiev State University T. G.
	Shevchenko.\footnoteRus{Киевского государственного университета им. Т.
	Г.  Шевченко}  Bonuses V.~I.\ Vernadsky are awarded for exceptional
	scientific work in the areas of mineralogy, geochemistry and
	astrochemistry by the Academy of Sciences of the USSR and by the
	Academy of Sciences of the UkrSSR.  A golden medal named after him has
	been established by the Academy of Sciences of the USSR.}

V.~I.\ Vernadsky's 125th birthday was celebrated in March 1988 in our country,
as well as abroad (in Prague and in Berlin).

The celebrations srpead very widely.  An exhibition dedicated to his work was
opened on January 15, 1988 at the Exhibition of the Achievements of the
National Economy.\footnoteRus{ВДНХ, from выставка достижений народного
хозяйства}  Scientific symposia on different directions of V.~I.\ Vernadsky's
research took place successively in Leningrad, Kiev, and Moscow with the
participation of foreign scientists from March 3 to 11.  A commemorative
conference took place in Balshoy Theatre\footnoteRus{Большой театр} in Moscow
on his birthaday, March 12, with the participtation of public organizations.
Separate conferences and scientific sessions took place during the same days in
Ivanov, Odessa, Simferopol, Rostov-na-Don, Yerevan, Baku,
Almaty,\footnoteRus{Алма-Ате} Novosibirsk, Irkutsk, and in many other
scientific centers of the nation.  The proposal to create an International Fund
V.~I.\ Vernadsky\footnoteRus{Международного фонда В.~И.\ Вернадского} for
subsidizing the translation of his works in other languages, finding materials
about him in foreign archives, and the invitation of scientists from foreign
nations to the USSR for reports and lectures on the contemporary development of
scientific problems noted by V.~I.\ Vernadsky was accepted.\footnote{See the
information at the end of the book.}  Articles about him, and his multifaceted
scientific work have appeared in almost all Soviet and international newspapers
and magazines.

Publishing house \rtitle{Nauka}\footnoteRus{Наука} released 4 volumes of works
by V.~I.\ Vernadsky, as well as his \tciteo{vernadsky1988pisma}, including the
book \tciteo{vernadsky1988filosofskie}, in which the work \rtitle{Scientific
Thought as a Planetary Phenomenon} was republished as a first part, now
published with reconstructions of all those passages abridged for its first
edition in 1977 according to the archived original manuscript, before the very
anniversary in February 1988.  The book was released in a 20,000 run.  The whole
run was bought out during the very first days after its appearance in the
bookstores' windows.  A barrage of letters requesting the release of an
additional run of \tciteo{vernadsky1988filosofskie}, or at least of its first
part, were received at the Scientific-publishing council of the Academy of
Sciences of the USSR.\footnoteRus{Научно-издательский совет АН СССР}

The appearance of \tciteo{vernadsky1988filosofskie} in 1988 found a broad
positive response from the press.  For example, the article
\tciteo{lukyanov1988neizvestny} was published in the journal
\ctciteo{lukyanov1988neizvestny} from September 29, 1988, in which the author
F.~Lukyanov\footnoteRus{Ф. Лукьянов} wrote:
\begin{quotation}
  The name of academician Vladimir Ivanovich Vernadsky (1863--1945) cannot be
  called unknown to the soviet reader.  However, he is still known among us in
  his homeland mainly as scientist-naturalist, historian of science, and is
  almost unknown as a thinker, philosopher, even though his philosophical
  heritage has become a recognized phenomenon of European and world scientific
  thought long ago.
  
  The just-released book by V. I. Vernadsky from publishing house \rtitle{Наука}
  (\rtitle{Science}) \tciteo{vernadsky1988filosofskie} finally presents him
  to\footnoteTransl{omitting `и'} our reading public as a philosohper and
  thinker.  This book is, in essence, the first realization of a complete,
  unabridged publication of the essential works of the Russian thinker, above
  all the fundamental work \rtitle{Scientific Thought as a Planetary
  Phenomenon}, created in the period between the 1880s and 1940s, and either
  completely disappeared, or long ago become bibliographic rarity.
\end{quotation}

During the preparation of the present edition of V.~I.\ Vernadsky's
\tciteo{vernadsky1991thought} for publication its text was compared
with the manuscript of S.~N.\ Zhidovinov\footnoteRus{С. Н. Жидовиновым} once
again, with the help of collaborators from the Archive of the Academy of
Sciences of the USSR,\footnoteRus{Архива АН СССР} which enabled the correction
of some small inaccuracies, unnoticed in the previous editions, as well as the
restoration of the author's style, orthography, and punctuation where possible.

What does the book offered to the reader's attention present?  It is necessary
to shortly pause for a look at the development of V.~I.\ Vernadsky's ideas,
which have found their fullest reflection in this work, to answer this
question.
\medskip

It follows from letters to his wife Natalya Egorovna\footnoteRus{Наталье
Егоровне}, and to a few scientists, as well as from preserved diaries of
Vladimir Ivanovich, that his attention was attracted by the ever-increasing
technological might of mankind, which became comparable in its scale to the
most formiddable geological processes, already in his early years, i.~e.\
already at the end of the last century.  This activity irreversibly changes the
face of the whole Earth, of all of its nature in the physico-geographical and
chemical aspect.  (V.~I.\ Vernadsky didn't yet use the term `biosphere' at that
time.)

Such thougts occurred not only to V.~I.\ Vernadsky, and he mentions his
predicessors and contemporaries in this aspect in his later works with his
characteristic courtesy.\footnoteRus{щепетильностью}  The American geologist
Charles Schuchert proposed viewing the contemporary epoch as the beginning of a
new, psychozoic age of the history of the Earth, emphasizing the significance
of the psychological activity of mankind as a geological factor with this name,
in 1933.\footcite[80]{schuchert1933geology}  Our Russian scientist A.~P.\
Pavlov, who invited V.~I.\ Vernadsky to teach mineralogy at Moscow
University in 1890, also thought that a new geological period in the Earth's
history began with the appearance of man on it, which he proposed to call
anthropogenic (from the Greek word
`anthropos'---man).\footcite{pavlov1922lednikovye}  There were also other
statements of similar character at the end of the past, and the beginning of
the present century.

. . .
