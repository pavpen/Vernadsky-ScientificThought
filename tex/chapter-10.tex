\Chapter{%
The biological sciences must come, together with the physical and the chemical
sciences, among the sciences encompassing the noosphere.}

151. But the contemporary state of biology and its excursions into philosophy
are also detrimental to philosophy.

The expectant attitude of the naturalist for the confirmation of philosophy
creates among philosophers the impression that precisely the
scientists\footnoteTransl{probably a typo, and should be: ``exact scientists''
[---Pav]}, proceeding from their data, accept the basic tenets of the
philosophical current of materialism about the lack of fundamental difference
between living and inert.  Vitalistic notions have remained so far in the past
in the general course of biological thought that their real significance hardly
influences large-scale work.  The overbearing majority of naturalists are far
from them.

The philosophers-naturalists, whose significance in contemporary philosophical
thought, in its global scope, is minute, receive [from the exact scientists]
what seems like firm ground, and calm their doubts.  This impacts their
creative work, which slowly dies down, and degenerates into dry, formal
scholasticism, or into verbal talmudism, especially in such cases as our
country, where dialectical materialism is the state philosophy, and is
favoured by the mighty support of government power, and by intellectual and
practical impossibility of its free criticism and of the free development of
any other philosophical views.

However, official dialectical materialism itself, being one of the many forms
of this current of philosophical thought, does not possess such freedom,
either.  And has been, meanwhile, never systematically philosophically worked
out to the end, remaining full of unclarity and unthoughtfulness.  Its official
exposition has changed more than once during the past twenty years, previous
ones were declared heretical, and new ones were created.  Our philosophers of
strict discipline, in which they work, have been obliged to obey without
objection, under the threat of persecution and material hardship, these new
ones, and to publicly repudiate their previous teachings, admitting their
mistakes.  It is easy to imagine what result follows, and how fruitfully can
one work intellectually in such a severe real environment.  As a result, a
condition very reminiscent of the condition of the orthodox church under
despotism has arisen, with the gradual downfall of lively work, work in this
area of philosophy, the exit into safe areas of knowledge, the publication of
classics, forebears; a new degeneration of thought has arisen.


152. It seems to me that for these 20 years, except the republication of old
works, which were released in the pre-revolutionary period, not a single
independent, purely philosophical work has been published, and there are not
even histories, based on primary sources, of the creation of dialectical
materialism itself.\footnoteEd{This part of the phrase is crossed out by the
author in the manuscript.}  Such decline of philosophical thought in the area
of dialectical materialism in our country, and the seemingly extensive
possibilities of its manifestation, are a consequence of the adopted
understanding of the goals of philosophy, and of the decrease of deep
philosophical work, thanks to the belief among our philosophers that a
philosophical truth, which cannot be changed and subjected to doubt any
further, has been reached.

Such an idea is, essentially, foreign to both K. Marx and F. Engels, not to
mention Feuerbach.

It was developed on Russian soil in the middle of emigration, and grew into a
state ideological influence completely unconsciously, its consequences being
unexpected for many very prominent freely thinking communists, as well.

The fight of the intellectual circles turned, in the end, imperceptibly and
unsuspectedly, into a state philosophy of the winning interpretation of
dialectical materialism.

Thanks to the strengthening of one definite current, this has been manifested
more and more clearly during the past 10 years.

As a result, we see, or we have, instead, a mass of literature of a transient
character, rooting out conscious or unconscious errors and heresies, deviations
from the officially accepted state philosophy.  On top of that, the state
philosophy itself has changed in very important nuances in the accepted
interpretation of dialectical materialism.  Such a sad state of work in our
country in the area of dialectical materialism at the presence of huge material
resources, which had never existed for any other philosophy (except for
theological ones---Catholic and Muslim philosophies in the Middle Ages), would
unavoidably come in another way, as well, thanks to many peculiarities in the
structure of state philosophy in our country.  On the one hand, thanks to the
emigration of intellectual circles, whose significance was already indicated;
and, on the other, thanks to the complexity, independent of life in our
country, of the environment, in which dialectical materialism was being
created.


153. Dialectical materialism, in the form in which it is actually manifested in
the history of thought, was never presented coherently by its authors---Marx,
Engels, and Ulyanov-Lenin.  These were prominent thinkers, and no less
prominent political activists.  Characteristic of them are a large breadth of
scientific knowledge and scientific interests, unusual for political activists.
They stood at the level of their time, but at the same time were volitional
personalities, organizers of the popular masses.  They were actively opposed
to, and regarded strongly negatively religious searches, judging them,
historically, as a force hostile, in the end, to the interests of the popular
masses and to the freedom of scientific work.  However, they, at the same time,
attributed great significance to philosophical thought, whose primacy over
scientific thought did not raise any doubt to them.

Their philosophical ideology was most closely related to their political
activity, and left an imprint on their scientific searches and understanding.
They were primarily philosophers, spokesmen for aspirations, and
organizers of the actions of the popular masses, whose social well-being---on a
real planetary basis---was the goal and meaning of their lives.  We see, by the
example of these people, a real, great impact of the personality not only on
the course human history, but, through it, on the noosphere, as well.

Part of the polemical works which their authors---Marx, Engels, Lenin,
Stalin---never intended for such a task were laid in the foundation of the
Soviet state philosophy; their statements on practical and political questions
of life, in which philosophy sometimes occupied a secondary place.  Such were,
secondly, draft notebooks, extracted from the manuscripts remaining after their
deaths, often reports and overview summaries related to the reading of
philosophers, which were never historically, scientifically, critically
published.  They were published by the scientific apparatus and with the
obeisance of believing students, and, as always under such circumstances, are
full of contradictions, and, in some cases, such as the Engels's
\rtitle{Dialectics of Nature}, the authorship of all of Engels's statements
cannot be considered proven.  A few works of Marx, and, partly, Engels, have a
different character, but they are completely insufficient for the firm
establishment of a new philosophy.  Marx' and Engels' life work was in another
domain.  Marx was a prominent scientist, who in the \rtitle{Kapital} reached
his conclusions by an exact scientific pathway, but presented them in the
language of Hegelian philosophy, independently reworked by him and Engels,
which already during their lifetimes did not (in general) correspond to current
scientific methodology and scientific searches.  The prominent mind could
permit itself such a peculiar form of presentation.

Already during Marx's lifetime---at the publication of the last volumes of his
\rtitle{Das Kapital}---such a presentation was an obvious anachronism, and it
is an even greater one in our time.  In essence, of course, what is important
is not the form of presentation of the scientific work, but rather the actual
methodology, by which what is presented has been reached.  The form of Marx's
presentation misleads the reader into thinking that what is presented was
reached by a philosophical pathway.  It is, in reality, only presented that
way, but was, in fact, reached by the exact scientific method of the historian
and economist-thinker, who Marx was in his scientific work.

It turned into a complete anachronism, since it was transferred from the area
of political economy and history into the area of natural and exact sciences.
This transfer, which can be observed in the works of both Marx and Engels,
acquired an extremely peculiar character with their epigons, having become the
state philosophy of a large and strong nation, most closely related to the
International.

Thirdly, the situation was worsened by the fact that the authors of these
philosophical searches were people, either actually exercising dictatorial
power in an unprecedented depth and degree, and considering the philosophical
ideology of dialectical materialism as the basis of their political and
practical activity, or people, such as Marx and Engels, who are not subject to
free criticism in our country for the same reason.  Their conclusions are, in
fact, accepted as impeccable dogma, defended by the full mechanism of
government power.

The stagnation of philosophical thought here, and its transformation into
fruitless scholasticism and talmudism, opulently blooming against that
background, is a direct consequence of this state of affairs.

This, in essence, great historical phenomenon was prepared in our country by
deeply-rooted submissiveness---unchanged during all the transformations of the
form of government---to the state religion.  The official Orthodoxy in the
Tsardom of Russia, as well as in the Russian Empire, prepared the ground for
the official philosophy, which replaced it, and which has acquired the clear
form of official religion with all of the consequences from that.

154. This, however, is, historically and in essence, only the everyday side of
the matter.  The ideology and its associated belief at its foundation are far
more important.

Dialectical materialism, in sharp contrast to contemporary forms of philosophy,
is extremely distant from philosophical scepticism.  It is convinced that a
universal method rules---an infallible criterion of philosophical and
scientific truth.  This is the effect of the temperament of its founders Marx
and Engels, who succeeded, thanks to their joining the still alive at that time
Hegelian philosophy, to impart to their scientific achievements the vibrantly
active form of faith, and not only of a philosophical doctrine---to create a
political force, able to move the masses and vividly manifest itself in the
\rtitle{Communist Manifesto} of '48---in a brilliant and profound work,
reflecting the age of the middle of the last century, when the primacy of
philosophy over science dominated ideologically Euro-American civilization.

In contrast to other forms of materialism, with which it is in fundamental
disagreement, dialectical materialism is closely related in its genesis and in
the basis of its formulations with idealism in its Hegelian form.

It is far from clear, whether it is possible to regard it as free from the
influence of such history, and to attribute it completely to the philosophical
current of materialism.

As far as I know, this question is historiographically unresolved, and in the
manifestation which materialism has in our country, its idealistic basis is
strongly emphasized, whereas its materialistic one is an outer appearance.

But this is a debatable area, far from my interests, and from my knowledge, and
I would not concern myself with it, if the sharp distinction between the
philosophical current of materialism and dialectical materialism did not become
completely clear in our country in the aspect which most concerns the
naturalist and seriously affects scientific work in our country.

Materialistic philosophy was evidently distinct---and that is where its force
lied---from the other philosophical currents of modern times, in the fact that
it did not conflict with science, was completely based on its achievements, as
far as possible.  It generalized and developed them.  In essence, it continued
that great movement, which developed in the 17th--18th centuries on the basis
of the new science, the new philosophy, and the new ways of everyday life and
technologies, which were created at that time.

Materialism, in essence, was striving to become a scientific philosophy, or a
philosophy of science.  It did not succeed in practice, since in its logical
conclusions, being part of the philosophy of the Enlightenment from the end of
the 18th century, when it clearly occupied a place on the historical stage for
the first time, it quickly fell behind the science of the times.

But in the aspect concerned in this book, what is important is not the success,
or failure of materialism in its historical manifestation during the age of its
flourishing at the end of the 18th century, and in the 1860s, but the
foundation of its ideology, which always recognized the primacy of science
above philosophy.  It considered everything proven by science as obligatory for
itself.

The dialectical materialism, created by Marx and Engels, did not accept that,
and, in that, sharply distinguished itself from all forms of philosophical
materialism, and, from that standpoint, did not differ at all from idealistic
Hegelianism.

For that very reason, it is also clearly distinct from philosophical
scepticism, which accepts the realistic worldview, as it is manifested
scientifically, as the only possibility, and does not recognize, in comparison,
either religious, or philosophical views on an equal basis.  Philosophical
scepticism, in contrast to philosophical materialism, does not recognize the
scientific view of reality as its complete view, taking into account the
increase of scientific knowledge, and the imperfections of human reason.  But
for it the scientific achievements at a given historical moment, and at a given
form of the human brain have the character of the most precise achievement of
reality.  Dialectical materialism does not proceed from scientific data, is not
limited to their boundaries, is not based on them, but is striving to change
and develop them, adapting them to its views, which have as a basis the laws of
Hegelian dialectics.  It seems to me that this dialectics is so closely related
to the whole philosophy of Hegel that through it foreign, from the standpoint
of materialism, formulations enter into the spiritual environment of
materialism---mystical, distorting to it, such as, for example, the
manifestation of dialectics in nature, or in the present case, speaking
scientifically, in the biosphere.

The introduction of the dialectics of nature in the philosophical purview of
our country, in its official philosophy, during our time of great increase, and
significance of science---is a remarkable historical phenomenon.

This has been the form of the post-mortem influence of the works of Marx and
Engels, based on faith---officially---but not expressed philosophically, or
scientifically, etc.\ [by them].


155. Effectiveness, i.e. the equal significance of methodological thought and
the instructions of the philosophers-dialecticians for current scientific work,
is strongly underscored in our philosophical literature, and is introduced into
science through the agency of government power.

The philosophers-dialecticians are convinced that they can aid current
scientific work with their dialectical method.

They believe in its significance for science, but the manifestation of that
belief in reality contradicts it.

It appears to me that this is a misunderstanding.  No philosophy has played, or
plays, such a role in the history of thought.  No philosopher can instruct the
scientist in the pathway to take in the methodology of scientific work,
especially in our times.  The philosopher is not capable of precisely
encompassing the complex problems, whose solutions stand today before the
naturalist in one's current work.  The methods of scientific work in the area
of experimental sciences and descriptive natural sciences, and the methods of
philosophical work, even in the area of dialectical thought, are expressly
different.  It seems to me, the two lie in different domains of thought, as far
as we are dealing with concrete natural phenomena, i.e. with empirically
established facts, and empirical generalizations built upon scientific facts.
It seems to me that the issue here is so clear that no argument is necessary.
Our philosophers-dialecticians must not interfere with this area of scientific
knowledge for their own benefit.  Here, also, their attempt is doomed to
failure from early on.  Here they are fighting with science on its native
terrain.

Science lived through a similar interference of religious thought and religious
constructs, erroneous at their roots, during the age of the Renaissance, during
the 17th--19th centuries.  Though the fight here is not yet over, hardly
anybody would deny that victory has remained on the side of science, that the
majority of religious constructs of that type remained in the past, or are
being reconstructed in their essence, reinterpreted, are shifting from the area
of reality into that of personal belief and interpretation.  The historical
experience was not taken into account by the official philosophers of our
country, and they, in their squareness and insufficient scientific literacy,
entered into a sharp conflict with scientific thought and work, which are
correctly placed ideologically high in our country---on an equal level with
dialectical materialism---at the foundation of our system of government.

The weakness of placing ``dialectical materialism'' at such a height,
unavoidably impacts its real power in nation building, does not correspond to
reality, and unavoidably proves to be transient.

Conflicts with the actual needs of life are beginning, which must unavoidably
have those consequences, which came into being \dots\ supreme
\dots\footnoteEd{Illegible in the manuscript.} in the old Christian nations.


156. I have collided with this kind of circumstances in my scientific work many
times, and have even mentioned the struggle of my predecessors in scientific
knowledge from the past century in public statements.

In 1934 little-educated philosophers, heading the planning of scientific work
of the former Geological Committee\footnoteRus{Геологический комитет},
erroneously attempted to prove, by means of dialectical materialism, that the
determination of geological age by means of radioactivity is based on erroneous
theses---dialectically unproven.  They thought that the facts and empirical
generalizations that radiologists relied upon were dialectically impossible.
They were joined by a few geologists, occupying themselves with philosophy, and
heading the scientific leadership of the Committee.  They held up my work by
one-two years, because the Radium Institute\footnoteRus{Радиевый институт},
which I headed, was completely unable to get in touch with the work of the
Committee geologists, and to put the investigations on a solid basis.  In the
end, after an uncareful statement at the public session of the Committee by the
Vice Scientific Director\footnoteRus{заместителя директора по научной части}
professor M.  M.  Tetyaev\footnoteRus{М. М. Тетяев}, a prominent geologist,
publicly indicating the incompatibility between dialectical materialism and the
conclusions of radiologists, it was possible to achieve a, now public,
discussion on this subject.  It was possible to do so, because the whole
radiological work of the Committee was under attack by his statement.  I was
able to intervene in my role as an Acting Chairman\footnoteRus{и.~о.\
председател} of the Committee on Geological Time\footnoteRus{Комитета по
геологическому времени}, having been elected at the Soviet Union Radiological
Conference\footnoteRus{Всесоюзной Радиологической конференцией}, and to acquire
a public debate of this question.  This took place under my chairmanship at the
premises of the Geological Commitee, where I placed the condition that we, as
insufficiently competent in dialectical philosophy, would only address the
scientific side of the phenomenon.  The striking ignorance of the basic facts
and achievements in the area of radiogeology of all philosophers and many
geologists became undeniably clear to all at that session, attended by a few
hundred geologists and philosophers.  We were able to freely develop our work
to a large degree thanks to the fact that the philosophical leaders of the
Geological Committee soon proved to be heretics according to the official
interpretation of dialectical materialism, and were excluded from the
Committee.  However, they still did harm---weakened our scientific work by a
few years.

The phenomenon which was manifested here---errors in the interpretation of
dialectical materialism by official representatives of the philosophy---is an
everyday and widespread phenomenon of our life.  There are a few philosophers,
whom it didn't suit to retract the philosophical theses set forth by them,
which has been explained by an unconscious mistake, or a conscious one, by a
hidden departure from the official philosophy, or, even, by a conscious
interference with the government.  The wide manifestation of this phenomenon,
totalling hundreds of our philosophers-dialecticians, indicates the clear to
every scientist difficulty in the application of the dialectical method in the
current scientific environment.  For, as is clear from §153, there has been not
one prominent thinker from among the founders of dialectical materialism
throughout the historical course of its development, who has given a complete
treatment of this philosophy, thought through to the end.  It has been created
by them in the dust of fights and polemics, from case to case.

None of them has made a complete presentation, and the attempts by less
prominent thinkers, unavoidably proved to be ephemeral.  Errors were found in
them, they were revoked from circulation, one was to never refer to them.  That
continued tens of times, and there remained no presentation, which could be
considered firm.  The present official presentation of both dialectical
materialism, and of the history of the Communist Party, whose ideology this is,
is dated 1936--1937, and there is no certainty than in a year or two they would
not require new reworking.

I have had the occasion to, also, encounter other manifestations of this
scientific environment.  Inexplicably, the Kant-Laplace hypothesis and the
acceptance of the possibility of abiogenesis were connected to dialectical
materialism, and their negation was considered unacceptable from a dialectical
standpoint.  Such a presentation met censorial difficulties.  Already in 1936
in my report \rtitle{On the Problems of Biogeochemistry}, I ran into objections
of that kind at the session of the Academy.  And I was able to establish the
presently unscientific character of the Kant-Laplace hypothesis, and its
incompatibility with radiogeological data the next year in my official speech
at the International Geological Congress\footnoteRus{Международном
геологическом конгрессе} to the tacit agreement of our geologists, including
those considering themselves dialecticians.

In this case the question is not about the interference of dialectical
materialism with the scientific work of the naturalist in the manner indicated
earlier.

Principally, the naturalist cannot deny the correctness and usefulness of the
interference of philosophers in one's scientific work in many cases, when what
is being dealt with are scientific theories, hypotheses, generalizations of a
non-empirical character, cosmogonic constructs.  Here the naturalist
unavoidably treads upon philosophical terrain.

Even here scientific thought finds itself in a condition, which interferes with
its correct scientific work, in our country.  In this case, our scientific
thought conflicts with an obligatory philosophical dogma, with a definite
philosophy, which, as we have seen, has no firm presentation.  This dogma, with
the lack of free scientific and philosophical investigation in our country,
with the extreme centralization of advance censorship, and all means of
dissemination of scientific knowledge---by way of printed or spoken word---in
the hands of government power, is accepted as obligatory for all, and is
introduced in popular life through the full power of government.

\begin{flushright}
								   1936--1938.
\end{flushright}
