\Chapter{%
The structure of scientific knowledge as a manifestation of the noosphere, the
geologically new state of the biosphere resulting from this knowledge.  The
historical course of the planetary manifestation of Homo sapiens by means of
its creation of a new form of cultural biogeochemical energy, and the noosphere
associated with it.}

100. The sciences of the biosphere and its objects, i.e. all humanities
without exception, natural sciences, in the term's own meaning, (botany,
zoology, geology, mineralogy, etc.), all engineering sciences---applied
sciences in the general meaning of the term---are areas of knowledge, which are
maximally accessible to mankind's scientific thought.  Here millions of
millions of incessantly scientifically established and systematized facts,
which are the results of organized scientific work, are concentrated, and are
unstoppably increasing, quickly and consciously, with every generation,
beginning with the 15th--17th centuries.

In particular, the scientific disciplines of the constitution of means of
scientific knowledge, inseparable from the biosphere, can be viewed
scientifically as a geological factor, as a manifestation of the biosphere's
organization.  These are sciences ``of the spiritual'' work of the human
individual in one's social environment, sciences of the brain and organs of
sense, the problems of psychology and logic.  They give rise to the search for
the fundamental laws of human scientific knowledge, that power which has, in
our geological age, transformed the biosphere encompassed by mankind into a
natural body, new in its geological and biological processes---into a new
state, into the noosphere,\footnote{
	\foreignlanguage{french}{\fullcite{leroy1928origines-p37-57}}
} to whose consideration I shall return below.\footnote{
	See \foreignlanguage{russian}{\cite[Гл.~21]{vernadsky1987himicheskoe}}}

Its emergence in the history of the planet, beginning intensively (on the
scale of historical time) a few tens of thousands of years ago, is an event of
great importance in the history of our planet, connected, in the first place,
with the growth of sciences about the biosphere, and is, obviously, not
accidental.\footnote{
	I will return to this process later.  Here I only note Le Roy's thought
	(1928): \foreignquote{french}{Deux grands faits, devant l'esquels tous
	les autres samblent presque svanouir, dominent dans l'histoire passe de
	la Terre: la vitalisation de la matire, puis l'hominisation de la
	vie.}---Op. cit., p.47.  \enquote{Two major facts, in comparison to
	which all others seem almost unnoticeable, predominate in the history
	of the Earth: the vitalization of matter, and the humanization of life.
	The first one is hypothetical, but the beginning of the second is
	clearly visible.}}

We can say that, in this manner, the biosphere is the main area of scientific
knowledge, even if we are only now beginning to differentiate it scientifically
from our surrounding reality.


101. It is clear from what has been said, that the biosphere corresponds to
that, which in the thought of naturalists and in most of philosophical
thought, in the cases where they were not concerned with the Cosmos as a
whole but remained within the limits of the Earth, corresponds to Nature as
usually understood, the Nature of the naturalist in particular.

However, this nature is not amorphous and shapeless, as it has been considered
for centuries, but has definite, very precisely delineated
structure,\footnote{
	This ``structure'' is very peculiar.  It is not a mechanism or anything
	motionless.  It is dynamic, always variable, moving, changing at every
	moment, and never returning to a previous type of equilibrium.  It is
	closest to a living organism, differing, however, from it in the
	physical-geometrical state of its space.  The space of the biosphere is
	physically-geometrically inhomogeneous.  I think that it is convenient
	to define this structure by means of a special concept of organization.
	See \fullcite{vernadsky1934problemy1, vernadsky1980problemy}.
} which must, as such,
be reflected, and considered in all conclusions and results concerning Nature.

It is especially important in scientific research that this is not forgotten
and that it is taken into account, since unconsciously, opposing the human
individual to Nature, the scientist and thinker gives in to the greatness of
Nature above the human individual.

But life in all of its manifestations, the manifestation of the human
individual included, radically changes the biosphere in such a degree that not
only the agglomeration of indivisible units of life, but, in a few problems,
also the single human individual in the noosphere could not be left without
attention in the biosphere.


102. Living nature\footnoteTransl{
	A literal translation of the Russian expression for the living part of
	nature.
} is a main characteristic of the manifestation of the biosphere, it is the
very distinction of the biosphere from the other earth envelopes.  The
structure of the biosphere is characterized, first of all, and most of all, by
life.

We shall see further on (§135) that between the physical-geometrical
properties of living organisms---they are manifested in the form of their
agglomerations in the biosphere---living matter, and those properties of inert
matter, which constitutes the dominant part of the biosphere by weight and by
number of atoms, there is in several respects an impassible gulf.  Living
matter is a carrier and creator of free energy absent from any other earth
envelope on such a scale.  This free energy---biogeochemical
energy\footnote{
	The concept of biogeochemical energy was introduced by me in 1925 in a
	still-unpublished report to the R.~Rosenthal fund in Paris. (The fund
	does not exist any more.)  This fund gave me the ability to work
	without interruption for two years.  The concept has been presented by
	me in print in numerous articles and books:
	\begin{itemize}
	  \item \foreignlanguage{russian}{\cite[30--48]{vernadsky1926biosfera}};
	  \item \foreignlanguage{french}{\cite{vernadsky1926etudes1,
		  vernadsky1927etudes2}};
	  \item \foreignlanguage{russian}{\cite{vernadsky1926razmnozhenii1},
		  \cite{vernadsky1926razmnozhenii2}};
	  \item \foreignlanguage{french}{\cite{vernadsky1926multiplication1,
		  vernadsky1926multiplication2}};
	  \item \foreignlanguage{russian}{\cite{vernadsky1927bakteriofag}}.
	\end{itemize}
	\parenNoteAuthPfx{Ed.} For the R.~Rosethal fund's report
	\foreignlanguage{russian}{\rtitle{Живое вещество в биосфере}} see:
	\foreignlanguage{russian}{\cite[555--602]{vernadsky1994zhivoe}}
}---encompasses the whole biosphere and generally determines all of its
history.  It gives rise to and sharply changes the intensity of the migration
of chemical elements constituting the biosphere, and determines their
geological significance.

A new form of this energy, even greater in its intensity and complexity, has
been created and has been quickly increasing in its significance in the domain
of living matter during the last ten thousand years.  This new form of energy,
connected with the activity of human societies, of the genus Homo and others
(Hominidae) close to it, preserves the manifestation of the usual
biogeochemical energy, but at the same time gives rise to a new kind of
migration of chemical elements, leaving, in its variety and power, the usual
biogeochemical energy of living matter on the planet far behind.

This new form of biogeochemical energy, which can be called energy of human
culture, or cultural biogeochemical energy, is the form of biogeochemical
energy, which is presently creating the noosphere.  Later on I shall return to
a more detailed presentation of our knowledge of the noosphere and its
analysis.  But it is now necessary to sketch its manifestation on the planet.

This form of biogeochemical energy is characteristic not only of Homo sapiens,
but also of all other living
organisms.\footnote{
	\foreignlanguage{russian}{\cite[30--48]{vernadsky1926biosfera}}.  See
	\foreignlanguage{russian}{\cites[330--341]{vernadsky1994zhivoe}{vernadsky1926razmnozhenii1}{vernadsky1926razmnozhenii2}}.
	Published under the title \foreignlanguage{russian}{\rtitle{О
	размножении организмов и его значении в строении биосферы}} in the book
	\foreignlanguage{russian}{\cite[75--101]{vernadsky1992trudy}}.
}  It is, however, negligible in them in comparison to the usual biogeochemical
energy, and has a hardly noticeable effect on the balance of nature, and that
only in geological time.  It is connected to the psychological activity of
organisms, to the development of the brain in the highly developed
manifestations of life, and is expressed in a form resulting in the
transformation of the biosphere into a noosphere only with the emergence of the
human mind.

Its manifestation in mankind's predecessors has been produced, apparently,
over hundreds of millions of years, but it could be expressed in the form of a
geological force only in our time, when Homo sapiens has encompassed with our
life and cultural work the whole biosphere.


103. The biogeochemical energy of living matter is determined, above all, by
the reproduction of organisms, and by their inevitable tendency, determined by
the energetics of the planet, toward a minimum of free energy---it is
determined by the fundamental laws of thermodynamics, corresponding to the
existence and stability of the planet.

It is expressed in the respiration and feeding of organisms---``laws of
nature'', which have not been discovered in their mathematical expression to
this day, but the task of searching for whose expression was clearly laid out
already in 1782 by C. Wolf at the St. Petersburg Academy of
Sciences\footnoteRus{Петербургской Академии наук} at the
time.\foreignlanguage{russian}{\footcite{vernadsky1954sochineniya-himicheskie}}

Obviously, this biogeochemical energy, in this form, is characteristic of Homo
sapiens, as well.  It is, as with all other living organisms, a species
characteristic,\footnote{
	On the species charactestic see \cite{vernadsky1930considerations}.
} and seems unchangeable to us in the course of historical time.  The other
form, of ``cultural'', biogeochemical energy is also unchanging, or hardly
changing for other organisms.  This other form is expressed in the everyday and
in the technical conditions of organisms' life---in their movement, in their
daily activity and construction of dwellings, in the transportation of their
surrounding matter, etc.  It, as I have already indicated, comprises a
negligible fraction of their biogeochemical energy.

With mankind, this form of biogeochemical energy, associated with the human
mind, grows and increases in the course of time, quickly taking first place.
This growth is possibly related to the growth of the mind itself---apparently,
a very slow process (if it, in fact, occurs at all)---but mainly---with the
increase of the precision and depth of its use, associated with the conscious
change of the social setting, and, particularly, with the growth of scientific
knowledge.

I shall proceed from the fact that the skeletons of Homo sapiens, including
the skull, over a hundred millenia gives us no basis for viewing them as
belonging to another species of man.  This is admissible only under the
condition that the brain of Paleolithic man does not differ in any significant
degree in its structure from the brain of contemporary man.  At the same time,
there is no doubt that the mind of that man from the Paleolithic for this
species of Homo cannot bear comparison to the mind of contemporary man.
Thence it follows that the mind is a complex social structure, built, for the
man of our times, just as for the Paleolithic man, upon the same nervous
substrate, but in a different social setting, which is being composed through
time (space-time, in essence).

Its change is the basic element, leading, in the end, to the transformation of
the biosphere into a noosphere in the obvious manner, above all---through the
creation and growth of the scientific understanding of our surroundings.


104. The emergence of cultural biogeochemical energy on our planet is a major
factor in its geological history.  This had been prepared for through all
geological time.  The main, decisive process here is the maximum manifestation
of the human mind.  But this is, in essence, inseparable from all
biogeochemical energy of living matter.

The life of the migration of atoms in the living process connects in a unified
whole all migrations of atoms of the biosphere's inert matter.

Organisms are alive only while the material and energetic exchange between
them and their surrounding biosphere is uninterrupted.\footnote{
	The complete absence of exchange for the latent forms of life cannot be
	considered proven, yet.  It is extremely slow---and, possibly, in a few
	cases there is no migration of atoms indeed---it could become
	noticeable only in geological time.
}  Colossal definite chemical cyclical processes of atomic migration, in which
living organisms enter as a lawful, inseparable, often main part of the
process, are being clarified in the biosphere.  These processes are constant in
geological time and, for example, the migration of magnesium atoms incorporated
in chlorophyll stretches uninterruptedly for at least two billion years through
innumerable, genetically related generations of green organisms.  Living
organisms, uninterruptedly and inseparably connected to the biosphere by such
atomic migrations, comprise a lawful part of its structure.

This must never be forgotten in the scientific study of life and in scientific
statements about any of its manifestations in Nature.  We cannot overlook the
fact that an uninterrupted connection---material and energetic of the living
organism with the biosphere, a completely definite connection, ``geologically
eternal'', which can be scientifically expressed precisely---is always present
in our every scientific approach to life and must be reflected in all of our
logical conclusions and results about it.

In moving to the study of the geochemistry of the biosphere we must, first of
all, precisely estimate the logical significance of this connection,
unavoidably entering all of our constructs related to life.  It does not
depend on our will, and cannot be excluded from our experiments and
observations, but must always be taken into account as something fundamental,
inherent in life.

The biosphere must, in this manner, be reflected in all of our scientific
statements without exception.  It must be manifest in every scientific
experiment and scientific observation---and in every thought of the human
individual, in every speculation, from which the human individual---even
thought---cannot escape.

Therefore, the human mind can be maximally expressed only with the maximum
development of the basic form of the biogeochemical energy of mankind, i.e.
with its maximum reproduction.


105. The potential for covering the surface of the whole planet by means of
reproduction of an organism of a single species is characteristic of all
organisms, since the law for reproduction is expressed in the same form for
all of them, in the form of a geometrical progression.  I have already
indicated the major significance of this phenomenon long ago,\footnote{
	See \cites[37--38]{vernadsky1926biosfera}[335, 413--424]{vernadsky1994zhivoe}{vernadsky1926etudes1}[59--83]{vernadsky1940biogeohimicheskie}[75--101]{vernadsky1992trudy}.
} and I will return to it at the appropriate place in this book.

The phenomenon of covering the whole surface of the planet by a given single
species can be seen widely developed for aquatic life in the microscopic
plankton of lakes and rivers, and for a few forms of---essentially also
aquatic---microbes, from the surface layers of the planet, propagating through
the troposphere.  Among larger organisms we observe this in almost full
measure in a few plants.

This has begun to be manifest for mankind in our times.  The whole globe and
all the seas have been encompassed by him in the 20th century.  Thanks to the
success of communications, man can be in constant communication with the whole
world, cannot be solitary and get himself lost in the grandiosity of the
earth's nature anywhere.

Presently, the number of the human population on Earth has reached
unprecedented height, nearing two billion people, despite the fact that murder
in the form of war, hunger, malnourishment, constantly affecting hundreds of
millions of people, extremely diminishes the course of the process.
Negligible time from the geological point of view would be necessary, hardly
more than a few hundred years, to end these relics of barbarism.  This could
be freely done even now; the ability to end this condition is presently in the
hands of mankind, and the reasonable will will inevitably go down that path,
because it corresponds to the natural tendency of the geological process.  It
should be so all the more, since the means to do it are increasing rapidly and
almost tempestuously.  The real significance of population masses, suffering
the most from this, is irrepressibly increasing.

The number of people inhabiting the planet began increasing, say, about 15--20
thousand years ago when mankind became less influenced by food shortage in
relation to the discovery of agriculture.  Apparently it was then, say, about
10--8 thousand years ago that the first population explosion
occurred.\footnote{\cite{childe1937man-p78-79}} G.~F.\ Nikolai (in
1918--1919)\footnote{\cite{nikolai1919biologie-p54}.} attempted to estimate the
actual population increase of mankind and the development of agriculture
numerically, the actual population of the planet by mankind.  According to his
calculations, taking the total territory of the Earth, there are 11.4 people
per square kilometer, which constitutes $2.10^{-4}\%$ of the possible
population.  Considering the amount of energy received from the Sun,
agriculture allows 150 people to be sustained per $1\,{\mathrm{km}}^2$, i.e.\
for the whole Earth (land area) it must be $22.5\cdot 10^9$ units, i.e. 22--24
times more than live presently.\footnote{\cite{nikolai1919biologie-p60}.}  But
mankind acquires energy for sustenance and for living not only through
agricultural labor.  Considering this possibility, Nikolai, for example,
estimated that the Earth in the historical age started in our time, using new
energy sources, could be populated by three hexillion people ($3\cdot
10^{16}$), i.e. more than tens of millions of times more than the present
number of mankind.  These numbers must be highly increased at the present
moment, when more than 20 years have passed since Nikolai's calculations, since
mankind can, in practice, presently use sources of energy, which Nikolai could
not imagine in 1917--1919---energy, connected to the atomic nucleus.  Must now
say, more simply, that the source of energy, which is encompassed by the human
mind in the energetic age of mankind, which we are entering---is practically
unlimited.  Hence, it is clear that the cultural biogeochemical energy (§17)
shares the same characteristic.  According to Nikolai's calculations, machines
increased mankind's energy more than ten times in his time.  We cannot
presently give a more precise calculation; however, recent accounts of the
American Geological Committee\footnoteRus{американского Геологического
комитета} indicate that water power, presently in use all around the world,
reached 60 million horsepowers at the end of 1936: it increased by 160 per cent
in 16 years, mainly in North America.\footnote{\fullcite{blair1938water}.}
Thanks to that, we must already increase Nikolai's calculations more than one
and a half times.

In essence, all of these calculations about the future, expressed in a
numerical form, have no significance, since our knowledge of the energy
accessible to mankind is, we can say, rudimentary.  Of course, the energy
accessible to mankind is not an infinite amount, since it is determined by the
size of the biosphere.  The limit to the cultural biogeochemical energy is
also determined by this.

We shall see (§138) that there is also a limit to the basic biogeochemical
energy of mankind---the speed of expansion of life, the limit of mankind's
reproduction.

The speed of reproduction\footnote{
	On the speed of expansion of life see below.  See
	\foreignlanguage{french}{\cite{vernadsky1926etudes1}};
	\foreignlanguage{russian}{\cite[413--424, 437--444]{vernadsky1994zhivoe}};
	\foreignlanguage{russian}{\cite[118--125]{vernadsky1940biogeohimicheskie}};
	\foreignlanguage{russian}{\cite[Гл.~20]{vernadsky1965himicheskoe}.}
}---the magnitude $V$ considered, in essence, by Nikolai, is based on the
actually observed population of the planet by mankind in unfavorable for his
life conditions.  We shall also see, further on, that there are still unknown
to us phenomena in the biosphere, which lead to a stationary maximum quantity
of living units per hectare which can exist in a given geological age in a
given condition of the biocenosis.


106. We can record the human population on the planet with any precision only
since the beginning of the 19th century.  It is still calculated with a high
percentage of possible error.  Our knowledge has considerably increased during
the last 137 years, but can still not be considered having reached the
precision which contemporary science may require.  For earlier times the
numbers are only provisional.  Still, they are helping us in the understanding
of the occurring process.

The following data may have significance for us in that aspect.

The number of people in the Paleolithic likely reached a few million.  It is
possible that it began with one family.  However, the opposite view is also
possible.\footnote{See E.~Le Roy. [The author's note has not been found.
\noteAuth{Ed.}]}

In the Neolithic we are likely dealing with tens of millions on the whole
surface of the Earth.  It is possible that even in historical time it did not
reach a hundred million, or that it did not exceed that number by
much.\footnote{
	\cite{weinberg1922dvuhdesyatitysyachiletiyu-p21} (assumes a
	population of 80 million at the beginning of our age).}

G.~F.\ Nikolai supposed that the human population of the planet increases by 12
million people annually for 1919, i.e. increases by, say, 30 thousand a day.
According to the critical report of the Kulischers
(1932)\footnote{\cite{kulischer1932kriegs-p135}.} the world population was 850
million in 1800 (A.~Fischer takes it to be 775 million).  We can assume its
number for the white race to be 30 million in 1000, 210 million in 1800, 645
million in 1915.  For the whole population in 1900, according to the
Kulischers---about 1,700 million, but according to A.  Hettner
(1929)\footnote{\cite{hettner1929gang-p196}}---1,564 million, and 1,856 million
in 1925, according to the same.

That number has evidently reached about two billion, more or less, at present.
The population of our country (about 160 million) comprises about 8\% of the
world population.  The world population is rapidly increasing, and, evidently,
the percentage of our population is increasing, since its growth is greater
than the average population growth.  In general, we should expect to
significantly exceed 2 billion by the end of the century.


107. The reproduction of organisms, i.e. the manifestation of biogeochemical
energy of the first type, without which there is no life, is inseparable from
man.  However, at his very differentiation from the mass of life on the
planet, man had already mastered the use of tools, even if they were very
primitive, which allowed him to increase his muscle power, and were the first
manifestation of contemporary machines, which distinguished him from the other
living organisms.  The energy by which they were powered, however, was
produced through the feeding and breathing of man's very organism.  It has
probably been hundreds of thousands of years already since man---genus
Homo,---and his predecessors mastered the use of wooden, bone, and stone
tools.  The skill of making and using those tools was being developed slowly,
in the course of many generations, skill---the mind in its first
manifestation---was being perfected.

Such tools can be observed already in the most ancient Paleolithic, 250
thousand--500 thousand years ago.

A significant part of the biosphere was living through critical times during
that period.  Apparently, a radical change---in its water and heat
regime---began already in the Pliocene, an ice age began and was developing
throughout the whole period.  We are, apparently, still living during the
dying out of its last manifestation, whether temporary or permanent is still
unknown.  We can see strong oscillations in the climate during these half a
million years; relatively warm periods---continuing for tens and hundreds of
thousands of years---replaced in the northern and southern hemispheres
periods, during which masses of ice which reached depth of up to a kilometer,
for example, in the vicinity of Moscow, moved slowly---on the historical
scale.  They disappeared a thousand and seven\footnoteTransl{
	The other English translation has seven thousand here, and notes ``Now
	we know that in the environs of Leningrad the ice has disappeared about
	12 thousand years ago.''
} years ago in the Leningrad region, and are still occupying Greenland and
Antarctica.  Apparently, Homo sapiens, or his closest predecessors, formed not
long before the onset of the ice age, or during one of its warm periods.  Man
survived the coldness during that time with hardship.  That was possible thanks
to a great discovery in the Paleolithic---the mastery of fire.

This discovery was made in one--two, possibly a few more, places and slowly
spread among the population of the Earth.  Apparently, we have a general
process of great discoveries here, where not the mass activity of mankind,
smoothing out and amending particularities, but rather the manifestation of
the separate human individuality plays a role.  We can trace that in the more
recent time and in very many cases, as we shall see later (§134).

The discovery of fire is the first case of a living organism mastering and
harnessing a force of nature.\footnote{\cite{childe1937man-p56}. Cp.:
	\cite{frazer1930myths}.}

This discovery is the foundation, as we shall now see, of all the following
increase of mankind, and of our present power.

This increase, however, took place extremely slowly, and it is hard for us to
imagine the conditions, under which it could occur.  Fire was already known to
the ancestors of the genus, or to the predecessors of that species of Hominid,
who is building the noosphere.  The latest discovery in China reveals the
cultural remains of Sinanthropus, which indicate his wide use of fire,
apparently, long before the last glaciation of Europe, a hundred thousand
years before our time.  We presently have no data of any credibility about how
that discovery was made by him.  Sinanthropus already possessed a mind, had
primitive tools, used speech, performed burial rites.  This was already a
human, but foreign to us in many morphological characteristics.  Also, the
possibility that he is one of the predecessors of the contemporary human
population of China has not been eliminated.\footnote{
	On Sinanthropus's technology, and on his use of fire see
	\cite{bogaevsky1936tehnika-p26-27}.  Pithecanthropus, who lived
	earlier, at the very beginning of the Pleistocene, hardly more than 550
	thousand years ago, also possessed fire.  Ср.:
	\cite{bogaevsky1936tehnika-p11.67}.  The use of fire by Pithecanthropus
	cannot be considered proven, yet, but is very likely.}


108. The discovery of fire is all the more remarkable because the
manifestation of fire and light emission in the biosphere had been a
relatively rare phenomenon before mankind, and had manifested mainly when
taking up a large space, in the form of cold light, in such forms as airglow,
aurora borealis, sheet lightning, stars and planets, noctilucent clouds.  The
Sun alone, the source of life, was simultaneously a bright manifestation of
light and heat, was lighting and heating the planet.

Living organisms had developed a manifestation of cold light long ago.  It
appeared in such large-scale phenomena as marine bioluminescence, usually
taking up hundreds of thousands of square kilometers, or the luminescence in
marine depths, whose significance is just beginning to be clarified.  Fire,
accompanied by high temperature, was manifested in local phenomena, rarely
taking up large spaces like volcanic eruptions.

But these colossal on the human scale phenomena, obviously, because of their
destructive force, could in no way have aided the discovery of fire.  Man had
to look for it in closer to him, and less scary and dangerous manifestations
of nature than volcanic eruptions, still exceeding mankind in their
manifestation of power.  We are only beginning to approach using them in
practice, in conditions which were inaccessible and unthinkable to Paleolithic
man.\footnote{
	Mankind has obtained superheated vapor at a ${140}^\circ C$ temperature
	as a source of power only in the 20th century with the aid of drilling
	in Larderello under Le Conte's initiative.  Still later, this method
	was greatly developed in Soffioni, in New Mexico, in Sonoma.  Parsons,
	before his death, worked on an implementable project to obtain an
	unlimited, from mankind's point of view, source of energy from the
	inner heat of the earth's crust with the aid of deep drilling.  The
	attempt to obtain energy from the cold depths of the ocean, which the
	French Academician Claude did not realize only because of criminal
	hooliganism in 1936, can be considered analogous.  Undoubtedly, we have
	in these phenomena a practically inexhaustible force in mankind's
	hands.}

He had to look for phenomena giving heat and fire in his surrounding everyday
phenomena of life; in his habitat---in the woods, steppes, among living
nature, with which he was in close (long forgotten by us) connection.  Here he
could encounter fire and heat in a safe form in numerous everyday phenomena.
These were, on the one hand, fires, the burning of living and dead matter.
They were the very sources of fire used by Paleolithic man.

He burned wood, plants, bones, that which produced fire around him without his
will.  This fire was due to two very different reasons before man's emergence.
On the one hand, lightning caused forest fires, or set dry grass on fire.
Mankind still suffers from fires caused this way.  The natural conditions in
the ice age, especially in interglacial ages, could have been even more
favorable for lightning phenomena.  There was, however, another cause which
produced fire independently of mankind.

That was the biological activity of lower organisms, which lead to fires in
dry steppes,\footnote{
	The spontaneous ignition of dry grass in the steppes, in pampas, in
	forests has sometimes been denied.  Presently the source of fires is
	almost always man, but there are cases which, it seems to me,
	undoubtedly indicate the possibility of spontaneous ignition in steppes
	under the direct action of the sun.  The cause remains unclear.  About
	such cases see \cite{popping1835reise-p398}.
	\cite{carpenter1920naturalist-p76-77}.
} to the burning of bituminous coal layers, to the burning of peat bogs, which
continued throughout a number of human generations and gave a convenient way of
obtaining fire.  We have direct indications of such bituminous coal fires in
Altai, in the Kuznetsk basin, where they occurred in the Pliocene and
post-Pliocene, but where they also occurred in historical time, and where we
still have to deal with them.  The causes of these fires are still not
completely clear, but all indications are that it is unlikely that we have
phenomena of purely chemical spontaneous combustion, i.e.  intensive oxidation
of coal fragments with oxygen from the atmosphere, or its spontaneous ignition
due to heat released during oxidation of sulphur compounds of iron in the
coal.\footnote{
	See \cite{usov1924sostav-p58, usov1933podzemnye-p34,
		obruchev1934podzemnye-p83-85}.
	J.~F.\ Hermann\footnoteRus{И.~Ф.\ Герман}, who discovered
	the Kuznetsk bituminous coal basin, already indicated these phenomena
	in 1796.  See \cite{hermann1793notice}.  Cp.
	\cite{jaworsky1933erdbrande, yavorski1932kamennougolnye}.}

The most probable source is the biochemical phenomena associated with the
biological activity of thermophilic bacteria.  We have the direct observations
of B. L. Isachenko\footnoteRus{Б. Л. Исаченко} and N. I.
Malchevskaya\footnoteRus{Н. И.  Мальчевская}\footnote{
	See \foreignlanguage{russian}{\cite{isachenko1936biogennoe}}.
} for peat bogs in recent times.

This phenomenon presently requires careful study.


109. Such regions of warm winter and summer, as well as places of outlets of
heat sources, were precious gifts of nature to Paleolithic man, who had to use
them just as they are used, or were used until recently by tribes and peoples
that we still find in a living Paleolithic stage.

Man at that time, with his great attentiveness and closeness to nature,
undoubtedly noticed such places, and must have been using them, especially in
glacial periods.

It is curious that we can observe the use of the same biochemical processes
among the instincts of animals.  This can be observed in the family of the
chickens, with the so-called incubator birds, or large-foots (Megapodiidae) of
Oceania and Australia, which make use of the heat of biological decay, i.e. of
a bacterial process, for the hatching of chicks form eggs, creating large
mounds of sand or dirt mixed with strongly rotting organic
remains.\footnote{
	See \foreignlanguage{russian}{\cite{brem1912zhizn-ptitsy}}.
}  These mounds can reach 4 meters in height, and the temperature in them
reaches no less than ${44}^\circ C$.  Apparently, these are the only birds
possessing such instincts.

It is possible that ants and termites increase the temperature of their
dwellings on purpose.

However, these are weak attempts, incomparable to that planetary revolution,
which mankind has produced.

Man has been using the products of life---dry plants---as a source of energy,
fire.  Numerous myths about its creation have been preserved and
created.\footnote{See \cite{frazer1930myths}.}  But most characteristic is the
fact that man used, for that purpose, methods which he hardly ever observed to
produce fire in the biosphere until his discovery.  The most ancient methods
were, apparently, the transformation of man's muscle power into heat (strong
friction of dry objects), and the making and catching of sparks from stones.  A
complex system for the preservation of fire was developed in the end in
everyday life a hundred, and more, thousand years ago.

The surface of the planet has been changed radically after this discovery.
Fireplaces shone, were extinguished and started everywhere, if only man lived
there.  Mankind was able, thanks to this, to survive the coldness of the
glacial period.

Man was producing fire among living nature, subjecting it to burning.  In this
way, by means of steppe and forest fires, he acquired a force which, in
comparison to that of his surrounding animal and plant world, put him above
the numerous other organisms and became a prototype of his future. Mankind has
mastered other sources of light and heat---electrical energy---only in our
time, in the 19th--20th centuries.  The planet started shining even more, and
we have found ourselves at the beginning of times, whose significance and
future still remain outside of our attention.


110. 


. . .


