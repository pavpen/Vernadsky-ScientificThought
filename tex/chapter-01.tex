\part{Scientific Thought and Scientific Work as a Geological Force in the
Biosphere}

\Chapter{Man and mankind in the biosphere as a lawful part of its living
matter, part of its organization.  Physical-chemical and geometric
heterogeneity of the biosphere: the fundamental organizational
distinction---material-energetic and temporal---of its living matter from its
inert matter.  Evolution of the species, and evolution of the biosphere.  The
manifestation of a new geological force in the biosphere---the scientific
thought of social mankind.  Its manifestation is related to the ice age, in
which we live, with one of the geological phenomena repeating in the history of
the planet, whose cause exceeds the bounds of the Earth's crust.}


\Section % 1
% Inseparability of scientific thought from the living environment.
% Inseparability of the concepts of nature & cosmos from the biosphere &
% science.
Man, as well as everything living, is not a self-sufficient, independent of the
environment natural object.  However, even natural scientists in our time,
counterposing human beings and living organisms in general to the environment
of their life, very often do not take this into account.  But the
inseparability between living organism and its environment cannot presently
raise any doubt among contemporary naturalists.  The biogeochemist proceeds
from it, and strives to understand, express, and establish this functional
dependence precisely, and as deeply as possible.  Philosophers and contemporary
philosophy predominantly do not take into account this functional dependence of
man, as a natural object, and mankind, as a natural phenomenon, on the
environment of their life and thought.

Philosophy cannot sufficiently take this into account, as it proceeds from the
laws of the mind, which is, in one way or another, a final and self-sufficient
criterium for it (even in those cases, like in religious and mystical
philosophies, in which the reach of the mind is, in fact, limited).

The contemporary scientist, proceeding from the recognition of the reality of
one's surroundings, of the world subject to one's investigation---nature, the
cosmos, or world reality,\footnote{
	I will talk about the reality of the cosmos, instead of that of nature,
	here and further.  The concept nature, if we take it in a historical
	aspect, is a complex concept.  It very often encompasses only the
	biosphere, and it is more convenient to use it with just this meaning,
	or even not to use it at all (\autoref{sec:6}).  This would correspond
	to the vast majority of the uses of this concept historically in
	natural science and in literature.  The concept `cosmos' can be,
	perhaps, more conveniently applied to only the part of reality
	encompassed by science, a philosophically pluralistic conception of
	reality is possible at that, where there would be no single criterium
	for the cosmos.
}---cannot adopt this point of view as a basis for scientific work.

Because one presently knows exactly that man is \emph{not} located on a
structureless surface of the Earth, is \emph{not} located in direct contact
with cosmic space in a structureless nature, which is not lawfully connected
with him.  True, even the deeply penetrating contemporary naturalist often, out
of routine and under the influence of philosophy, forgets this, and does not
take it into account in his thought, and does not identify this.

Man and mankind are most closely connected, above all, with the living matter
inhabiting our planet, from which they cannot, in reality, be isolated by any
physical process.  That is possible only in thought.


\Section % 2
% We shall study the scientifically unambiguous 'living matter', and 'living
% natural body', rather than 'life'.
The concept of life and the living is clear to us in everyday life, and cannot
raise scientifically serious doubts in the actual manifestations of it, and in
natural objects corresponding to it---in natural bodies.  It was only in the
$20^{\mathrm{th}}$ century [with the discovery of] filter-passing viruses that
in science appeared facts, compelling us for the first time to ask
seriously---not philosophically, but scientifically---the question: Are we
dealing with a living natural body, or with a non-living natural body---an
inert one?

With viruses the doubt is called for by scientific observations, rather than
philosophical notions.  In this consists the great scientific significance of
their study.  That is presently on a right and firm path.  The doubt will be
resolved, and nothing, except a more precise notion of \emph{living organism},
would give, with this approach couldn't fail to give \dots

Along with this, however, we encounter another kind of doubts in
\emph{science}, called for by philosophical and religious searches.  Thus, for
example, phenomena, concerning the material-energetic environment of
manifestations which are philosophically \emph{common} to both living and inert
natural bodies, are scientifically studied in the works of the Bose Institute
in Calcutta.\footnoteRus{Института Бозе в Калькутте}\footnoteEd{
	The Bose
	Institute\footnoteTransl{\url{http://www.boseinst.ernet.in/index.html}}
	in Calcutta was founded by the Indian scientist Acharya Jagadish
	Chandra Bose\footnoteRus{Бозе Джегдиш Чандра} (1858--1937) in 1917.
	The institute studied the problems of physics, biophysics, inorganic
	and organic chemistry, biochemistry, the physiology of plants,
	selection, microbiology, etc. [---Ed.]
} They are not characteristic of, are weakly expressed in inert natural bodies,
and are strongly manifested in living ones, but are common to both.

This area, if it exists in the form in which Bose tried to establish it, of
phenomena common to inert and living natural bodies introduces nothing new in
the sharp distinction between them.  The distinction must manifest itself in
this area, as well, if only the latter's existence would be proven.

We must approach phenomena here, as well, not in the aspect in which Bose
approached them, not as phenomena of \emph{life}, but as phenomena of living
natural bodies, of \emph{living matter}.

To avoid any misunderstanding, I shall avoid the concepts `life', `living' in
all further exposition, since, if we stepped outside those
[phenomena],\footnoteTransl{inserted by transl. [---Pav]} we would inevitably
go beyond the limits of the phenomena of life studied in science, into a
foreign area or science---the area of philosophy, or, as is taking place in the
Bose Institute, into a new area of new material-energetic manifestations common
to all natural bodies of the biosphere, one lying outside the bounds of the
fundamental question of living organism, and living matter, which we are
presently interested in.

I shall, therefore, avoid the terms and concepts `life', and `living', and
limit the area which is subject to our investigation to the concepts
\emph{`living natural body'}, and \emph{`living matter'}.  Each living organism
\emph{in the biosphere}---natural object---is a living natural body.  \emph{The
living matter of the biosphere is the complex of living organisms in it.}

`Living matter', so defined, is a concept, completely precise, and fully
encompassing the objects studied by biology and biogeochemistry.  It is simple,
clear, and cannot raise any doubt.  We study only the living organism and its
complexes in science.  They are scientifically identical with the concept of
life.


\Section % 3
% Living matter is in a state of organization, which is determining for the
% biosphere.  Living matter's cosmic significance, and the constantly
% increasing extent of the biosphere.
Man, like every living natural (or naturally occurring)\footnoteRus{природное
(или естественное)} body, is inseparably connected with a certain geological
envelope of our planet---\emph{the biosphere}, clearly distinct from the rest
of its envelopes, with a structure which is determined by its specific
\emph{state of organization},\footnoteTransl{организованностью, \ie\ 
organizedness, or being (constantly) organized. [---Pav]} and occupying a
lawfully expressible place in it as a distinct part of the whole.

Living matter, just like the biosphere, possesses its peculiar state of
organization, and can be viewed as a lawfully expressible \emph{function of the
biosphere.}

\emph{A state of organization} is not a mechanism.  It sharply differs from a
mechanism in that it is constantly in a state of
becoming,\footnoteTransl{становлении, \eg\ formation. [---Pav]} of motion of
all of its smallest material and energetic particles.  We can express this
state of organization in the course of time---in a generalization of mechanics,
and in a simplified model---as being such that none of its points (material or
energetic) lawfully returns to, ends up in a place, in a point of the biosphere
which it occupied at any earlier moment.  It can return to one of them only on
the order of mathematical accident, of very small probability.

The Earth's envelope, the biosphere, embracing the whole globe, has clearly
distinct dimensions, is determined to a large degree by the existence of living
matter in it---\emph{populating}\footnoteRus{заселена} it.  There is a
constant material and energetic exchange, materially expressed in the motion of
atoms brought on by living matter, between its inert non-living part, its inert
natural bodies, and the living matter inhabiting it.  This exchange in the
course of time is expressed as lawfully changing, constantly tending toward the
stability of an \emph{equilibrium}.\footnoteRus{равновесием}  The latter
permeates the whole biosphere, and this \emph{biogenic flow of
atoms}\footnoteRus{биогенный ток атомов} creates the biosphere to a large
extent.  The biosphere is thus connected inseparably and inherently with the
living matter populating it throughout the whole duration of geological time.

The planetary, cosmic significance of living matter is distinctly expressed in
this biogenic flow of atoms, and in the energy connected with it.  That, since
the biosphere is that single envelope of the Earth, in which cosmic energy,
cosmic radiation, mainly radiation from the Sun, which maintains the dynamic
equilibrium, the state of organization: $\mathrm{biosphere} \leftrightarrow
\mathrm{living matter}$, constantly penetrates.

The biosphere stretches from the surface of the geoid up to the boundary of the
stratosphere, penetrating in it; it, however, would be unlikely to be able to
reach the ionosphere---the Earth's electromagnetic vacuum, which is just now
entering the scientific consciousness.  Living matter reaches below the surface
of the geoid into the stratisphere, and into the top regions of the
metamorphic, and of the granitic envelope.  It rises up to
20--$25\,\mathrm{km}$ above the surface of the geoid, and extends down to
4--$5\,\mathrm{km}$ below that level on average.  These boundaries change in
the course of time, and, there are places of, it is true, small extent where
they are far beyond these.  Apparently, living matter must reach deeper than
$11\,\mathrm{km}$ at places in the depths of the ocean, and its presence has
been established deeper than $6\,\mathrm{km}$.\footnoteEd{
	Ocean floor organisms have indeed been observed at all depths of the
	world ocean, including at greater than $11\,\mathrm{km}$.  (See
	\foreignlanguage{russian}{\cites{belyaev1966donnaya}{belayev1989glubokovodnye}}.)
	[---Ed.]
}  We are just now living through the penetration of mankind, always
inseparable from other organisms---insects, plants, microbes,---into the
stratosphere, and by this means living matter has already exceeded
$40\,\mathrm{km}$ above the surface of the geoid, and is quiclky rising.

Apparently, a process of incessant expansion of the boundaries of the
biosphere: its population with living matter, is observable in the course of
geological time.


\Section % 4
\emph{The state of organization of the biosphere}---the state of organization
of living matter---must be viewed as an equilibrium, which is changeable,
always oscillating around a precisely expressible mean, not in historical, but
in geological time.  The shifts, or oscillations of this mean are constantly
manifested not in historical, but rather in geological time.  In the course of
geological time, in the cyclical processes which are characteristic of the
biogeochemical state of organization, no point (for example, atom or chemical
element) ever returns to a position identical with a previous one for eons.

This characteristic of the biosphere was expressed very prominently and vividly
by Leibnitz[1646--1716] in one of his philosophical reflections, it seems to
me, in the \rtitle{Theodicy}.\footnoteTransl{
	See \refsmartcite{leibnitz1985theodicy, leibnitz1900oeuvres-v2}.
}  He was among a large company from the high society in a large garden, he
recounts, at the end of the $18^\mathrm{th}$ c., and Leibniz, speaking of the
infinite variety of nature, and of the infinite perfectability of the mind's
precision,\footnoteRus{бесконечной четкости ума} indicated that two leaves of
any tree or plant are never completely identical.  All efforts of the large
company to find such leaves were, of course, in vain.  Leibniz was reflecting
here not as an observer of nature, discovering this phenomenon for the first
time, but as en erudite, taking it from his readings.  It is possible to trace
that precisely this example of the leaves appeared in philosophical folklore
centuries earlier.\footnote{
	See, for example, \cite[кн.~2,
	с.~54]{carus1913prirode}\nocite{carus1936prirode, carus1851nature}.}

This is manifested for us in everyday life in
\emph{identity},\footnoteRus{личности} in the absence of two identical
individuals, indistinguishable from one another.  It is manifested in biology
in the fact that every mean \emph{individuum}\footnoteRus{индивидуум} of living
matter is \emph{chemically distinct} in its chemical compounds, as, obviously,
also in its chemical elements having \emph{their own} specific compounds.


\Section % 5
% The heterogeneity of the biosphere.  The connection between living and inert
% matter.
Especially characteristic in the structure of \emph{the biosphere is its
physical-chemical, and geometric \emph{(\autoref{sec:47})} heterogeneity}.  It
consists of living and inert matter, which are sharply separate in their
genesis and structure throughout all of geological time.  Living organisms,
\ie\ all living matter, are born from living matter, form generations in the
course of time, which never arise directly, outside of such a living organism,
from any possible inert matter on the planet.  There is, however, an inherent,
unceasing connection\footnoteRus{непрерывная, никогда не прекращающаяся связь}
between inert and living matter, which can be expressed as an incessant
biogenic flow of atoms from the living matter into the inert matter of the
biosphere, and vice versa.  This biogenic flow of atoms is originated by living
matter.  It is expressed in its always unceasing breathing, feeding,
reproduction, etc.

This heterogeneity in the biosphere, unceasing during all geological time, is
the main dominant factor, strongly distinguishing it from all other envelopes
of the globe.

It goes deeper than the phenomena usually studied in natural science---to the
properties of space-time, which scientific thought has approached only in our
time, in the $20^\mathrm{th}$ c.

Living matter encompasses the whole biosphere, creates it, and changes it, but
amounts to a small part of it by mass, and by volume.  Inert, non-living matter
is strongly dominant; greatly diluted gases dominate by volume, hard rocks,
and, to a lesser degree, the liquid salt water of the world ocean---by mass.
Living matter, even in the greatest concentrations, in exceptional cases with
insignificant masses, amounts to tens of percent of the biosphere's matter, and
amounts to hardly one--two hundredths of a percent by mass on average.
Geologically, however, it is the greatest force in the biosphere, and
determines, as we can see, all processes occurring in it, and accumulates vast
free energy, creating the main geologically manifesting force in the biosphere,
whose power still cannot be quantitatively determined, but, possibly, exceeds
all other geological manifestations in the biosphere.

In connection with this, it is convenient to introduce a few basic concepts
which we will be dealing with in all of the following exposition.


\Section \label{sec:6} % 6
% The concepts of natural body (natural object) and natural phenomenon; living,
% inert, and bioinert bodies and phenomena; heterogeneity of the biosphere's
% structure.
Such are the concepts connected with the concepts of natural body (natural
object),\footnoteRus{природного тела (природного объекта)} and natural
phenomenon.\footnoteRus{природного явления}  They have often been referred to
as naturally occurring bodies or phenomena.\footnoteRus{естественные тела или
явления}

Living matter is a natural body or phenomenon in the biosphere.  The concepts
\emph{natural body or natural phenomenon}, little logically studied, are main
concepts of natural science.  There is no need to delve into their logical
analysis for our purposes.  These are bodies or phenomena, formed by natural
processes,---\emph{natural objects}.

Living organisms, living matter, are not the only natural bodies of the
biosphere, but rather the main mass of the biosphere's matter forms non-living
bodies or phenomena, which I will be referring to as
\emph{inert}.\footnoteRus{косными}  Such are, for example, gases, the
atmosphere, rocks, a chemical element, an atom, quartz, serpentine, etc.

In addition to living and inert natural bodies, a great role in the biosphere
play its lawful structures, heterogeneous natural bodies, such as soils, silts,
surface waters, the very biosphere, etc., which consist of living and inert
natural bodies existing at the same time, forming complex, lawful inert-living
structures.\footnoteRus{косно-живые структуры}  I will be calling these complex
natural bodies \emph{bioinert} natural bodies.\footnoteRus{биокосными
природными телами}

The difference between living and inert natural bodies is so great, as we shall
see further on, that the transformation of one into the other is never and
nowhere observed in terrestrial processes; we encounter them nowhere and never
in scientific work.  As we shall see, such a process is deeper than the
physical-chemical phenomena known to us.

The related \emph{heterogeneity of the biosphere's structure}, the sharp
distinction between its matter and its energy in the form of living, and in the
form of inert natural bodies is its main manifestation.


\Section % 7
% The heterogeneity of time (historical and geological) in the biosphere.
One of the manifestations of this heterogeneity of the biosphere consist in the
fact that processes occur completely differently in living matter than in inert
matter, if they are viewed in the aspect of time.  They take place on the scale
of \emph{historical time}\footnoteRus{\emph{исторического времени}} in living
matter, on the scale of \emph{geological
time},\footnoteRus{\emph{геологического времени}} whose `second' is under
decamyriads, \ie\ a hundred thousand years of historical time,\footnote{
	On decamyriads see \cite{vernadsky1935nekotoryh}, [as well as
	\cite{vernadsky1954sochineniya-nekotorye}].
} in inert matter.  This difference is expressed even more sharply outside the
boundaries of the biosphere, and we observe in the litosphere, for the
predominant part of its matter, a state of organization in which the majority
of atoms, as radioactive studies show, are immobile, do not mix, noticably for
us, in the course of tens of thousands of decamyriads---a span of time now
accessible to our measurement.

Not long ago the view that geologists cannot study the manifestation of
geologically long changes, occurring in the age of mankind's existence,
dominated.  In the times of my youth we learned and thought that the change in
climate, orography, the emergence of new species of organisms are not, as a
general rule, detected in geological studies, are not \emph{current
phenomena}\footnoteRus{текущим явлением} for the geologist.  Now this
conceptual circumstance of the naturalist has sharply changed, and we can see
ever more, and more emphatically the geological forces around us.  This
coincided, hardly by accident, with the penetration of the conviction of the
geological significance of Homo sapiens in the scientific consciousness, with
the detection of a new state of the biosphere---the noosphere---and is one of
the forms of its manifestation.  This is, of course, connected, above all, with
the increase in precision in the natural scientific work and thought in the
domain of the biosphere, where living matter plays a major role.

The sharply distinct manifestation of living from inert in the aspect of time
in the biosphere, with all of its significance, is a special expression of a
far greater phenomenon, reflected in the biosphere at every step.


\Section % 8
% The plasticity of living matter.  The evolution of the biosphere.
The living matter of the biosphere sharply differs from its inert matter in two
main processes, which have an immense geological significance, and give the
biosphere a completely different shape, which does not exist in any other
envelope of the planet.  These two processes are manifested only against the
background of geological time.  They never cease, and never go backwards.

First, \emph{the power of the expression of living matter in the biosphere
grows} in the course of geological time, the significance of living matter in
the biosphere, and its influence on the inert matter of the biosphere
increases.  This process is little taken into account to this day.  I will have
to deal with it all the time further on.

Another process, known to all, and having imprinted a deepest impression on all
scientific thought of the $19^\mathrm{th}$ and $20^\mathrm{th}$ centuries since
the middle of the $19^\mathrm{th}$ century, has attracted far more attention,
and has been studied much more.  This is the process of \emph{the evolution of
species} in the course of geological time---the sharp change in the living
natural bodies themselves.

We observe a sharp change in the natural bodies themselves in the course of
geological time only in living matter.  Some organisms turn into others, die
out, as we say, or change fundamentally.

Living matter is \emph{plastic}, changes, adapts to changes in its environment,
but also, possibly, has its own evolutionary process, manifested in changes in
the course of geological time, independent of the changes in the environment.
This is, perhaps, indicated by the incessant growth, with intermissions, of the
central nervous system of animals in its significance in the biosphere, and in
the depth of the reflection of living matter in its surroundings,\footnote{
	That the evolution of nervous tissue is incessantly ongoing in the
	course of geological time has been indicated more than once, but, as
	far as I know, it has not been completely analyzed scientifically and
	philosophically.  As the question here is not about a hypothesis, and
	not about a theory, the fact of its evolution cannot be denied---there
	can only be objections to its explanation.  The recognition of Redi's
	principle limits the number of explanations.
} in the former's penetration into the latter, in the course of geological
time.

The plasticity of living matter is, obviously, a very complex phenomenon, as
there are organisms which do not change in their morphological and
physiological structure noticeably for us for hundreds of millions of years, up
to five hundred million and more, over myriads of generations.  These are the
so-called \emph{persistents}\footnoteRus{
	персистенты.  These seem to be popularly known as \emph{living fossils}
	today. [---Pav]
}\fncomma\footnoteEd{
	persistents\dots\ See \cite[269]{vernadsky1965himicheskoe}. [---Ed.]
}---a phenomenon in biology which has been, unfortunately, extremely little
studied.  Nevertheless, we observe in them, as a phenomenon common to living
matter, a \emph{plastic evolutionary} process, for which there is not even a
symptom in inert natural bodies.  In these latter we see the same minerals, the
same formation processes, the same rocks, and so forth \emph{now,} which were
there \emph{two billion years, and more, ago.}

The evolutionary process of living matter encompasses the whole biosphere
incessantly during all geological time and, in different ways, less strongly,
still affects its inert natural bodies.  With this we already can, and must
talk about \emph{the evolutionary process of the biosphere itself,} occurring
in the inert mass\footnoteRus{инертной массе} of its inert and living natural
bodies, changing visibly in the course of geological time.

The reflection of living matter in its surrounding environment changes sharply
due to the evolution of species, ongoing constantly, and never ceasing.  Thanks
to this process, evolution---change---is transferred to the natural bioinert
and biogenic bodies, which play a major role in the biosphere---to soils, to
surface and underground waters (in seas, lakes, rivers, etc.), to coal,
bitumen, limestones, organogenic ores, and so forth.  The soils and rivers of
the Devonian, for example, are different from the soils [and
rivers]\footnoteTransl{Interpolated from the implied meaning. [---Pav]} of the
Tertiery, and of our period.  This is an area of new phenomena, hardly taken
into account by scientific thought.  \emph{The evolution of species turns into
an evolution of the biosphere.}


\Section % 9

. . .

\Section % 10

. . .

\Section % 11

. . .

\Section % 12

. . .

\Section % 13

. . .
