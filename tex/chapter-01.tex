\part{Scientific Thought and Scientific Work as a Geological Force in the
Biosphere}

\Chapter{Man and mankind in the biosphere as a lawful part of its living
matter, part of its organization.  Physical-chemical and geometric
heterogeneity of the biosphere: the fundamental organizational
distinction---material-energetic and temporal---of its living matter from its
inert matter.  Evolution of the species, and evolution of the biosphere.  The
manifestation of a new geological force in the biosphere---the scientific
thought of social mankind.  Its manifestation is related to the ice age, in
which we live, with one of the geological phenomena repeating in the history of
the planet, whose cause exceeds the bounds of the Earth's crust.}


\Section
Man, as well as everything living, is not a self-sufficient, independent of the
environment natural object.  However, even natural scientists in our time,
counterposing human beings and living organisms in general to the environment
of their life, very often do not take this into account.  But the
inseparability between living organism and its environment cannot presently
raise any doubt among contemporary naturalists.  The biogeochemist proceeds
from it, and strives to understand, express, and establish this functional
dependence precisely, and as deeply as possible.  Philosophers and contemporary
philosophy predominantly do not take into account this functional dependence of
man, as a natural object, and mankind, as a natural phenomenon, on the
environment of their life and thought.

Philosophy cannot sufficiently take this into account, as it proceeds from the
laws of the mind, which is, in one way or another, a final and self-sufficient
criterium for it (even in those cases, like in religious and mystical
philosophies, in which the reach of the mind is, in fact, limited).

The contemporary scientist, proceeding from the recognition of the reality of
one's surroundings, of the world subject to one's investigation---nature, the
cosmos, or world reality,\footnote{
	I will talk about the reality of the cosmos, instead of that of nature,
	here and further.  The concept nature, if we take it in a historical
	aspect, is a complex concept.  It very often encompasses only the
	biosphere, and it is more convenient to use it with just this meaning,
	or even not to use it at all (\autoref{sec:6}).  This would correspond
	to the vast majority of the uses of this concept historically in
	natural science and in literature.  The concept `cosmos' can be,
	perhaps, more conveniently applied to only the part of reality
	encompassed by science, a philosophically pluralistic conception of
	reality is possible at that, where there would be no single criterium
	for the cosmos.
}---cannot adopt this point of view as a basis for scientific work.

Because one presently knows exactly that man is \emph{not} located on a
structureless surface of the Earth, is \emph{not} located in direct contact
with cosmic space in a structureless nature, which is not lawfully connected
with him.  True, even the deeply penetrating contemporary naturalist often, out
of routine and under the influence of philosophy, forgets this, and does not
take it into account in his thought, and does not identify this.

Man and mankind are most closely connected, above all, with the living matter
inhabiting our planet, from which they cannot, in reality, be isolated by any
physical process.  That is possible only in thought.


\Section % 2

. . .

\Section % 3

. . .

\Section % 4

. . .

\Section % 5

. . .

\Section \label{sec:6} % 6

. . .

\Section % 7

. . .

\Section % 8

. . .

\Section % 9

. . .

\Section % 10

. . .

\Section % 11

. . .

\Section % 12

. . .

\Section % 13

. . .
