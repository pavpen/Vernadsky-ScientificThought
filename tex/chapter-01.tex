\part{Scientific Thought and Scientific Work as a Geological Force in the
Biosphere}

\Chapter{Man and mankind in the biosphere as a lawful part of its living
matter, part of its organization.  Physical-chemical and geometric
heterogeneity of the biosphere: the fundamental organizational
distinction---material-energetic and temporal---of its living matter from its
inert matter.  Evolution of the species, and evolution of the biosphere.  The
manifestation of a new geological force in the biosphere---the scientific
thought of social mankind.  Its manifestation is related to the ice age, in
which we live, to one of the geological phenomena repeating in the history of
the planet, whose cause exceeds the bounds of the Earth's crust.}


\Section % 1
% Inseparability of scientific thought from the living environment.
% Inseparability of the concepts of nature & cosmos from the biosphere &
% science.
Man, as well as everything living, is not a self-sufficient, independent of the
environment natural object.  However, even natural scientists in our time,
counterposing human beings and living organisms in general to the environment
of their life, very often do not take this into account.  But the
inseparability between living organism and its environment cannot presently
raise any doubt among contemporary naturalists.  The biogeochemist proceeds
from it, and strives to understand, express, and establish this functional
dependence precisely, and as deeply as possible.  Philosophers and contemporary
philosophy predominantly do not take into account this functional dependence of
man, as a natural object, and mankind, as a natural phenomenon, on the
environment of their life and thought.

Philosophy cannot sufficiently take this into account, as it proceeds from the
laws of the mind, which is, in one way or another, a final and self-sufficient
criterion for it (even in those cases, like religious and mystical
philosophies, in which the reach of the mind is, in fact, limited).

The contemporary scientist, proceeding from the recognition of the reality of
one's surroundings, of the world subject to one's investigation---nature, the
cosmos, or world reality,\footnote{
	I will talk about the reality of the cosmos, instead of that of nature,
	here and further.  The concept nature, if we take it in a historical
	aspect, is a complex concept.  It very often encompasses only the
	biosphere, and it is more convenient to use it with just this meaning,
	or even not to use it at all (\autoref{sec:6}).  This would correspond
	to the vast majority of the uses of this concept historically in
	natural science and in literature.  The concept `cosmos' can be,
	perhaps, more conveniently applied to only the part of reality
	encompassed by science, a philosophically pluralistic conception of
	reality is possible at that, where there would be no single criterion
	for the cosmos.
}---cannot adopt this point of view as a basis for scientific work.

[Thus,]\footnoteTransl{Interpolated for meaning in English.} because one
presently knows with scientific precision that man is \emph{not} located on a
structureless surface of the Earth, is \emph{not} located in direct contact
with cosmic space in a structureless nature, which is not lawfully connected
with him.  True, even the deeply penetrating contemporary naturalist often, out
of routine and under the influence of philosophy, forgets this, and does not
take it into account in his thought, and does not identify this.

Man and mankind are most closely connected, above all, with the living matter
inhabiting our planet, from which they cannot, in reality, be isolated by any
physical process.  That is possible only in thought.


\Section % 2
% We shall study the scientifically unambiguous 'living matter', and 'living
% natural body', rather than 'life'.
The concept of life and the living is clear to us in everyday life, and cannot
raise scientifically serious doubts in the actual manifestations of it, and in
natural objects corresponding to it---in natural bodies.  It was only in the
$20^{\mathrm{th}}$ century, [with the discovery of] filter-passing viruses,
that there appeared facts in science compelling us for the first time to ask
seriously---not philosophically, but scientifically---the question: Are we
dealing with a living natural body, or with a non-living natural body---an
inert one?

With viruses the doubt is cast by scientific observations, rather than
philosophical notions.  In this consists the great scientific significance of
their study.  That is presently on a right and firm path.  The doubt will be
resolved, and nothing, except a more precise notion of \emph{living organism},
would give, with this approach couldn't fail to give
\dots\footnoteEd{Incomplete sentence in the original.}

Along with this, however, we encounter another kind of doubt in \emph{science},
arising from philosophical and religious searches.  For example, phenomena
concerning the material-energetic environment of manifestations, which are
philosophically \emph{common} to both living and inert natural bodies, are
scientifically studied in the works of the Bose Institute in
Calcutta.\footnoteRus{Института Бозе в Калькутте}\footnoteEd{
	The Bose
	Institute\footnoteTransl{\url{http://www.boseinst.ernet.in/index.html}}
	in Calcutta was founded by the Indian scientist Acharya Jagadish
	Chandra Bose\footnoteRus{Бозе Джегдиш Чандра} (1858--1937) in 1917.
	The institute studied the problems of physics, biophysics, inorganic
	and organic chemistry, biochemistry, the physiology of plants,
	selection, microbiology, etc. [---Ed.]
} They are not characteristic of, but are weakly expressed in inert natural
bodies, are strongly manifested in living ones, but are common to both.

This area, if it exists in the form in which Bose tried to establish it, of
phenomena common to inert and living natural bodies, introduces nothing new in
the sharp distinction between them.  The distinction must manifest itself in
this area, as well, if only its existence would be proven.

We must approach phenomena here, as well, not in the aspect in which Bose
approached them, not as phenomena of \emph{life}, but as phenomena of living
natural bodies, of \emph{living matter}.

To avoid any misunderstanding, I shall avoid the concepts `life', and `living'
in all further exposition, since, if we proceeded from those
[phenomena],\footnoteTransl{inserted by transl. [---Pav]} we would inevitably
go beyond the limits of the phenomena of life studied in science, into a
foreign area or science---the area of philosophy, or, as is taking place in the
Bose Institute, into a new area of new material-energetic manifestations common
to all natural bodies of the biosphere, one lying outside the bounds of the
fundamental question of living organism, and living matter, which we are
presently interested in.

I shall, therefore, avoid the terms and concepts `life', and `living', and
limit the area which is subject to our investigation to the concepts
\emph{`living natural body'}, and \emph{`living matter'}.  Each living organism
\emph{in the biosphere}---natural object---is a living natural body.  \emph{The
living matter of the biosphere is the complex of living organisms in it.}

`Living matter', so defined, is a concept, completely precise, and fully
encompassing the objects studied by biology and biogeochemistry.  It is simple,
clear, and cannot raise any doubt.  We study only the living organism and its
complexes in science.  They are scientifically identical with the concept of
life.


\Section % 3
% Living matter is in a state of organization, which is determining for the
% biosphere.  Living matter's cosmic significance, and the constantly
% increasing extent of the biosphere.
Man, like every living natural (or naturally-occurring)\footnoteRus{природное
(или естественное)} body, is inseparably connected with a certain geological
envelope of our planet---\emph{the biosphere}, clearly distinct from the rest
of its envelopes, with a structure which is determined by its specific
\emph{state of organization},\footnoteTransl{организованностью, \ie\ 
organizedness, or being (constantly) organized. [---Pav]} and occupying a
lawfully expressible place in it as a distinct part of the whole.

Living matter, just like the biosphere, possesses its peculiar state of
organization, and can be viewed as a lawfully expressible \emph{function of the
biosphere.}

\emph{A state of organization} is not a mechanism.  It sharply differs from a
mechanism in that it is constantly in a state of
becoming,\footnoteTransl{становлении, \eg\ formation. [---Pav]} of motion of
all of its smallest material and energetic particles.  We can express this
state of organization in the course of time---in a generalization of mechanics,
and in a simplified model---as being such that none of its points (material or
energetic) lawfully returns to, \altStylePhr{or}\footnoteEd{
	Phrases like this have been inserted for more familiar reading to
	today's audience.  Otherwise, Vernadsky's style often includes
	restating a phrase with a slight change in meaning to communicate an
	intended range of ambiguity: a style which can be better understood as
	coming from a living speaker, rather than from a page of reference
	`facts'.  [---Pav]
} ends up in a place, \altStylePhr{or} a point of the biosphere, which it
occupied at any earlier moment.  It can return to one of them only on the order
of mathematical accident, of very small probability.

The Earth's envelope, the biosphere, embracing the whole globe, has clearly
distinct dimensions, \altStylePhr{and} is determined to a large degree by the
existence of living matter in it---\emph{populating}\footnoteRus{заселена} it.
There is a constant material and energetic exchange, materially expressed in
the motion of atoms brought about by living matter, between its inert
non-living part, its inert natural bodies, and the living matter inhabiting it.
This exchange in the course of time is expressed as lawfully changing,
constantly tending toward the stability of an
\emph{equilibrium}.\footnoteRus{равновесием}  It permeates the whole biosphere,
and this \emph{biogenic flow of atoms}\footnoteRus{биогенный ток атомов}
creates the biosphere to a large extent.  The biosphere is thus connected
inseparably and inherently with the living matter populating it throughout the
whole duration of geological time.

The planetary, cosmic significance of living matter is distinctly expressed in
this biogenic flow of atoms, and in the energy connected with it.  That, since
the biosphere is that single envelope of the Earth, in which cosmic energy,
cosmic radiation, mainly radiation from the Sun, which maintains the dynamic
equilibrium, the state of organization: $\mathrm{biosphere} \leftrightarrow
\mathrm{living matter}$, constantly penetrates.

The biosphere stretches from the surface of the geoid up to the boundary of the
stratosphere, penetrating into it; it, however, would be unlikely to be able to
reach the ionosphere---the Earth's electromagnetic vacuum, which is just now
entering the scientific consciousness.  Living matter reaches below the surface
of the geoid into the stratisphere, and into the top regions of the
metamorphic, and of the granitic envelope.  It rises up to
20--$25\,\mathrm{km}$ above the surface of the geoid, and extends down to
4--$5\,\mathrm{km}$ below that level on average.  These boundaries change in
the course of time, and, there are places of, it is true, small extent, where
they are far beyond these.  Apparently, living matter must reach deeper than
$11\,\mathrm{km}$ at places in the depths of the ocean, and its presence has
been established deeper than $6\,\mathrm{km}$.\footnoteEd{
	Ocean floor organisms have indeed been observed at all depths of the
	world ocean, including at greater than $11\,\mathrm{km}$.  (See
	\foreignlanguage{russian}{\cites{belyaev1966donnaya}{belayev1989glubokovodnye}}.)
	[---Ed.]
}  We are just now living through the penetration of mankind, always
inseparable from other organisms---insects, plants, microbes,---into the
stratosphere, and by this means living matter has already exceeded
$40\,\mathrm{km}$ above the surface of the geoid, and is quiclky rising.

Evidently, a process of incessant expansion of the boundaries of the biosphere:
its population by living matter, is observable in the course of geological
time.


\Section % 4
\emph{The state of organization of the biosphere}---the state of organization
of living matter---must be viewed as an equilibrium, which is changeable,
always oscillating around a precisely expressible mean, not in historical, but
in geological time.  The shifts, or oscillations of this mean are constantly
manifested not in historical, but rather in geological time.  In the course of
geological time, in the cyclical processes which are characteristic of the
biogeochemical state of organization, no point (for example, atom or chemical
element) ever returns to a position identical with a previous one for eons.

This characteristic of the biosphere was expressed very prominently and vividly
by Leibniz [1646--1716] in one of his philosophical reflections, it seems to
me, in the \rtitle{Theodicy}.\footnoteTransl{
	See \refsmartcite{leibnitz1985theodicy, leibnitz1900oeuvres-v2}.
}  At the end of the $18^\mathrm{th}$ c., Leibniz recounts, he was among a
large company from the high society in a large garden, and speaking of the
infinite variety of nature, and of the infinite perfectability of the mind's
precision,\footnoteRus{бесконечной четкости ума} indicated that two leaves of
any tree or plant are never completely identical.  All efforts of the large
company to find such leaves were, of course, in vain.  Leibniz was reflecting
here not as an observer of nature, discovering this phenomenon for the first
time, but as en erudite, taking it from his readings.  It is possible to trace
that precisely this example of the leaves appeared in philosophical folklore
centuries earlier.\footnote{
	See, for example, \cite[кн.~2,
	с.~54]{carus1913prirode}\nocite{carus1936prirode, carus1851nature}.
	(\Eg, \cite[book~2]{carus1851nature}. ---Pav)
}

This is manifested for us in everyday life in [individual]
\emph{identity},\footnoteRus{личности} in the absence of two identical
individuals, indistinguishable from one another.  It is manifested in biology
in the fact that every mean \emph{individuum}\footnoteRus{индивидуум} of living
matter is \emph{chemically distinct} in its chemical compounds, as, obviously,
also in its chemical elements having \emph{their own} specific compounds.


\Section % 5
% The heterogeneity of the biosphere.  The connection between living and inert
% matter.
Especially characteristic in the structure of \emph{the biosphere is its
physical-chemical, and geometric \emph{(\autoref{sec:47})} heterogeneity}.  It
consists of living and inert matter, which are sharply separate in their
genesis and structure throughout all of geological time.  Living organisms,
\ie\ all living matter, are born from living matter, \altStylePhr{and} form
generations in the course of time, which never arise directly, outside of such
a living organism, from any possible inert matter on the planet.  There is,
however, an inherent, unceasing connection\footnoteRus{непрерывная, никогда не
прекращающаяся связь} between inert and living matter, which can be expressed
as an incessant biogenic flow of atoms from the living matter into the inert
matter of the biosphere, and vice versa.  This biogenic flow of atoms is
originated by living matter.  It is expressed in its always unceasing
breathing, feeding, reproducing, etc.

This heterogeneity in the biosphere, unceasing throughout all geological time,
is the main dominant factor, strongly distinguishing it from all other
envelopes of the globe.

It goes deeper than the phenomena usually studied in natural science---to the
properties of space-time, which scientific thought has approached only in our
time, in the $20^\mathrm{th}$ c.

Living matter encompasses the whole biosphere, creates it, and changes it, but
amounts to a small part of it by mass, and by volume.  Inert, non-living matter
is strongly dominant; greatly diluted gases dominate by volume, hard rocks,
and, to a lesser degree, the liquid salt water of the world ocean---by mass.
Living matter, even in the greatest concentrations, in exceptional cases with
insignificant masses, amounts to tens of percent of the biosphere's matter, and
amounts to hardly one--two hundredths of a percent by mass on average.
Geologically, however, it is the greatest force in the biosphere, and
determines, as we can see, all processes occurring in it, and accumulates vast
free energy, creating the main geologically manifesting force in the biosphere,
whose power still cannot be quantitatively determined, but, possibly, exceeds
all other geological manifestations in the biosphere.

In connection with this, it is convenient to introduce a few basic concepts
which we will be dealing with in all of the following exposition.


\Section \label{sec:6} % 6
% The concepts of natural body (natural object) and natural phenomenon; living,
% inert, and bioinert bodies and phenomena; heterogeneity of the biosphere's
% structure.
Such are the concepts connected with the concepts of natural body (natural
object),\footnoteRus{природного тела (природного объекта)} and natural
phenomenon.\footnoteRus{природного явления}  They have often been referred to
as naturally-occurring bodies or phenomena.\footnoteRus{естественные тела или
явления}

Living matter is a natural body or phenomenon in the biosphere.  The concepts
\emph{natural body or natural phenomenon}, little logically studied, are main
concepts of natural science.  There is no need to delve into their logical
analysis for our purposes.  These are bodies or phenomena, formed by natural
processes,---\emph{natural objects}.

Living organisms, living matter, are not the only natural bodies of the
biosphere, but rather the main mass of the biosphere's matter is in the form of
non-living bodies or phenomena, which I will be referring to as
\emph{inert}.\footnoteRus{косными}  Such are, for example, gases, the
atmosphere, rocks, a chemical element, an atom, quartz, serpentine, etc.

In addition to living and inert natural bodies, its lawful structures,
heterogeneous natural bodies, such as soils, silts, surface waters, the very
biosphere, etc., which consist of living and inert natural bodies existing at
the same time, forming complex, lawful inert-living
structures,\footnoteRus{косно-живые структуры} play a great role in the
biosphere.  I will be calling these complex natural bodies \emph{bioinert}
natural bodies.\footnoteRus{биокосными природными телами}

The difference between living and inert natural bodies is so great, as we shall
see further on, that the transformation of one into the other is never and
nowhere observed in terrestrial processes; we encounter them nowhere and never
in scientific work.  As we shall see, such a process is deeper than the
physical-chemical phenomena known to us.

The related \emph{heterogeneity of the biosphere's structure}, the sharp
distinction between its matter and its energy in the form of living, and inert
natural bodies, is its main manifestation.


\Section % 7
% The heterogeneity of time (historical and geological) in the biosphere.
One of the manifestations of this heterogeneity of the biosphere consist in the
fact that processes occur completely differently in living matter than in inert
matter, if they are viewed in the aspect of time.  They take place on the scale
of \emph{historical time}\footnoteRus{\emph{исторического времени}} in living
matter, on the scale of \emph{geological
time},\footnoteRus{\emph{геологического времени}} whose `second' is under
decamyriads, \ie\ a hundred thousand years of historical time,\footnote{
	On decamyriads see \cite{vernadsky1935nekotoryh}, [as well as
	\cite{vernadsky1954sochineniya-nekotorye}].
} in inert matter.  This difference is expressed even more sharply outside the
boundaries of the biosphere, and we observe in the litosphere, for the
predominant part of its matter, a state of organization in which the majority
of atoms, as radioactive studies show, are immobile, do not mix, noticably for
us, in the course of tens of thousands of decamyriads---a span of time now
accessible to our measurement.

Not long ago the view that geologists cannot study the manifestation of
geologically long changes, occurring in the age of mankind's existence,
dominated.  In the times of my youth we learned and thought that the change in
climate, orography, the emergence of new species of organisms are not, as a
general rule, detected in geological studies, are not \emph{current
phenomena}\footnoteRus{текущим явлением} for the geologist.  Now this
conceptual circumstance of the naturalist has sharply changed, and we can see
ever more, and more emphatically the geological forces around us.  This
coincided, hardly by accident, with the penetration of the conviction of the
geological significance of Homo sapiens in the scientific consciousness, with
the detection of a new state of the biosphere---the noosphere---and is one of
the forms of its manifestation.  This is, of course, connected, above all, with
the increase in precision in the natural scientific work and thought in the
domain of the biosphere, where living matter plays a major role.

The sharply distinct manifestation of living from inert in the aspect of time
in the biosphere, with all of its significance, is a special expression of a
far greater phenomenon, reflected in the biosphere at every step.


\Section \label{sec:8}% 8
% The plasticity of living matter.  The evolution of the biosphere.
The living matter of the biosphere sharply differs from its inert matter in two
main processes, which have an immense geological significance, and give the
biosphere a completely different shape, which does not exist in any other
envelope of the planet.  These two processes are manifested only against the
background of geological time.  They never cease, and never go backwards.

First, \emph{the power of the expression of living matter in the biosphere
grows} in the course of geological time, the significance of living matter in
the biosphere, and its influence on the inert matter of the biosphere
increases.  This process is little taken into account to this day.  I will have
to deal with it all the time further on.

Another process, known to all, and having imprinted a deepest impression on all
scientific thought of the $19^\mathrm{th}$ and $20^\mathrm{th}$ centuries since
the middle of the $19^\mathrm{th}$ century, has attracted far more attention,
and has been studied much more.  This is the process of \emph{the evolution of
species} in the course of geological time---the sharp change in the living
natural bodies themselves.

We observe a sharp change in the natural bodies themselves in the course of
geological time only in living matter.  Some organisms turn into others, die
out, as we say, or change fundamentally.

Living matter is \emph{plastic}, changes, adapts to changes in its environment,
but also, possibly, has its own evolutionary process, manifested in changes in
the course of geological time, independent of the changes in the environment.
This is, perhaps, indicated by the incessant growth, with intermissions, of the
central nervous system of animals in its significance in the biosphere, and in
the depth of the reflection of living matter in its surroundings,\footnote{
	That the evolution of nervous tissue is incessantly ongoing in the
	course of geological time has been indicated more than once, but, as
	far as I know, it has not been completely analyzed scientifically and
	philosophically.  As the question here is not about a hypothesis, and
	not about a theory, the fact of its evolution cannot be denied---there
	can only be objections to its explanation.  The recognition of Redi's
	principle limits the number of explanations.
} in the former's penetration into the latter, in the course of geological
time.

The plasticity of living matter is, obviously, a very complex phenomenon, as
there are organisms which do not change in their morphological and
physiological structure noticeably for us for hundreds of millions of years, up
to five hundred million and more, over myriads of generations.  These are the
so-called \emph{persistents}\footnoteRus{
	персистенты.  These seem to be popularly known as \emph{living fossils}
	today. [---Pav]
}\fncomma\footnoteEd{
	persistents\dots\ See \cite[269]{vernadsky1965himicheskoe}. [---Ed.]
}---a phenomenon in biology which has been, unfortunately, extremely little
studied.  Nevertheless, we observe in them, as a phenomenon common to living
matter, a \emph{plastic evolutionary} process, for which there is not even a
symptom in inert natural bodies.  In these latter we see the same minerals, the
same formation processes, the same rocks, and so forth \emph{now,} which were
there \emph{two billion years, and more, ago.}

The evolutionary process of living matter encompasses the whole biosphere
incessantly during all geological time and, in different ways, less strongly,
still affects its inert natural bodies.  With this we already can, and must
talk about \emph{the evolutionary process of the biosphere itself,} occurring
in the inert mass\footnoteRus{инертной массе} of its inert and living natural
bodies, changing visibly in the course of geological time.

The reflection of living matter in its surrounding environment changes sharply
due to the evolution of species, ongoing constantly, and never ceasing.  Thanks
to this process, evolution---change---is transferred to the natural bioinert
and biogenic bodies, which play a major role in the biosphere---to soils, to
surface and underground waters (in seas, lakes, rivers, etc.), to coal,
bitumen, limestones, organogenic ores, and so forth.  The soils and rivers of
the Devonian, for example, are different from the soils [and
rivers]\footnoteTransl{Interpolated from the implied meaning. [---Pav]} of the
Tertiery, and of our period.  This is an area of new phenomena, hardly taken
into account by scientific thought.  \emph{The evolution of species turns into
an evolution of the biosphere.}


\Section \label{sec:9}% 9
% Scientific thought is not an arbitrary process, and is not governed by the
% arbitrary will of man.
The evolutionary process has acquired, in addition, a special geological
significance thanks to the fact that it has created a new geological
force---the scientific thought of social mankind.\footnoteRus{научную мысль
социального человечества}

We are now fully living through its prominent entry in the geological history
of the planet.  The intensive growth of the influence of a single species of
living matter\footnoteRus{одного видового живого вещества}---civilized
mankind---on the biosphere is observable during the last millenia.  The
biosphere is transitioning into a new state---\emph{into the noosphere}---under
the influence of scientific thought and human labor.

Mankind is encopassing the whole planet, is distinguishing itself, is diverging
from other living organisms as a new, unprecedented geological force by a
lawful motion, stretching one--two million years, with an ever-increasing in
its manifestation rate.  An ever-growing set of inert natural bodies, \emph{new
for the biosphere,} and new, great natural phenomena are being created by this
means in the biosphere at a speed comparable to that of reproduction,
expressible by a geometric progression in the course of time.

The biosphere is changing drastically in front of our eyes.  And there can
harldy be any doubt that its transformation, manifested in this way, by
scientific thought through organized human labor is not an arbitrary
phenomenon,\footnoteRus{случайное явление, \ie\ also chance phenomenon, or
accidental phenomenon} depending on the will of man, but is a tempestuous
\emph{natural process,}\footnoteRus{стихийный \emph{природный процесс}} whose
roots lie deep, and had been prepared by an evolutionary process whose duration
is calculated in the hundreds of millions of years.

Man must understand, as only a scientific, but not a philosophical or a
religious conception of the world can encompass this, that \emph{he is not an
arbitrary, independent of its surroundings}---biosphere and
noosphere---freely-acting natural phenomenon.  He comprises an unavoidable
manifestation of a great natural process, lawfully stretching throughout the
course of, at least, two billion years.

In the present times, under the influence of the surrounding horrors of life,
along with an unprecedented flowering of scientific thought, it has become
necessary to hear of the approach of barbarity, of the breakdown of
civilization, of the self-annihilation of mankind.  These attitudes, and these
reasonings appear to me to be consequences of insufficiently deep penetration
into our surroundings.  Scientific thought not having entered everyday life
yet, we are still living under the strong influence of philosophical and
religious habits, not corresponding to the reality of contemporary knowledge,
which we still have not grown out of.

Scientific knowledge, being manifested as a geological force creating the
noosphere, cannot lead to results, which contradict that geological process,
whose creation it is.  This is not an arbitrary phenomenon---its roots are
extremely deep.


\Section % 10
% The related process of cephalization, and the intensification of evolution.
% The critical periods of geological history, which reflect the periods of
% evolutionary intensificaton.
This process is connected with the creation of the human brain.  It was
detected in the history of science in the form of an empirical generalization
by the profound American naturalist, great geologist, zoologist,
paleontologist, and mineralogist J.~D.\ Dana [1813--1895] in New Haven.  He
published his conclusion almost 80 years ago.  Strangely, this generalization
has not entered daily life to this day, has been almost forgotten, and has not
undergone the necessary development to this day.  I will return to this later.
Here I will note that Dana presented his empirical generalization in the
language of philosophy and theology, and it, it seems, was connected to
conceptions, scientifically unacceptable today.

Speaking in contemporary scientific language, Dana noted that a more and more
advanced---central nervous system---\emph{brain} than that which existed
earlier on our planet is manifested [in] some parts of its inhabitants in the
course of geological time.\nocite{dana1866classification}  This process, called
\emph{encephalization} by him, never goes backward, [even though] it ceases
many times, sometimes for many millions of years.  The process is, therefore,
expressed by a polar temporal vector, whose direction does not change.  We
shall see that the geometrical state of space\footnoteRus{геометрическое
состояние пространства} occupied by living matter is also characterized by
polar vectors, that there is no place for straight lines in it.

The evolution of the biosphere is connected with \emph{the intensification of
the evolutionary process} of living matter.

We now know that critical periods in the history of the terrestrial crust are
emerging, in which geological activity, in the most diverse of its
manifestations, is increasing in its rate.  This increase is, of course,
unnoticeable in historical time, and can be noted scientifically only on the
scale of geological time.

These periods can be considered \emph{critical} in the history of the planet,
and everything is indicating that they are occasioned by deep, from the
standpoint of the Earth's crust, processes, apparently exceeding its
boundaries.  A simultaneous increase in volcanic, orogenic, glacial phenomena,
marine transgressions, and other geological processes simultaneously
encompassing a great part of the biosphere throughout its whole extent has been
observed.\footnoteEd{
	More precise stratigraphic studies, done in various parts of our planet
	during the 45-year post-war period require us to modify our conception
	of `critical ages' in the history of the Earth somewhat.  Orogenic
	phenomena, as well as marine transgressions, turned out to have
	occurred at greatly different times on different continents, and even
	in different parts of the same massive continent.  [See
	\cite{yanshin1966tektonika, yanshin1973nazyvaemyh}.]  However, there
	undoubtedly were outbreaks of volcanic activity on the territories of
	contemporary continents in the history of the Earth.  Judging from the
	estimates produced by A.~B.\ Ronov\footnoteRus{А.~Б.\ Роновым} of the
	mass of volcanic products, they took place throughout the last 600
	million years in the middle Devonian, at the end of the
	Carboniferous--beginning of the Permian, at the end of the Triassic,
	and to a less significant degree in the middle Cretaceous, and in the
	Neogene.  Each such outbreak of volcanism led to a planetary change in
	the composition of the atmosphere---to an increase in its $CO_2$
	content, and to a decrease of the oxygen content, which brought, on the
	one hand, a decrease of temperature, leading to the formation of polar
	ice caps, and, on the other,---an intense development of vegetation,
	and the return of oxygen to the atmosphere, as a result of the
	processes of photosynthesis.  [See \cite{budyko1974klimat}] Apparently,
	``most important, and great changes in the structure of living matter''
	were created in these periods, \ie\ they were `critical' in the sense
	in which V.~I.\ Vernadsky used this word.  [---Ed.]
}  The evolutionary process coincides in its intensification, in its greatest
changes with these periods.  Most important and great changes in the structure
of living matter, which are a clear expression of the depth of the geological
significance of this plastic reflection of living matter in the resulting
changes of the planet, were created in these periods.

There is no theory, precise scientific explanation of this main phenomenon in
the history of the planet.  It emerged empirically, and
unconsciously---penetrated science unnoticed, and its history has remained
unwritten.  A major role in it played American geologists, specifically, J.~D.\ 
Dana.  It has pervaded the scientific thought of our century.

It is, however, possible, and necessary to approach it with measure and number.
The geological length of its duration can be measured, and, in this way, the
change in the rate of geological processes can be characterized numerically.
This is one of the immediate tasks of radiogeology.


\Section % 11
% The new evolutionary state of the biosphere, associated with the scientific
% thought of social mankind.
While this remains uncompleted, we must note, and take into account that the
process of evolution of the biosphere, its transition into \emph{the
noosphere}, clearly manifests an acceleration in the rate of geological
processes.  The changes, which are presently manifesting themselves in the
biosphere through the course of [the last] several \emph{thousand years} in
connection with the growth of scientific thought and the social activity of
mankind, have never existed in the history of the biosphere before.

Such, at the very least, are the conceptions which we can now derive from the
study of the course of evolution of organisms during geological time.
Decamyriads\footnoteRus{декамириада} are much less than a historical time's
second for geological time.  Consequently, a thousand years on a geological
scale would be more than 300 million years of geological time.  This does not
contradict [the existence of the periods of]\footnoteTransl{
	Interpolated from implied meaning.
} the great changes of the biosphere which took place, for example, in the
Cambrian, when calcareous skeletal parts emerged in microscopic marine
organisms, or [in] the Paleocene, when the fauna of mammals grew.\footnoteEd{
	Numerous findings of small mammals are now known from the deposits of
	different horizons of the Upper, and the top strata of the Lower
	Cretaceous, and the most ancient remains of primitive mammals have been
	observed already in Triassic deposits.  However, the intensive
	evolutionary development of this class of vertebrates began after the
	dying out of the dinosaurs in the Paleocene, by which the boundary
	between the Cretaceous and the Paleogene in the history of the Earth is
	determined to a large degree. [---Ed.]
}  We must not fail to keep in mind that the time we are living through
corresponds, geologically, to such a critical period, since the ice age has
% FIXME: Does Vernadsky mean an ice age cycle, rather than just an ice age by
% `ледниковый период'?
still not ended---the rate of change is, nevertheless, so slow that man could
not notice it.

Man and mankind, his kingdom in the biosphere lie completely in this period,
and do not exceed its boundaries.

A picture of the evolution of the biosphere since the Algonkian, and, more
sharply, since the Cambrian, over 500--800 million years can be given.  The
biosphere transitioned into a new evolutionary state\footnoteRus{эволюционное
состояние} more than once.  New geological manifestations, which had never
existed before, emerged.  This occurred, for example, in the Cambrian, when
large organisms with calcium skeletons came into existence, and in the Tertiary
(or, possibly, at the end of the Cretaceous), 15--80 million years ago, when
our forests and steppes were coming into existence, and the life of large
mammals developed.  We have also been living through this presently, for the
past 10--20 thousand years, when man, having developed scientific thought in a
social environment, has been creating a new geological force in the biosphere,
unprecedented in it.  The biosphere has transitioned into, or, more precisely,
is transitioning into \emph{a new evolutionary state---into the noosphere}---is
being transformed by the scientific thought of social mankind.


\Section % 12
% The irreversibility of evolution, an expression of the Curie-Pesteur
% discoveries principles of handedness and dissymmetry, and, actually, of
% Vernadsky's polar and enantiomorphic state of space.
The irreversibility of the evolutionary process is a manifestation of the
characteristic distinction of living matter in the geological history of the
planet from its inert naturally-occurring bodies and processes.  It can be seen
that this irreversibility is connected to the special properites of the space
occupied by the bodies of living organisms, to its special geometric structure,
as P.~Curie said, with its special \emph{state of
space}.\footnoteRus{\emph{состоянием пространства}}  L.~Pasteur first
understood the fundamental significance of this phenomenon, which he
inadequately called dissymmetry, in 1862.\footnote{
	The principle was formulated by P.~Curie (1859--1906), but was
	understood and expressed completely clearly and intuitively by
	L.~Pasteur (1822--1895).  I have delimited it here as a special
	principle (\cite{pasteur1922oeuvres, curie1908oeuvres}).
}  He studied this phenomenon in another aspect, in the inequality between
left and right phenomena in the organism, in the existence of left-handedness
and right-handedness\footnoteRus{правизны и левизны} for it.\footnote{
	It is striking that the phenomenon of `left-handedness' and
	`right-handedness' remained outside of philosophical and mathematical
	thought, even though individual great philosophers and mathematicians,
	like Kant and Gauss, approached it.  Pasteur was a complete innovator
	in thought, and it is extremely important that he arrived at this
	phenomenon, and to the recognition of its significance proceeding from
	experiment and observation.  Curie proceeded from Pasteur's ideas, but
	developed them from a physical standpoint.  On the significance of
	these ideas for life see \cite{vernadsky1940biogeohimicheskie} [A large
	part of them was published in the book
	\cite[22--271]{vernadsky1992trudy}]; \cite{vernadsky1934problemy1}
	[\cite{vernadsky1980problemy}].
}  Right-handedness and left-handedness can be manifested geometrically only in
a space, in which vectors are polar and enantiomorphic.  The lack of straight
lines, and the strongly expressed curvature of the forms of life is,
apparently, connected to this geometrical property.  I will return to this
question further on, but I consider it necessary to note presently that we are,
apparently, dealing with a space, not corresponding to Euclidean space, but to
one of the forms of Riemannian space, inside organisms.

We are presently justified to admit the manifestation of the geometrical
properties corresponding to all three forms of geometry---Euclidean,
Lobachevskian and Riemannian---in the space in which we are living.  Further
investigation will show whether such a conclusion, logically completely
uncontradictable, is correct.\footnote{
	Mathematical thought has admitted the equal permissibility of the
	search for the manifestations of non-Euclidean geometry in the reality
	surrounding us long ago.  Perhaps, the thought of this was clear to
	Euclid himself when he separated the parallel postulate from his
	axioms.  Lobachevsky (1793--1856) was striving to prove the existence
	of the triangles introduced by him, proceeding from the rejection of
	this postulate, for cosmic space.  It seems to me that H.~Poincaré
	(\cite[3, 66]{poincare1902science}) most prominently emphasized the
	possibility of searches for the manifestations of non-Euclidean
	geometry in our physical environment.  This question raised no doubt
	with the ferment of thought,\footnoteRus{брожении мысли} occasioned by
	A.~Einstein (\cite{einstein1921geometrie}).  It can be objected that in
	these cases it was admitted, as it were, \lphr{tacito consensu} (by a
	tacit agreement) that geometry, of this form or another, is the same in
	all reality, while, as in the given case, we are dealing with a
	geometrical heterogeneity of the space in our reality.  The space of
	life is different from the space of inert matter.  I cannot see any
	basis for presuming such an admission contradictory to the foundations
	of our exact knowledge.
}  Unfortunately, the great amount of empirical observations, relevant here and
scientifically established, has not been assimilated in its significance by
biologists, and has not entered into their scientific world outlook.
Meanwhile, as P.~Curie showed, such a special state of space cannot occur in
the usual space without special circumstances; a dissymetrical phenomenon,
speaking his language, must always result from such a dyssimetrical cause.  The
fundamental empirical generalization that living originates only from living,
and an organism is born from an organism corresponds to this.  This is
manifested geologically in the fact that we observe an impassable boundary
between living and inert naturally-occurring bodies and processes in the
biosphere, which is not observable in any other terrestrial envelope.  There
are two sharply materially [and] energetically distinct media, mutually
penetrating and exchanging their constituent atoms, connected with the biogenic
flow of chemical elements, in it.  I will return to this phenomenon in more
detail further on.


\Section % 13
% The current period of an extraordinary manifestation of living matter through
% scientific thought.
We are currently living through an extraordinary manifestation of living matter
in the biosphere, genetically connected with the emergence of \lphr{Homo
sapiens} thousands of years ago, the creation, in this way, of a new geological
force, \emph{scientific thought,} dramatically increasing the influence of
living matter on the evolution of the biosphere.  Completely encompassed by
living matter, the biosphere is increasing the geological force of living
matter to an, apparently, unlimited degree, and, being transformed by the
scientific thought of \lphr{Homo sapiens,} is transitioning into one of its new
states---\emph{into the noosphere.}

As a manifestation of living matter, scientific thought \emph{cannot be} in
essence a reversible phenomenon---it can stop in the course of its motion, but,
once created and manifested in the evolution of the biosphere, it carries in
itself the ability of unlimited development in the course of time.  The course
of scientific thought in this respect, for example in the creation of machines,
is, as has been remarked long ago, completely analogous to the course of the
reproduction of organisms.

There is no irreversibility in the inert medium of the biosphere.  Reversible
cyclical physico-chemical, and geochemical processes strongly predominate in
it.  Living matter enters in them with its physico-chemical manifestations of
dissonance.\footnoteEd{
	The Earth as a whole has an irreversible development, as well, as is
	shown by the work with radioactive determination of the age of the
	rocks of the early Precambrian.  Biological evolution is strongly
	distinguished by a different rate of development
	(\cite{yanshin1988evolyuciya}).  [---Ed]
}

The growth of scientific thought, closely connected with the growth of man's
population of the biosphere---his reproduction, and his cultures of living
matter in the biosphere,---must be limited by the foreign living matter of the
environment, and must exert a \emph{pressure} on it.  For this growth is
connected to the quantity of rapidly increasing living matter, directly and
indirectly participating in scientific work.

This growth and the pressure connected to it is ever increasing, due to the
fact that the activity of the mass of created machines, whose increase in the
noosphere is subject to the same laws as those of the reproduction of living
matter itself, \ie\ is expressed by a geometrical progression, is dramatically
manifested in this work.

As the reproduction of organisms is manifested in the \emph{pressure} of living
matter in the biosphere, the course of the geological manifestation of
scientific thought puts pressure, by the instruments created by it, on the
inert medium of the biosphere containing it, creating the noosphere, the
kingdom of reason.

The history of scientific thought, of scientific knowledge, of its historical
course is being manifested in a new aspect, which has not been sufficiently
recognized to this day.  It must never be viewed as simply the history of one
of the humanities.  This history is, at the same time, \emph{the history of the
creation a new geological force---scientific thought,---in the biosphere,} not
present in the biosphere before.  This is the history of manifestation of a new
geological factor, of a new expression of the biosphere's state of
ogranization, forming tempestuously, as a natural phenomenon during the last
few tens of thousands of years.  It is not arbitrary, as every natural
phenomenon, it is lawful, as the paleontological process, creating the brain of
\lphr{Homo sapiens} and that social environment, in which scientific thought, a
new geological, consciously directed, force is being created as a consequence
of this environment, as a natural phenomenon connected with it, is lawful in
the course of time.

But the history of scientific knowledge, even as a history of one of the
humanities, is still unrecognized and unwritten.  There is not a single attempt
to do that.  It has just begun to exceed the limits of `biblical' time for us
only in the recent years, the existence of \emph{a single center} of its
emergence, somewhere in the region of the future Mediterranean culture eighty
thousand years ago, has started to become clear.  We are beginning to detect,
to establish unexpected for us, completely forgotten scientific facts lived
through by mankind, only with great gaps from cultural remains, attempting to
encompass them by new empirical generalizations.\footnote{
	The rapid change in our knowledge thanks to archeological excavations
	allows us to hope for very great changes in the near future.
}
