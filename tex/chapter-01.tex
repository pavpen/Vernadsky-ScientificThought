\part{Scientific Thought and Scientific Work as a Geological Force in the
Biosphere}

\Chapter{Man and mankind in the biosphere as a lawful part of its living
matter, part of its organization.  Physical-chemical and geometric
heterogeneity of the biosphere: the fundamental organizational
distinction---material-energetic and temporal---of its living matter from its
inert matter.  Evolution of the species, and evolution of the biosphere.  The
manifestation of a new geological force in the biosphere---the scientific
thought of social mankind.  Its manifestation is related to the ice age, in
which we live, with one of the geological phenomena repeating in the history of
the planet, whose cause exceeds the bounds of the Earth's crust.}


\Section % 1
% Inseparability of scientific thought from the living environment.
% Inseparability of the concepts of nature & cosmos from the biosphere &
% science.
Man, as well as everything living, is not a self-sufficient, independent of the
environment natural object.  However, even natural scientists in our time,
counterposing human beings and living organisms in general to the environment
of their life, very often do not take this into account.  But the
inseparability between living organism and its environment cannot presently
raise any doubt among contemporary naturalists.  The biogeochemist proceeds
from it, and strives to understand, express, and establish this functional
dependence precisely, and as deeply as possible.  Philosophers and contemporary
philosophy predominantly do not take into account this functional dependence of
man, as a natural object, and mankind, as a natural phenomenon, on the
environment of their life and thought.

Philosophy cannot sufficiently take this into account, as it proceeds from the
laws of the mind, which is, in one way or another, a final and self-sufficient
criterium for it (even in those cases, like in religious and mystical
philosophies, in which the reach of the mind is, in fact, limited).

The contemporary scientist, proceeding from the recognition of the reality of
one's surroundings, of the world subject to one's investigation---nature, the
cosmos, or world reality,\footnote{
	I will talk about the reality of the cosmos, instead of that of nature,
	here and further.  The concept nature, if we take it in a historical
	aspect, is a complex concept.  It very often encompasses only the
	biosphere, and it is more convenient to use it with just this meaning,
	or even not to use it at all (\autoref{sec:6}).  This would correspond
	to the vast majority of the uses of this concept historically in
	natural science and in literature.  The concept `cosmos' can be,
	perhaps, more conveniently applied to only the part of reality
	encompassed by science, a philosophically pluralistic conception of
	reality is possible at that, where there would be no single criterium
	for the cosmos.
}---cannot adopt this point of view as a basis for scientific work.

Because one presently knows exactly that man is \emph{not} located on a
structureless surface of the Earth, is \emph{not} located in direct contact
with cosmic space in a structureless nature, which is not lawfully connected
with him.  True, even the deeply penetrating contemporary naturalist often, out
of routine and under the influence of philosophy, forgets this, and does not
take it into account in his thought, and does not identify this.

Man and mankind are most closely connected, above all, with the living matter
inhabiting our planet, from which they cannot, in reality, be isolated by any
physical process.  That is possible only in thought.


\Section % 2
% We shall study the scientifically unambiguous 'living matter', and 'living
% natural body', rather than 'life'.
The concept of life and the living is clear to us in everyday life, and cannot
raise scientifically serious doubts in the actual manifestations of it, and in
natural objects corresponding to it---in natural bodies.  It was only in the
$20^{\mathrm{th}}$ century [with the discovery of] filter-passing viruses that
in science appeared facts, compelling us for the first time to ask
seriously---not philosophically, but scientifically---the question: Are we
dealing with a living natural body, or with a non-living natural body---an
inert one?

With viruses the doubt is called for by scientific observations, rather than
philosophical notions.  In this consists the great scientific significance of
their study.  That is presently on a right and firm path.  The doubt will be
resolved, and nothing, except a more precise notion of \emph{living organism},
would give, with this approach couldn't fail to give \dots

Along with this, however, we encounter another kind of doubts in
\emph{science}, called for by philosophical and religious searches.  Thus, for
example, phenomena, concerning the material-energetic environment of
manifestations which are philosophically \emph{common} to both living and inert
natural bodies, are scientifically studied in the works of the Bose Institute
in Calcutta.\footnoteRus{Института Бозе в Калькутте}\footnoteEd{
	The Bose
	Institute\footnoteTransl{\url{http://www.boseinst.ernet.in/index.html}}
	in Calcutta was founded by the Indian scientist Acharya Jagadish
	Chandra Bose\footnoteRus{Бозе Джегдиш Чандра} (1858--1937) in 1917.
	The institute studied the problems of physics, biophysics, inorganic
	and organic chemistry, biochemistry, the physiology of plants,
	selection, microbiology, etc. [---Ed.]
} They are not characteristic of, are weakly expressed in inert natural bodies,
and are strongly manifested in living ones, but are common to both.

This area, if it exists in the form in which Bose tried to establish it, of
phenomena common to inert and living natural bodies introduces nothing new in
the sharp distinction between them.  The distinction must manifest itself in
this area, as well, if only the latter's existence would be proven.

We must approach phenomena here, as well, not in the aspect in which Bose
approached them, not as phenomena of \emph{life}, but as phenomena of living
natural bodies, of \emph{living matter}.

To avoid any misunderstanding, I shall avoid the concepts `life', `living' in
all further exposition, since, if we stepped outside those
[phenomena],\footnoteTransl{inserted by transl. [---Pav]} we would inevitably
go beyond the limits of the phenomena of life studied in science, into a
foreign area or science---the area of philosophy, or, as is taking place in the
Bose Institute, into a new area of new material-energetic manifestations common
to all natural bodies of the biosphere, one lying outside the bounds of the
fundamental question of living organism, and living matter, which we are
presently interested in.

I shall, therefore, avoid the terms and concepts `life', and `living', and
limit the area which is subject to our investigation to the concepts
\emph{`living natural body'}, and \emph{`living matter'}.  Each living organism
\emph{in the biosphere}---natural object---is a living natural body.  \emph{The
living matter of the biosphere is the complex of living organisms in it.}

`Living matter', so defined, is a concept, completely precise, and fully
encompassing the objects studied by biology and biogeochemistry.  It is simple,
clear, and cannot raise any doubt.  We study only the living organism and its
complexes in science.  They are scientifically identical with the concept of
life.


\Section % 3
Man, like every living natural (or naturally occurring)\footnoteRus{природное
(или естественное)} body, is inseparably connected with a certain geological
envelope of our planet---\emph{the biosphere}, clearly distinct from the rest
of its envelopes, with a structure which is determined by its specific
\emph{state of organization},\footnoteTransl{организованностью, \ie\ 
organizedness, or being (constantly) organized. [---Pav]} and occupying a
lawfully expressible place in it as a distinct part of the whole.

Living matter, just like the biosphere, possesses its peculiar state of
organization, and can be viewed as a lawfully expressible \emph{function of the
biosphere.}

\emph{A state of organization} is not a mechanism.  It sharply differs from a
mechanism in that it is constantly in a state of
becoming,\footnoteTransl{становлении, \eg\ formation. [---Pav]} of motion of
all of its smallest material and energetic particles.  We can express this
state of organization in the course of time---in a generalization of mechanics,
and in a simplified model---as being such that none of its points (material or
energetic) lawfully returns to, ends up in a place, in a point of the biosphere
which it occupied at any earlier moment.  It can return to one of them only on
the order of mathematical accident, of very small probability.

The Earth's envelope, the biosphere, embracing the whole globe, has clearly
distinct dimensions, is determined to a large degree by the existence of living
matter in it---\emph{populating}\footnoteRus{заселена} it.  There is a
constant material and energetic exchange, materially expressed in the motion of
atoms brought on by living matter, between its inert non-living part, its inert
natural bodies, and the living matter inhabiting it.  This exchange in the
course of time is expressed as lawfully changing, constantly tending toward the
stability of an \emph{equilibrium}.\footnoteRus{равновесием}  The latter
permeates the whole biosphere, and this \emph{biogenic flow of
atoms}\footnoteRus{биогенный ток атомов} creates the biosphere to a large
extent.  The biosphere is thus connected inseparably and inherently with the
living matter populating it throughout the whole duration of geological time.

. . .

\Section % 4

. . .

\Section % 5

. . .

\Section \label{sec:6} % 6

. . .

\Section % 7

. . .

\Section % 8

. . .

\Section % 9

. . .

\Section % 10

. . .

\Section % 11

. . .

\Section % 12

. . .

\Section % 13

. . .
