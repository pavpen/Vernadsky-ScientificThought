\documentclass[twocolumn]{book}

\usepackage{etex} % Load this first.
\usepackage{etoolbox}
\usepackage{url}
\usepackage{graphicx}
\usepackage[T2A]{fontenc}
\usepackage[utf8]{luainputenc}
\usepackage{csquotes}
\usepackage[russian,french,german,english]{babel}
%\usepackage{balance} % Misplaces footnotes when balancing the columns.
%\usepackage{flushend} % Useless for our purpose.
\usepackage{footmisc}
\usepackage{bigfoot} % install {ncctools} and {bigfoot}
\usepackage{color}
%\usepackage[usenames,dvipsnames,svgnames]{xcolor}

% Font configuration:
%\usepackage[sc]{mathpazo}
%\linespread{1.05}  % line spacing for Palatino.
%\usepackage[scaled=0.95]{helvet}
%\usepackage{courier}

\usepackage[backend=biber,sortlocale=ru,bibencoding=utf8,citestyle=authortitle-trans]{biblatex}
%\usepackage{multicol} % We use up all of TeX's counters with this package.

\usepackage{tocloft}
%\usepackage[nottoc,numbib]{tocbibind} % This didn't seem to work.


%\usepackage[pdftex,colorlinks=true]{hyperref} % Load last.
% Add option 'draft' when debugging hyper-links accross page breaks:
\usepackage[colorlinks=true,pdfencoding=auto,linktoc=all]{hyperref}


% Load bibliography databases:
\addbibresource{bib/vernadsky.bib}
\addbibresource{bib/leibnitz.bib}
\addbibresource{bib/thought.bib}



%%
% Utility macros:
\def\eattoken#1{}



%%
% Colors:
\definecolor{deemphGray}{rgb}{0.65,0.65,0.65}



%%
% Typesetting macros:

% Typeset 'e.g.':
\newcommand{\eg}{e.$\,$g.}
\newcommand{\Eg}{E.$\,$g.}

% Typeset 'i.e.':
\newcommand{\ie}{i.$\,$e.}
\newcommand{\Ie}{I.$\,$e.}

% Typeset ---Ed., ---Pav, etc.:
\newcommand{\noteAuth}[1]{\textit{\mbox{---#1}}}

% Typeset [---Ed.], [---Pav], etc.:
\newcommand{\parenNoteAuth}[1]{\mbox{[\textit{---#1}]}}

% Typeset [Ed.:], [Pav:], etc.:
\newcommand{\parenNoteAuthPfx}[1]{\mbox{[\textit{#1:}]}}

% Typeset titles of works:
\newcommand{\rtitle}[1]{\emph{#1}}

% Typeset (e.g.) Latin phrases:
\newcommand{\lphr}[1]{\textsf{\emph{#1}}}

% Typeset text insertions, included for changing to an, e.g., more familiar
% style:
%  This is pretty ugly (or ugly pretty):
% \font\lightrm = cmr20 at 10pt
%\newcommand{\altStylePhr}[1]{{[\lightrm #1]}}
\newcommand{\altStylePhr}[1]{\textcolor{deemphGray}{[#1]}}

% Typeset chapters with by-line-type subtitles:
\newcommand{\ChapterByLine}[2]{\chapter[#1]{#1\\[1ex]\sc\large #2}}

% Typeset chapters:
\newcommand{\Chapter}[1]{%
	\refstepcounter{chapter}%
	\chapter*{Chapter \thechapter%
	  \begin{center}%
	    \rule{0.5\linewidth}{0.1mm}\\
	    \it\large #1\\[-1ex]%
	    \rule{0.25\linewidth}{0.1mm}
	  \end{center}%
	}%
	\chaptermark{}%
	\addcontentsline{toc}{chapter}{Chapter \thechapter}%
	\cftchapterprecistoc{#1}%
}

% Start a new section:
\newcounter{Section}  % A counter that doesn't reset every chapter.

\newcommand{\Section}{%
	\ifnumless{\value{section}}{1}{%
	}{%
	  \bigskip%
	}%
	\stepcounter{section}%
	\refstepcounter{Section}%
	\textbf{§\arabic{Section}. }%
}

% We use \eattoken to consume the space that hyperref inserts between
% \*autorefname and the reference number:
\def\Sectionautorefname{§\eattoken}
%\def\theHSection{\arabic{Section}}


%%
% Footnotes:
% Add a comma between the footnote marks:
\newcommand{\fncomma}{${\rule{-1pt}{0pt}}^{, }$}
\DeclareNewFootnote{default} % Author's footnotes.
\DeclareNewFootnote{Ed}[fnsymbol] % The editor's footnotes.
%\DeclareNewFootnote{Ed}[greek] % The editor's footnotes.
\MakePerPage{footnoteEd}
\DeclareNewFootnote{Transl}[Roman] % Our (translator's) footnotes.
\MakePerPage{footnoteTransl}

% Footnotes with original Russian phrases (e.g., which don't have a direct
% English language equivallent).
\DeclareNewFootnote{Rus}[roman]
\MakePerPage{footnoteRus}
%\newcommand{\footnoteRus}[1]{%
%	\footnoteTransl{%
%	  \begin{otherlanguage}{russian}%
%	    #1%
%	  \end{otherlanguage}%
%	}}

% Use when running pdflatex to generate the font cache, instead of the above
% (when avoiding running out ouf TeX counters by disabling <bigfoot>):
%\newcommand{\footnoteEd}[1]{\footnote{#1}}
%\newcommand{\footnoteTransl}[1]{\footnote{#1}}
%\newcommand{\footnotemarkTransl}{\footnotemark}
%\newcommand{\footnotetextTransl}{\footnotetext}


%%
% Widow, orphan, etc. control:
\clubpenalty=10000
\widowpenalty=10000
\brokenpenalty=4991
\predisplaypenalty=10000
\postdisplaypenalty=1549
\displaywidowpenalty=1602



\title{Scientific Thought as a Planetary Phenomenon}
\author{Vladimir Ivanovich Vernadsky \and In translation by the LaRouche Movement.}


\begin{document}

\selectlanguage{english}

\frontmatter

\begin{titlepage}

\begin{center}
\Huge\textbf{Scientific Thought as a Planetary Phenomenon}


\bigskip
\LARGE{V. I. Vernadsky}

\Large{1936--1938}


\bigskip
\Large{Translated from Russion by Pavel M. Penev for the LaRouche movement.}
\foreignlanguage{russian}{\nocite{vernadsky2001thought}}

October 20, 2010--present

\vfill

\copyright\ 2012 Pavel M. Penev

\end{center}

\end{titlepage}


% Add the table of contents to the PDF bookmarks:
\cleardoublepage
\phantomsection
\pdfbookmark[0]{\contentsname}{toc}

\tableofcontents
\ChapterByLine{Remarks to the Electronic Edition~\dots}{
V. I. Vernadsky Electronic Archive\\
\url{http://vernadsky.lib.ru}
}

The present electronic edition of V. I. Vendasky's book \rtitle{Scientific
Thought as a Planetary Phenomenon\footnoteRus{Научная мысль как планетное
явление}} was being prepared according to the
edition\fullcite{vernadsky1991thought} at the end of 1999.

The first four chapters were prepared by April, 2000, and added to the Maxim
Moshkov library (\url{http://lib.ru/FILOSOF/WERNADSKIJ/}).  These first
chapters were carefully proofread and, I hope, contain very few printing
errors.

The fifth and sixth chapters were proofread (also quite carefully, though not
as well as the first four) by the end of November, 2000.  They were published
on the server of the Electronic Archive (\url{http://vernadsky.lib.ru}), but
were not sent to the Moshkov library in the hope that the remaining four
chapters would be prepared sufficiently quickly.

Unfortunately, because of insufficient time, the work on the remaining chapters
kept dragging on and on, to the point that I decided to use the electronic
version of these chapters, which was prepared by the Russian Foundation for
Fundamental Research\footnoteRus{Росийским Фондом фундаментальных исследований}
from the edition \fullcite{vernadsky1997thought}.

However, comparing these two editions, it seemed to me, that the earlier one,
from 1991, was much closer to the original text of V. I. Vernadsky.  The 1997
edition is filled with slight editorial corrections, which, though nowhere (it
seems) distort Vernadsky's meaning, nevertheless, quite strongly change his
manner of exrpession, and that in such a way that at these places the mind is
often just tripped up, and it is at once apparent that Vladimir Ivanovich could
not have written in that manner.  It is, therefore, necessary to streighten out
chapters 7--10 according to the 1991 edition with time.  It is also necessary
to proofread all chapters once again, and correct any remaining errors.

I include the introductions of the editors of both editions at the begining of
this book, which tell about the history of the writing of Vladimir Ivanovich's
book, as well as about the history of its hard and quite controversial
publication.

For commercial use of the electronic edition of \rtitle{Scientific Thought as a
Planetary Phenomenon}, or (which would be just terrific ;-) for aid with its
proofreading, contact me at the address indicated on the
\url{http://vernadsky.lib.ru} server.

*Note:* The electronic edition is being prepared in the LaTeX format; it is
necessary to update that version, and not the derived HTML version in the Maxim
Moshkov library when proofreading.

\begin{flushright}
Sergey Mingaleev\footnoteRus{Сергей Мингалеев}\\
October 16, 2001
\end{flushright}

\ChapterByLine{Preface and remarks by A.~L.\ Yanshin~\dots}{%
A.~L.\ Yanshin\\
{\rm\normalsize Chairman of the Committe of the Academy of Sciences of the USSR
for the Exploitation of the Scientific Heritage of Academician V. I.
Vernadsky}\footnotemarkTransl
\\[1ex]
F.~T.\ Yanshina\\
{\rm\normalsize Director-founder of the museum home of Academician V.~I.\ 
Vernadsky}\\[4ex]
\begin{center}
	\textit{The electronic version of the preface and the remarks was
	prepared from the edition in the book
	\fullcite{vernadsky1991thought}.}
\end{center}
}%
\footnotetextTransl{\foreignlanguage{russian}{Комиссия по разработке научного
наследия академика В.~И.\ Вернадского}}%


\section*{Preface}

The name of Vladimir Ivanovich Vernadsky has become widely known in our
country.  There is nobody with even the slightest degree of education, who
hasn't read, if not Vernadsky's works, then, at least, numerous newspaper and
magazine articles about him and his work.

There is a Vernadsky Avenue in Moscow.  One of the largest institutes at the
Academy of Sciences of the USSR, the Institute of Geochemistry and Analytical
Chemistry,\footnoteRus{Институт геохимии и аналитической химии} bears his name.
There is a Committee for the Exploitation of the Scientific Heritage of
Academician V.~I.\ Vernadsky, which publishes its own circular, at the
Presidium of the Academy of Sciences of the USSR.  Branches of that Committee
work in Leningrad and in Kiev.  There have been grants under Vernadsky's name
established at Moscow, Leningrad, Kiev, and Simferopol University.  National
scientific centers for the study of the work of this prominent thinker and for
its application to the solution of contemporary problems exist in Odessa,
Rostov-na-Don, Erevan, Simferopol, Ivanov, and in other cities in the USSR, and
abroad---in Prague, Oldenburg and Berlin.\footnote{
	Also named after V. I.  Vernadsky are: the National Geological
	Museum,\footnoteRus{Государственный геологический музей} the National
	Public University of Biospheric Studies,\footnoteRus{Всесоюзный
	народный университет биосферных знаний} the Central Scientific Library
	of the AS UkrSSR,\footnoteRus{Центральная научная библиотека АН УССР}
	the Student Sociological Center ``Noosphere'',\footnoteRus{Студенческий
	социологический центр ``Ноосфера''} the peak in the basin of
	Podkamennaya Tunguska River, the crater on the dark side of the Moon,
	the peninsula in Eastern Antarctica near the Sea of
	Astronauts,\footnoteRus{Море Космонавтов} the forest on the island of
	Paramushir (Kuril Island), the subglacial forests in Eastern
	Antarctica, the underwater volcano in the Atlantic Ocean, the mine in
	the region of Lake Baikal, the mineral Vernadit,\footnoteRus{вернадит,
	$Mn^{4+}, Fe^{3+}, Ca, NaS(O,OH)_{2n}\cdot H_2O$} the diatomaceous
	algae, research vessel ``Academician Vernadsky'' of AS UkrSSR, the
	steamboat ``Geologist Vernadsy''\footnoteRus{Геолог Вернадский} of the
	Kama River Shipping company,\footnoteRus{Камское речное пароходство}
	the Vernadsky village near Simferopol, the Vernadsky railway station on
	the Kazan line, the subway stop ``Vernadsky Avenue'' in Moscow, the
	Biosphere Museum at the Leningrad branch of the Institute of the
	History of Natural Science and Technology of the AS USSR.  A V.~I.\ 
	Vernadsky monument has been erected in Kiev, a memorial plate is in
	place on the old building of Moscow State University M.~V.\ 
	Lomonosov,\footnoteRus{МГУ им.\ М.~В.\ Ломоносова} on Vernadsky Avenue
	in Moscow, on the building of Leningrad State
	University,\footnoteRus{Ленинградского государственного университета}
	as well as on the building of the Kiev State University T. G.
	Shevchenko.\footnoteRus{Киевского государственного университета им. Т.
	Г.  Шевченко}  V.~I.\ Vernadsky grants are awarded for exceptional
	scientific work in the areas of mineralogy, geochemistry and
	astrochemistry by the Academy of Sciences of the USSR and by the
	Academy of Sciences of the UkrSSR.  A golden medal named after him has
	been established by the Academy of Sciences of the USSR.}

V.~I.\ Vernadsky's 125th birthday was celebrated in March 1988 in our country,
as well as abroad (in Prague and in Berlin).

The celebrations srpead very widely.  An exhibition dedicated to his work was
opened on January 15, 1988 at the Exhibition of the Achievements of the
National Economy.\footnoteRus{ВДНХ, from выставка достижений народного
хозяйства}  Scientific symposia on different directions of V.~I.\ Vernadsky's
research took place successively in Leningrad, Kiev, and Moscow with the
participation of foreign scientists from March 3 to 11.  A commemorative
conference took place in Balshoy Theatre\footnoteRus{Большой театр} in Moscow
on his birthaday, March 12, with the participtation of public organizations.
Separate conferences and scientific sessions took place during the same days in
Ivanov, Odessa, Simferopol, Rostov-na-Don, Yerevan, Baku,
Almaty,\footnoteRus{Алма-Ате} Novosibirsk, Irkutsk, and in many other
scientific centers of the nation.  The proposal to create an International
V.~I.\ Vernadsky\footnoteRus{Международного фонда В.~И.\ Вернадского} Fund for
subsidizing the translation of his works in other languages, finding materials
about him in foreign archives, and the invitation of scientists from foreign
nations to the USSR for reports and lectures on the contemporary development of
scientific problems noted by V.~I.\ Vernadsky was accepted.\footnote{See the
information at the end of the book.}  Articles about him, and his multifaceted
scientific work have appeared in almost all Soviet and international newspapers
and magazines.

Publishing house \rtitle{Nauka}\footnoteRus{Наука} released 4 volumes of works
by V.~I.\ Vernadsky, as well as his \tciteo{vernadsky1988pisma}, including the
book \tciteo{vernadsky1988filosofskie}, in which the work \rtitle{Scientific
Thought as a Planetary Phenomenon} was republished as a first part, now
published with reconstructions of all those passages, abridged for its first
edition in 1977, according to the archived original manuscript, before the very
anniversary in February, 1988.  The book was released in a 20,000 run.  The
whole run was bought out during the very first days after its appearance in the
bookstores' windows.  A barrage of letters requesting the release of an
additional run of \tciteo{vernadsky1988filosofskie}, or at least of its first
part, was received at the Scientific-publishing council of the Academy of
Sciences of the USSR.\footnoteRus{Научно-издательский совет АН СССР}

The appearance of \tciteo{vernadsky1988filosofskie} in 1988 found a broad
positive response from the press.  For example, the article
\tciteo{lukyanov1988neizvestny} was published in the journal
\ctciteo{lukyanov1988neizvestny} from September 29, 1988, in which the author
F.~Lukyanov\footnoteRus{Ф. Лукьянов} wrote: \begin{quotation}
  The name of academician Vladimir Ivanovich Vernadsky (1863--1945) cannot be
  called unknown to the Soviet reader.  However, he is still known among us in
  his homeland mainly as a scientist-naturalist, a historian of science, and is
  almost unknown as a thinker, a philosopher, even though his philosophical
  heritage has become a recognized phenomenon of European and world scientific
  thought long ago.
  
  The just-released book by V. I. Vernadsky from publishing house \rtitle{Наука}
  (\rtitle{Science}) \tciteo{vernadsky1988filosofskie} finally presents him
  to\footnoteTransl{omitting `и'} our reading public as a philosopher and
  thinker.  This book is, in essence, the first realization of a complete,
  unabridged publication of the essential works of the Russian thinker, above
  all the fundamental work \rtitle{Scientific Thought as a Planetary
  Phenomenon}, written in the period between the 1880s and 1940s, which has
  either completely disappeared, or has long ago become a bibliographic rarity.
\end{quotation}

During the preparation of the present edition of V.~I.\ Vernadsky's
% \tciteo: Add English translation of the title.
\tciteo{vernadsky1991thought} for publication its text was compared
with the manuscript of S.~N.\ Zhidovinov\footnoteRus{С. Н. Жидовиновым} once
again, with the help of collaborators from the Archive of the Academy of
Sciences of the USSR,\footnoteRus{Архива АН СССР} which enabled the correction
of some small inaccuracies, unnoticed in the previous editions, as well as the
restoration of the author's style, orthography, and punctuation where possible.

What does the book offered to the reader's attention present?  It is necessary
to shortly pause for a look at the development of V.~I.\ Vernadsky's ideas,
which have found their fullest reflection in this work, to answer this
question.
\medskip

It follows from letters to his wife Natalya Egorovna,\footnoteRus{Наталье
Егоровне} and to a few scientists, as well as from preserved diaries of
Vladimir Ivanovich, that his attention was attracted by the ever-increasing
technological might of mankind, which became comparable in its scale to the
most formiddable geological processes, already in his early years, i.~e.\
already at the end of the last century.  This activity irreversibly changes the
face of the whole Earth, of all of its nature in the physical-geographical and
chemical aspect.  (V.~I.\ Vernadsky didn't yet use the term `biosphere' at that
time.)

Such thoughts occurred not only to V.~I.\ Vernadsky.  He mentions his
predecessors and contemporaries in this aspect in his later works with his
characteristic courtesy.\footnoteRus{щепетильностью}  The American geologist
Charles Schuchert proposed viewing the contemporary epoch as the beginning of a
new, psychozoic age of the history of the Earth, emphasizing the significance
of the psychological activity of mankind as a geological factor with this name,
in 1933.\footcite[80]{schuchert1933geology}  Our Russian scientist A.~P.\
Pavlov, who invited V.~I.\ Vernadsky to teach mineralogy at Moscow
University in 1890, also thought that a new geological period in the Earth's
history began with the appearance of man on it, which he proposed to call
anthropogenic (from the Greek word
`anthropos'---man).\footcite{pavlov1922lednikovye}  There were also other
statements of similar character at the end of the past, and the beginning of
the present century.

However, V.~I.\ Vernadsky, not satisfied with general statements, began
dilligent labor on a quantitative estimate of the scale of human activity.
V.~I.\ Vernadsky noted the minerals and new chemical compounds formed as a
result of mankind's industrial activity, and gave the first estimates of the
total volume and mass of such `technogenic' minerals already in his
\rtitle{Mineralogy} courses,\footnoteTransl{
	See \refsmartcites{vernadsky1891mineralogy-1,
	vernadsky1891mineralogy, vernadsky1898mineralogy,
	vernadsky1899mineralogy, vernadsky1900mineralogy,
	vernadsky1906mineralogy, vernadsky1908mineralogy,
	vernadsky1910mineralogy-v1, vernadsky1910mineralogy-v2}.
} which were being republished, with additions every time, during the years of
his work at Moscow University (between 1891 and 1912).

He started publishing his \rtitle{Опыт описательной минералогии} (\rtitle{Essay
on Descriptive Mineralogy}),\footnoteTransl{
	See \refsmartcites{vernadsky1909opyt-v2, vernadsky1910opyt-v3,
	vernadsky1912opyt-v4, vernadsky1914opyt-v5}.
} subsequently encompassing all native elements, including gases, as well as
their sulphuric\footnoteRus{сернистые} and selenious\footnoteRus{селенистые}
compounds, in 1908.  In these installments, which were later collected in the
$2^\mathrm{nd}$ and $3^\mathrm{rd}$ volume of \rtitle{V.~I.\ Vernadsky's
Selected Works} (1955 and 1959),\footnoteTransl{
	\Ie\ \refsmartcites{vernadsky1955sochineniya-v2,
	vernadsky1959sochineniya-v3}.
} he includes, within the description of almost every mineral, or its groups, a
separate section titled ``Mankind's Work'',\footnoteRus{Труд человека} or
``Mankind's Activity'',\footnoteRus{Деятельность человека} in which he gives
numbers for their global extraction and refining, and communicates information
about the direct and indirect influence of human activity on the formation and
distribution of one or another mineral or chemical compound (for example,
hydrogen sulphide).

V.~I.\ Vernadsky published \rtitle{The History of Natural
Waters},\footnoteRus{\rtitle{Историю природных вод}} which he himself viewed as
the second volume of the \rtitle{History of Minerals of the Earth's
Crust},\footnoteRus{\rtitle{Истории минералов земной коры}} in two books in
1933, and 1934.\footnoteTransl{
	\Ie\ \refsmartcites{vernadsky1933istoriya-v2p1-1,
	vernadsky1934istoriya-v2p1-2}.  The third book of the series, published
	in 1936, is \refcite{vernadsky1936istoriya-v2p1-3}.
}  He dedicates quite a few pages to the conscious, and unconscious influence
of mankind on the geographical distribution, and on the composition of all
waters on the Earth in this work.  Vernadsky had concluded even then that
\begin{quote}
  the pristine rivers are quickly disappearing, or have disappeared, and have
  been replaced by a new type of formation, new waters, which had not existed
  earlier.  A transformation of the natural waters, and the simultaneous
  creation of new cultural rivers, lakes, reservoirs, coastal sea formations,
  soil solutions is going on on the vast territory of Eurasia, and in the last
  century also in America and in Australia---in the whole biosphere.
  
  This process reaches inward, changes the mode of the interstitial
  % FIXME: stratisphere?
  waters\footnoteRus{режим пластовых вод} of the biosphere and stratisphere.
  The transformation of vadose water---ground
  water\footnoteRus{верховодок---вод грунтовых} has been going on for millenia,
  then started the transformation of interstitial artesian
  waters\footnoteRus{вод пластовых напорных} by boring and ore mining.  Now its
  effect reaches more than two kilometers below the Earth's surface.
  
  The old species of surface, interstitial waters, soil waters, and
  springs\footnoteRus{старые виды поверхностных, пластовых вод, вод почв и
  источников} are disappearing and changing throughout the whole biosphere, new
  cultural waters are emerging.\footcite{vernadsky1960sochineniya-istoriya}
\end{quote}

Parallel to the study of the influence of mankind's activity on the changing of
Earth's nature, V.~I.\ Vernadsky began developing the study of the
biosphere---that envelope of the Earth, in which `living matter' is
concentrated---already in 1914--1916.  He didn't like the unnecessary coinage
of words, the creation of new terms, but had magnificent knowledge of all world
scientific literature, and employed its terminology extensively.  Such was the
case with the term `biosphere'.  It was first used by the French scientist
Jean-Baptiste Lamarck\footnoteRus{
	Жаном Батистом Ламарком, a.k.a.\ Жан Батист Пьер Антуан де Моне Ламарк
	(Jean-Baptiste Pierre Antoine de Monet, Chevalier de Lamarck).
} in a work on hydrogeology to refer to the complex of living organisms
inhabiting the globe, already in 1804.\footnoteTransl{
	From what was available on the Internet, it seems that the reference is
	to Lamarck's 1802 \btcite{lamarck1802hydrogeologie} where, in the last
	but one paragraph of the foreword, Lamarck seems to say that a good
	physics of the Earth requires studying three aspects of it, which share
	the same physical body: the atmosphere (meteorology), the Earth's crust
	(hydrogeology), and that of living bodies (biology).  The name
	`biosphere', however, does not seem to be used there.
}  The Austrian geologist Eduard Sueß,\footnoteRus{Эдуард Зюсс} and the German
scientist Johannes Walther\footnoteRus{Иоган Вальтер} used it at the end of the
$19^\mathrm{th}$ c., again with a meaning similar to Lamarck's concept.  V.~I.\ 
Vernadsky introduced a completely different, far deeper meaning to this term.
He introduced the term `living matter' for the complex of living organisms
inhabiting the Earth, but called biosphere that environment in which this
living matter is located, \ie\ the whole water envelope of the Earth, since
living organisms exist at even the greatest depths of the World Ocean, the
lower part of the atmosphere, where insects, birds, and people fly, as well as
the top part of the solid envelope of the Earth---the lithosphere, where living
bacteria can be encountered in underground waters at depths on the order of
$2\,\mathrm{km}$, and man has now penetrated to even greater depths, exceeding
$3\,\mathrm{km}$, with his shafts in the regions of gold deposits in India,
South Africa, and Brazil.  There is a `film of life',\footnoteRus{пленка жизни}
where the concentration of living matter is maximum, in the biosphere.  This is
the land surface, the soils, and the top layers of the World Ocean's waters.
The amount of living matter in the biosphere rapidly diminishes with distance
above and below it.

V.~I.\ Verndasky estimated the total amount of living matter in the
contemporary biosphere of the Earth, established the magnitude of the energy
locked up in it, carefully studied the process of absorption of solar energy
with the aid of chlorophyll in green plants on land, and algae in the World
Ocean, traced its transformation, and its influence on the generation of many
`vadose'\footnoteRus{``вадозных''} minerals, characteristic only of the
biosphere, clarified the character of solar energy's entry into the depths of
the Earth with the deposits of organic matter created by it, and analyzed all
transformations which occur in living, bioinert, and inert, as he called them,
matter of this most important envelope of the Earth for mankind.

V.~I.\ Vernadsky presented the results of his studies in numerous articles, in
the book \btcite{vernadsky1926biosfera}, which was first published in 1926, and
has been subsequently reprinted a few times, and in the fundamental work
\btcite{vernadsky1965himicheskoe}, which was first published after the author's
death, in 1965.  Many articles, brochures, and books dedicated to V.~I.\
Vernadsky's teachings about the biosphere, to its detailed presentation, to
commentary on it, and, unfortunately, only partially, to its development, have
appeared in the Soviet and in the foreign press in connection with the
increased attention to the tasks of the preservation of
nature\footnoteRus{охраны природы} during the past decade.  There is,
therefore, no need to delve into it in the foreword to the present book.  It
is, however, important to emphasize that V.~I.\ Vernadsky viewed human activity
as a process imposed on the biosphere, foreign to it by its nature from the
beginning.  We can suppose that the technogenic character of this human
activity, interfering much with the naturally occurring course of natural
phenomena,\footnoteRus{естественный ход природных процессов} contradicting
them, prompted him to have such thoughts.

We can judge of the [view of the]\footnoteTransl{
	Interpolated to express the implied meaning. [---Pav]
} `imposed',\footnoteRus{``наложенном''} foreign character of mankind's
industrial activity from numerous statements of V.~I.\ Vernadsky even in his
works from the beginning of the thirties.  So, he wrote in the mentioned
\rtitle{History of Natural Waters} about technogenic solid minerals, and
waters: ``These new chemical compounds---`artificial', \ie\ created with the
participation of the will and the consciousness of man, can, for now, be put
aside in the study of the history of natural
bodies''.\footcite[87]{vernadsky1960sochineniya-istoriya}\fncomma
\footnoteTransl{
	Here's what \refcite[§133]{vernadsky1933istoriya-v2p1-1} actually says:
	
	\smallskip
	\begingroup \leftskip 3em\rightskip 1em
	  \textbf{§133. }We are now living at only the very beginning of the
	  Psychozoic Age.  It is impossible to encompass its results
	  completely.
	  
	  We are still in a transitional period.  However, we cannot leave
	  without attention an on-rushing force changing the history and
	  composition of natural waters.
	  
	  Problems of the quantification---in all varieties---of the change of
	  natural waters by human culture have come to the fore, it being
	  necessary to reconstruct the character of those waters, which existed
	  a hundred thousand, and more, years ago, as well as in the previous
	  geological periods.  These problems have hardly been touched upon,
	  but they can be encompassed by scientific thought, and it is
	  necessary to strive toward their resolution, and have them in mind in
	  studying the history of natural waters.
	  
	  We find ourselves in the same condition, which we run into in other
	  branches of mineralogy,---with the emergence of new natural
	  compounds created by culture, changing the history of natural bodies
	  of the same, or of similar composition.  I already regarded that in
	  the history of metals (I, §267 etc.); and the same is reflected in
	  the history of natural gases ($CO_2, SO_3, H_2S$ and so forth).
	  These new types of chemical compounds---`artificial', \ie\ created
	  with the participation of the will and consciousness of man, can, for
	  now, be left aside in studying the history of natural bodies.
	  
	  But this is, obviously, a temporary solution to the problem.  We must
	  never fail to take into account the products of human work in the
	  history of numerous minerals, for example, carbon dioxide.  We must
	  neither fail to take them into account in the history of natural
	  waters.  However, on the other hand, it is impossible to include it
	  completely in our present considerations.  A host of waters connected
	  with engineering are constantly and quickly changing,---are
	  temporary, transitional phenomena.  Many of the new waters are
	  negligible in mass---are rare, quickly disappearing `minerals'.
	  
	  I will consider these new waters---a creation of cultural life---only
	  in so far as this is necessary for understanding the essential main
	  features of the history of natural waters.  This, however, is, of
	  course, only a temporary solution of the question---this is the
	  intrinsic\footnoteRus{исконный} path of the naturalist, seeking the
	  important, but not a logical sequence, in the complex phenomenon,
	  real at a given moment.
	  \par
	\endgroup
	[---Pav]
}

However, V.~I.\ Vernadsky arrived at the unavoidable conclusion about the
evolution of the Earth's biosphere, about the quantitative and the qualitative
change of its main component part---living matter, about the stages of the
biosphere's evolution in the last decade of his life.  Such a course of
thoughts brought him to the conclusion that the emergence of man, and the
impact of his activity on the surrounding natural environment is not an
accident, is not an `imposed' process on the natural course of events, but is,
rather, a definite, lawful stage of the evolution of the biosphere.  This stage
has to lead to the condition that the Earth's biosphere must transition into a
new state, which he proposed to call `noosphere' (from the Greek word
`noos'\footnoteRus{``ноос''}---mind), under the influence of scientific
thought, and the collective labor of unified mankind, directed toward the
satisfaction of all of its material and spiritual necessities.  V.~I.\
Vernadsky did not invent this term, nor the term `biosphere' himself.  He
lectured on biogeochemistry and the development of the biosphere at Collège de
France from 1922 to 1926 during his long foreign assignment, and the French
mathematician Édouard Le Roy, student of these lectures, published an article
about them in 1927, in which he employed, for the first time, the term
`noosphere', used by other French scientists and by V.~I.\ Vernadsky further
on.

The work \rtitle{Scientific Thought as a Planetary Phenomenon}, judging from
the diaries of V.~I.\ Vernadsky and from his letters, was written mainly in
1937--1938, \ie\ during the most tragic years of our history.  V.~I.\ Vernadsky
was far from indifferent to the events of those days.  His friends and students
were repressed.  Trying to prove their innocence, and the erroneousness of
their arrests, he wrote letters to J.~V.\ Stalin, N.~I.\ Ezhov, and
L.~P.\ Beria.  His diaries from these years are filled with heavy words.  But
the book, written by him for the future generations, was permeated by optimism,
faith in the triumph of human reason.

It is hard to completely characterize the content of the book.  It is
significantly broader than the book's title, though the idea of the global
significance of scientific thought permeates it from beginning to end, and
connects all of its parts.  Essentially, this book is an introduction to the
teaching of the noosphere.  Many places in it are dedicated to the analysis of
the conception of this term.  Along with that, the role of mankind in the
development of the biosphere is painted with the broad strokes of a great
painter, the concept of living matter and its state of organization, of the
evolution of the biosphere and the inevitability of its gradual transformation
into the noosphere, of the conditions neccessary for such a transition, of the
basic stages of development of human culture and its further destiny, of
biogeochemistry as a main scientific area of the study of the biosphere, of the
fundamental differences between the living and the inert matter of this evelope
of the Earth is given.

The work \rtitle{Scientific Thought as a Planetary Phenomenon} occupies a
special place among the works of V.~I.\ Vernadsky.  It is distinguished by an
unusual breadth of the range of questions considered in it, and by the specific
character of the main problems examined.  The breadth of the views of their
author about things, and the significance of the scale on which he poses
questions have always been inherent to V.~I.\ Vernadsky's works.  However,
these qualities of the scientist have been brought to a most prominent and
powerful expression in the work being published.  Nature, human society,
scientific thought are examined in their indissoluble unity, and the reality
surrounding us is painted in a truly universal\footnoteRus{вселенской, \ie,
also, `ecumenical' (see \autoref{ch:3}).} vastness.

\rtitle{Scientific Thought as a Planetary Phenomenon}---this is an apex of
V.~I.\ Vernadsky's work, a grandiose, in its intention, summary of his
meditations on the destiny of scientific knowledge, on the relationship between
science and philosophy, on the future of mankind.  It can be characterized as
an impressive, though unfinished, synthesis of the ideas being developed by the
scientist in the last period of his life.

Deep thoughts about the evolution of mankind on geological and socio-historical
scales of time are contained in the book.  It must be admitted that this is the
first attempt in world literature to generalize the evolution of our planet as
a single cosmic, geological, biogenic, and anthropogenic process.  The leading
transformative role of science and the socially organized labor of mankind in
the present and future of the planet is revealed in the work.  Scientific
thought, science is viewed and analyzed as the most important force of
transformation and evolution of the planet.

We must not fail to note that the book offered to the reader's attention has
also a deeply philosophical content.  V.\ I.\ Vernadsky was not simply
intrested in philosophy, but studied the works of philosophers of various
schools and currents thoroughly since his teenage years.  He considered the
collection and generalization of scientific facts as inseparable from the
philosophical understanding of the reached scientific conclusions, which is
especially distincly evident from his diaries and his correspondence.

Already in 1902, beginning his work on the history of the development of human
culture, he wrote to his wife Natalya Egorovna: ``I view the significance of
philosophy in the development of human knowledge entirely differently from the
majority of naturalists, and ascribe an enormous, fruitful signicance to it.
It seems to me that these are two sides of the same process---completely
unavoidable and inseparable sides.  They are separated only in our minds.  Were
one of them to die away, the living growth of the other would cease\dots{}
Philosophy always concludes \emph{germs, }sometimes even anticipates whole
areas of the future development of science, and, only thanks to the
simultaneous work of the human mind in this area as well, correct criticism of
the unavoidably over-simplified notions of science is produced.  Such
significance of philosophy, as the roots and vital atmosphere of scientific
endeavor, can be precisely and clearly traced in the history of the development
of scientific thought.''\footcite{vernadsky1988trudy-p21}

V.\ I.\ Vernadsky stayed true to the principle presented in this letter his
whole life.  Statements with similar meaning can be found in numerous other
letters and works of his, especially in the many publications on the history of
scientific knowledge.  All of them are permeated by philosophical
conceptualizations of the presented material.

However, we, it seems, unexpectedly encounter different statements of
V.\ I.\ Vernadsky's, which separate philosophy from scientific knowledge, and
even mention it alongside religion, in works, letters, and diaries from the
30s.  In order to understand this, it is necessary to take into account that in
the given case we are talking about the dominant in those years philosophy of
vulgar dialectical materialism, ordering not only the representatives of the
social sciences, but also natural scientists what conclusions and inferences to
make, in order for them to completely correspond to the philosophical ``laws''.
V.\ I.\ Vernadsky could not accept such a philosophy, for which he was
criticized by A.\ M.\ Deborin, who accused him of
idealism.\footnote{\cite{deborin1932problema}}  V.~I.~Vernadsky responded to
this criticism with great dignity, even though it painfully wounded his
pride.\footnote{\cite{vernadsky1933povodu}}  He always thought that an unbiased
collection of as many facts about the topic of investigation as possible, their
subsequent objective generalization, and, only afterward, a philosophical
understanding must be the basis for every investigation.  By the way,
V.~I.~Vernadsky regarded Karl Marx as a scientist with great respect
precisely because a great amount of thoroughly and conscientiously collected
material lay at the foundation of the
\btcite{marx2010capital-v1}\nocite{marx2007capital-v2,marx2010capital-v3}.

The development of V.\ I.\ Vernadsky's philosophical views is reviewed more in
depth in the article ``From the Editorial Board'' in the mentioned book
\tciteo{vernadsky1988filosofskie}.  Wide-ranging commentaries on the work
\rtitle{Scientific Thought as a Planetary Phenomenon} were published in that
book, as well as in the form of appendices, articles by B.~M.~Kedrov,
I.~V.~Kuznetsov, S.~R.~Mikulinskiy, and A.~L.~Yanshin written at various times,
in which the questions of V.~I.~Vernadsky's worldview and his teaching about
the gradual transition of the biosphere into the noosphere are reviewed from
different points of view.

We have significantly reduced the commentary in the present edition, which is
indented for the widest possible circle of readers, transferring the necessary
part of it to footnotes.  The majority of editorial comments have been updated.

V.~I.\ Vernadsky's articles \btcite{vernadsky1902nauchnom} and ''A Few Words
about the Noosphere'', as well as fragments (six sections) from the manuscript
\rtitle{Scientific Thought as a Planetary Phenomenon}, which, as was proper,
were not included in the text by the author himself.  The reader may be
interested in the content of these fragments, and obtaining the the small-run
journal \refparencite{voprosyIstorii1988n1} is fairly difficult.  Therefore, we
decided to reproduce them in the present edition, after comparison with the
author's original, and the correction of errata and modifications in the
journal's version.

The article \btcite{vernadsky1902nauchnom} was first published in the journal
\ctciteo{vernadsky1902nauchnom} in 1902, and was re-published a few more times
subsequently in various collections with little changes during Vernadsky's
life.\footnoteEd{
	We are re-publishing it according to the edition
	\fullcite{vernadsky1988nauchnom}.
}

. . .



\mainmatter

\part{Scientific Thought and Scientific Work as a Geological Force in the
Biosphere}

\Chapter{Man and mankind in the biosphere as a lawful part of its living
matter, part of its organization.  Physical-chemical and geometric
heterogeneity of the biosphere: the fundamental organizational
distinction---material-energetic and temporal---of its living matter from its
inert matter.  Evolution of the species, and evolution of the biosphere.  The
manifestation of a new geological force in the biosphere---the scientific
thought of social mankind.  Its manifestation is related to the ice age, in
which we live, to one of the geological phenomena repeating in the history of
the planet, whose cause exceeds the bounds of the Earth's crust.}


\Section % 1
% Inseparability of scientific thought from the living environment.
% Inseparability of the concepts of nature & cosmos from the biosphere &
% science.
Man, as well as everything living, is not a self-sufficient, independent of the
environment natural object.  However, even natural scientists in our time,
counterposing human beings and living organisms in general to the environment
of their life, very often do not take this into account.  But the
inseparability between living organism and its environment cannot presently
raise any doubt among contemporary naturalists.  The biogeochemist proceeds
from it, and strives to understand, express, and establish this functional
dependence precisely, and as deeply as possible.  Philosophers and contemporary
philosophy predominantly do not take into account this functional dependence of
man, as a natural object, and mankind, as a natural phenomenon, on the
environment of their life and thought.

Philosophy cannot sufficiently take this into account, as it proceeds from the
laws of the mind, which is, in one way or another, a final and self-sufficient
criterion for it (even in those cases, like religious and mystical
philosophies, in which the reach of the mind is, in fact, limited).

The contemporary scientist, proceeding from the recognition of the reality of
one's surroundings, of the world subject to one's investigation---nature, the
cosmos, or world reality,\footnote{
	I will talk about the reality of the cosmos, instead of that of nature,
	here and further.  The concept nature, if we take it in a historical
	aspect, is a complex concept.  It very often encompasses only the
	biosphere, and it is more convenient to use it with just this meaning,
	or even not to use it at all (\autoref{sec:6}).  This would correspond
	to the vast majority of the uses of this concept historically in
	natural science and in literature.  The concept `cosmos' can be,
	perhaps, more conveniently applied to only the part of reality
	encompassed by science, a philosophically pluralistic conception of
	reality is possible at that, where there would be no single criterion
	for the cosmos.
}---cannot adopt this point of view as a basis for scientific work.

[Thus,]\footnoteTransl{Interpolated for meaning in English.} because one
presently knows with scientific precision that man is \emph{not} located on a
structureless surface of the Earth, is \emph{not} located in direct contact
with cosmic space in a structureless nature, which is not lawfully connected
with him.  True, even the deeply penetrating contemporary naturalist often, out
of routine and under the influence of philosophy, forgets this, and does not
take it into account in his thought, and does not identify this.

Man and mankind are most closely connected, above all, with the living matter
inhabiting our planet, from which they cannot, in reality, be isolated by any
physical process.  That is possible only in thought.


\Section % 2
% We shall study the scientifically unambiguous 'living matter', and 'living
% natural body', rather than 'life'.
The concept of life and the living is clear to us in everyday life, and cannot
raise scientifically serious doubts in the actual manifestations of it, and in
natural objects corresponding to it---in natural bodies.  It was only in the
$20^{\mathrm{th}}$ century, [with the discovery of] filter-passing viruses,
that there appeared facts in science compelling us for the first time to ask
seriously---not philosophically, but scientifically---the question: Are we
dealing with a living natural body, or with a non-living natural body---an
inert one?

With viruses the doubt is cast by scientific observations, rather than
philosophical notions.  In this consists the great scientific significance of
their study.  That is presently on a right and firm path.  The doubt will be
resolved, and nothing, except a more precise notion of \emph{living organism},
would give, with this approach couldn't fail to give
\dots\footnoteEd{Incomplete sentence in the original.}

Along with this, however, we encounter another kind of doubt in \emph{science},
arising from philosophical and religious searches.  For example, phenomena
concerning the material-energetic environment of manifestations, which are
philosophically \emph{common} to both living and inert natural bodies, are
scientifically studied in the works of the Bose Institute in
Calcutta.\footnoteRus{Института Бозе в Калькутте}\footnoteEd{
	The Bose
	Institute\footnoteTransl{\url{http://www.boseinst.ernet.in/index.html}}
	in Calcutta was founded by the Indian scientist Acharya Jagadish
	Chandra Bose\footnoteRus{Бозе Джегдиш Чандра} (1858--1937) in 1917.
	The institute studied the problems of physics, biophysics, inorganic
	and organic chemistry, biochemistry, the physiology of plants,
	selection, microbiology, etc. [---Ed.]
} They are not characteristic of, but are weakly expressed in inert natural
bodies, are strongly manifested in living ones, but are common to both.

This area, if it exists in the form in which Bose tried to establish it, of
phenomena common to inert and living natural bodies, introduces nothing new in
the sharp distinction between them.  The distinction must manifest itself in
this area, as well, if only the latter's existence would be proven.

We must approach phenomena here, as well, not in the aspect in which Bose
approached them, not as phenomena of \emph{life}, but as phenomena of living
natural bodies, of \emph{living matter}.

To avoid any misunderstanding, I shall avoid the concepts `life', `living' in
all further exposition, since, if we stepped outside those
[phenomena],\footnoteTransl{inserted by transl. [---Pav]} we would inevitably
go beyond the limits of the phenomena of life studied in science, into a
foreign area or science---the area of philosophy, or, as is taking place in the
Bose Institute, into a new area of new material-energetic manifestations common
to all natural bodies of the biosphere, one lying outside the bounds of the
fundamental question of living organism, and living matter, which we are
presently interested in.

I shall, therefore, avoid the terms and concepts `life', and `living', and
limit the area which is subject to our investigation to the concepts
\emph{`living natural body'}, and \emph{`living matter'}.  Each living organism
\emph{in the biosphere}---natural object---is a living natural body.  \emph{The
living matter of the biosphere is the complex of living organisms in it.}

`Living matter', so defined, is a concept, completely precise, and fully
encompassing the objects studied by biology and biogeochemistry.  It is simple,
clear, and cannot raise any doubt.  We study only the living organism and its
complexes in science.  They are scientifically identical with the concept of
life.


\Section % 3
% Living matter is in a state of organization, which is determining for the
% biosphere.  Living matter's cosmic significance, and the constantly
% increasing extent of the biosphere.
Man, like every living natural (or naturally-occurring)\footnoteRus{природное
(или естественное)} body, is inseparably connected with a certain geological
envelope of our planet---\emph{the biosphere}, clearly distinct from the rest
of its envelopes, with a structure which is determined by its specific
\emph{state of organization},\footnoteTransl{организованностью, \ie\ 
organizedness, or being (constantly) organized. [---Pav]} and occupying a
lawfully expressible place in it as a distinct part of the whole.

Living matter, just like the biosphere, possesses its peculiar state of
organization, and can be viewed as a lawfully expressible \emph{function of the
biosphere.}

\emph{A state of organization} is not a mechanism.  It sharply differs from a
mechanism in that it is constantly in a state of
becoming,\footnoteTransl{становлении, \eg\ formation. [---Pav]} of motion of
all of its smallest material and energetic particles.  We can express this
state of organization in the course of time---in a generalization of mechanics,
and in a simplified model---as being such that none of its points (material or
energetic) lawfully returns to, ends up in a place, in a point of the biosphere
which it occupied at any earlier moment.  It can return to one of them only on
the order of mathematical accident, of very small probability.

The Earth's envelope, the biosphere, embracing the whole globe, has clearly
distinct dimensions, is determined to a large degree by the existence of living
matter in it---\emph{populating}\footnoteRus{заселена} it.  There is a
constant material and energetic exchange, materially expressed in the motion of
atoms brought on by living matter, between its inert non-living part, its inert
natural bodies, and the living matter inhabiting it.  This exchange in the
course of time is expressed as lawfully changing, constantly tending toward the
stability of an \emph{equilibrium}.\footnoteRus{равновесием}  The latter
permeates the whole biosphere, and this \emph{biogenic flow of
atoms}\footnoteRus{биогенный ток атомов} creates the biosphere to a large
extent.  The biosphere is thus connected inseparably and inherently with the
living matter populating it throughout the whole duration of geological time.

The planetary, cosmic significance of living matter is distinctly expressed in
this biogenic flow of atoms, and in the energy connected with it.  That, since
the biosphere is that single envelope of the Earth, in which cosmic energy,
cosmic radiation, mainly radiation from the Sun, which maintains the dynamic
equilibrium, the state of organization: $\mathrm{biosphere} \leftrightarrow
\mathrm{living matter}$, constantly penetrates.

The biosphere stretches from the surface of the geoid up to the boundary of the
stratosphere, penetrating in it; it, however, would be unlikely to be able to
reach the ionosphere---the Earth's electromagnetic vacuum, which is just now
entering the scientific consciousness.  Living matter reaches below the surface
of the geoid into the stratisphere, and into the top regions of the
metamorphic, and of the granitic envelope.  It rises up to
20--$25\,\mathrm{km}$ above the surface of the geoid, and extends down to
4--$5\,\mathrm{km}$ below that level on average.  These boundaries change in
the course of time, and, there are places of, it is true, small extent where
they are far beyond these.  Apparently, living matter must reach deeper than
$11\,\mathrm{km}$ at places in the depths of the ocean, and its presence has
been established deeper than $6\,\mathrm{km}$.\footnoteEd{
	Ocean floor organisms have indeed been observed at all depths of the
	world ocean, including at greater than $11\,\mathrm{km}$.  (See
	\foreignlanguage{russian}{\cites{belyaev1966donnaya}{belayev1989glubokovodnye}}.)
	[---Ed.]
}  We are just now living through the penetration of mankind, always
inseparable from other organisms---insects, plants, microbes,---into the
stratosphere, and by this means living matter has already exceeded
$40\,\mathrm{km}$ above the surface of the geoid, and is quiclky rising.

Apparently, a process of incessant expansion of the boundaries of the
biosphere: its population with living matter, is observable in the course of
geological time.


\Section % 4
\emph{The state of organization of the biosphere}---the state of organization
of living matter---must be viewed as an equilibrium, which is changeable,
always oscillating around a precisely expressible mean, not in historical, but
in geological time.  The shifts, or oscillations of this mean are constantly
manifested not in historical, but rather in geological time.  In the course of
geological time, in the cyclical processes which are characteristic of the
biogeochemical state of organization, no point (for example, atom or chemical
element) ever returns to a position identical with a previous one for eons.

This characteristic of the biosphere was expressed very prominently and vividly
by Leibnitz[1646--1716] in one of his philosophical reflections, it seems to
me, in the \rtitle{Theodicy}.\footnoteTransl{
	See \refsmartcite{leibnitz1985theodicy, leibnitz1900oeuvres-v2}.
}  He was among a large company from the high society in a large garden, he
recounts, at the end of the $18^\mathrm{th}$ c., and Leibniz, speaking of the
infinite variety of nature, and of the infinite perfectability of the mind's
precision,\footnoteRus{бесконечной четкости ума} indicated that two leaves of
any tree or plant are never completely identical.  All efforts of the large
company to find such leaves were, of course, in vain.  Leibniz was reflecting
here not as an observer of nature, discovering this phenomenon for the first
time, but as en erudite, taking it from his readings.  It is possible to trace
that precisely this example of the leaves appeared in philosophical folklore
centuries earlier.\footnote{
	See, for example, \cite[кн.~2,
	с.~54]{carus1913prirode}\nocite{carus1936prirode, carus1851nature}.}

This is manifested for us in everyday life in
\emph{identity},\footnoteRus{личности} in the absence of two identical
individuals, indistinguishable from one another.  It is manifested in biology
in the fact that every mean \emph{individuum}\footnoteRus{индивидуум} of living
matter is \emph{chemically distinct} in its chemical compounds, as, obviously,
also in its chemical elements having \emph{their own} specific compounds.


\Section % 5
% The heterogeneity of the biosphere.  The connection between living and inert
% matter.
Especially characteristic in the structure of \emph{the biosphere is its
physical-chemical, and geometric \emph{(\autoref{sec:47})} heterogeneity}.  It
consists of living and inert matter, which are sharply separate in their
genesis and structure throughout all of geological time.  Living organisms,
\ie\ all living matter, are born from living matter, form generations in the
course of time, which never arise directly, outside of such a living organism,
from any possible inert matter on the planet.  There is, however, an inherent,
unceasing connection\footnoteRus{непрерывная, никогда не прекращающаяся связь}
between inert and living matter, which can be expressed as an incessant
biogenic flow of atoms from the living matter into the inert matter of the
biosphere, and vice versa.  This biogenic flow of atoms is originated by living
matter.  It is expressed in its always unceasing breathing, feeding,
reproduction, etc.

This heterogeneity in the biosphere, unceasing during all geological time, is
the main dominant factor, strongly distinguishing it from all other envelopes
of the globe.

It goes deeper than the phenomena usually studied in natural science---to the
properties of space-time, which scientific thought has approached only in our
time, in the $20^\mathrm{th}$ c.

Living matter encompasses the whole biosphere, creates it, and changes it, but
amounts to a small part of it by mass, and by volume.  Inert, non-living matter
is strongly dominant; greatly diluted gases dominate by volume, hard rocks,
and, to a lesser degree, the liquid salt water of the world ocean---by mass.
Living matter, even in the greatest concentrations, in exceptional cases with
insignificant masses, amounts to tens of percent of the biosphere's matter, and
amounts to hardly one--two hundredths of a percent by mass on average.
Geologically, however, it is the greatest force in the biosphere, and
determines, as we can see, all processes occurring in it, and accumulates vast
free energy, creating the main geologically manifesting force in the biosphere,
whose power still cannot be quantitatively determined, but, possibly, exceeds
all other geological manifestations in the biosphere.

In connection with this, it is convenient to introduce a few basic concepts
which we will be dealing with in all of the following exposition.


\Section \label{sec:6} % 6
% The concepts of natural body (natural object) and natural phenomenon; living,
% inert, and bioinert bodies and phenomena; heterogeneity of the biosphere's
% structure.
Such are the concepts connected with the concepts of natural body (natural
object),\footnoteRus{природного тела (природного объекта)} and natural
phenomenon.\footnoteRus{природного явления}  They have often been referred to
as naturally-occurring bodies or phenomena.\footnoteRus{естественные тела или
явления}

Living matter is a natural body or phenomenon in the biosphere.  The concepts
\emph{natural body or natural phenomenon}, little logically studied, are main
concepts of natural science.  There is no need to delve into their logical
analysis for our purposes.  These are bodies or phenomena, formed by natural
processes,---\emph{natural objects}.

Living organisms, living matter, are not the only natural bodies of the
biosphere, but rather the main mass of the biosphere's matter forms non-living
bodies or phenomena, which I will be referring to as
\emph{inert}.\footnoteRus{косными}  Such are, for example, gases, the
atmosphere, rocks, a chemical element, an atom, quartz, serpentine, etc.

In addition to living and inert natural bodies, a great role in the biosphere
play its lawful structures, heterogeneous natural bodies, such as soils, silts,
surface waters, the very biosphere, etc., which consist of living and inert
natural bodies existing at the same time, forming complex, lawful inert-living
structures.\footnoteRus{косно-живые структуры}  I will be calling these complex
natural bodies \emph{bioinert} natural bodies.\footnoteRus{биокосными
природными телами}

The difference between living and inert natural bodies is so great, as we shall
see further on, that the transformation of one into the other is never and
nowhere observed in terrestrial processes; we encounter them nowhere and never
in scientific work.  As we shall see, such a process is deeper than the
physical-chemical phenomena known to us.

The related \emph{heterogeneity of the biosphere's structure}, the sharp
distinction between its matter and its energy in the form of living, and in the
form of inert natural bodies is its main manifestation.


\Section % 7
% The heterogeneity of time (historical and geological) in the biosphere.
One of the manifestations of this heterogeneity of the biosphere consist in the
fact that processes occur completely differently in living matter than in inert
matter, if they are viewed in the aspect of time.  They take place on the scale
of \emph{historical time}\footnoteRus{\emph{исторического времени}} in living
matter, on the scale of \emph{geological
time},\footnoteRus{\emph{геологического времени}} whose `second' is under
decamyriads, \ie\ a hundred thousand years of historical time,\footnote{
	On decamyriads see \cite{vernadsky1935nekotoryh}, [as well as
	\cite{vernadsky1954sochineniya-nekotorye}].
} in inert matter.  This difference is expressed even more sharply outside the
boundaries of the biosphere, and we observe in the litosphere, for the
predominant part of its matter, a state of organization in which the majority
of atoms, as radioactive studies show, are immobile, do not mix, noticably for
us, in the course of tens of thousands of decamyriads---a span of time now
accessible to our measurement.

Not long ago the view that geologists cannot study the manifestation of
geologically long changes, occurring in the age of mankind's existence,
dominated.  In the times of my youth we learned and thought that the change in
climate, orography, the emergence of new species of organisms are not, as a
general rule, detected in geological studies, are not \emph{current
phenomena}\footnoteRus{текущим явлением} for the geologist.  Now this
conceptual circumstance of the naturalist has sharply changed, and we can see
ever more, and more emphatically the geological forces around us.  This
coincided, hardly by accident, with the penetration of the conviction of the
geological significance of Homo sapiens in the scientific consciousness, with
the detection of a new state of the biosphere---the noosphere---and is one of
the forms of its manifestation.  This is, of course, connected, above all, with
the increase in precision in the natural scientific work and thought in the
domain of the biosphere, where living matter plays a major role.

The sharply distinct manifestation of living from inert in the aspect of time
in the biosphere, with all of its significance, is a special expression of a
far greater phenomenon, reflected in the biosphere at every step.


\Section \label{sec:8}% 8
% The plasticity of living matter.  The evolution of the biosphere.
The living matter of the biosphere sharply differs from its inert matter in two
main processes, which have an immense geological significance, and give the
biosphere a completely different shape, which does not exist in any other
envelope of the planet.  These two processes are manifested only against the
background of geological time.  They never cease, and never go backwards.

First, \emph{the power of the expression of living matter in the biosphere
grows} in the course of geological time, the significance of living matter in
the biosphere, and its influence on the inert matter of the biosphere
increases.  This process is little taken into account to this day.  I will have
to deal with it all the time further on.

Another process, known to all, and having imprinted a deepest impression on all
scientific thought of the $19^\mathrm{th}$ and $20^\mathrm{th}$ centuries since
the middle of the $19^\mathrm{th}$ century, has attracted far more attention,
and has been studied much more.  This is the process of \emph{the evolution of
species} in the course of geological time---the sharp change in the living
natural bodies themselves.

We observe a sharp change in the natural bodies themselves in the course of
geological time only in living matter.  Some organisms turn into others, die
out, as we say, or change fundamentally.

Living matter is \emph{plastic}, changes, adapts to changes in its environment,
but also, possibly, has its own evolutionary process, manifested in changes in
the course of geological time, independent of the changes in the environment.
This is, perhaps, indicated by the incessant growth, with intermissions, of the
central nervous system of animals in its significance in the biosphere, and in
the depth of the reflection of living matter in its surroundings,\footnote{
	That the evolution of nervous tissue is incessantly ongoing in the
	course of geological time has been indicated more than once, but, as
	far as I know, it has not been completely analyzed scientifically and
	philosophically.  As the question here is not about a hypothesis, and
	not about a theory, the fact of its evolution cannot be denied---there
	can only be objections to its explanation.  The recognition of Redi's
	principle limits the number of explanations.
} in the former's penetration into the latter, in the course of geological
time.

The plasticity of living matter is, obviously, a very complex phenomenon, as
there are organisms which do not change in their morphological and
physiological structure noticeably for us for hundreds of millions of years, up
to five hundred million and more, over myriads of generations.  These are the
so-called \emph{persistents}\footnoteRus{
	персистенты.  These seem to be popularly known as \emph{living fossils}
	today. [---Pav]
}\fncomma\footnoteEd{
	persistents\dots\ See \cite[269]{vernadsky1965himicheskoe}. [---Ed.]
}---a phenomenon in biology which has been, unfortunately, extremely little
studied.  Nevertheless, we observe in them, as a phenomenon common to living
matter, a \emph{plastic evolutionary} process, for which there is not even a
symptom in inert natural bodies.  In these latter we see the same minerals, the
same formation processes, the same rocks, and so forth \emph{now,} which were
there \emph{two billion years, and more, ago.}

The evolutionary process of living matter encompasses the whole biosphere
incessantly during all geological time and, in different ways, less strongly,
still affects its inert natural bodies.  With this we already can, and must
talk about \emph{the evolutionary process of the biosphere itself,} occurring
in the inert mass\footnoteRus{инертной массе} of its inert and living natural
bodies, changing visibly in the course of geological time.

The reflection of living matter in its surrounding environment changes sharply
due to the evolution of species, ongoing constantly, and never ceasing.  Thanks
to this process, evolution---change---is transferred to the natural bioinert
and biogenic bodies, which play a major role in the biosphere---to soils, to
surface and underground waters (in seas, lakes, rivers, etc.), to coal,
bitumen, limestones, organogenic ores, and so forth.  The soils and rivers of
the Devonian, for example, are different from the soils [and
rivers]\footnoteTransl{Interpolated from the implied meaning. [---Pav]} of the
Tertiery, and of our period.  This is an area of new phenomena, hardly taken
into account by scientific thought.  \emph{The evolution of species turns into
an evolution of the biosphere.}


\Section \label{sec:9}% 9
% Scientific thought is not an arbitrary process, and is not governed by the
% arbitrary will of man.
The evolutionary process has acquired, in addition, a special geological
significance thanks to the fact that it has created a new geological
force---the scientific thought of social mankind.\footnoteRus{научную мысль
социального человечества}

We are now fully living through its prominent entry in the geological history
of the planet.  The intensive growth of the influence of a single species of
living matter\footnoteRus{одного видового живого вещества}---civilized
mankind---on the biosphere is observable during the last millenia.  The
biosphere is transitioning into a new state---\emph{into the noosphere}---under
the influence of scientific thought and human labor.

Mankind is encopassing the whole planet, is distinguishing itself, is diverging
from other living organisms as a new, unprecedented geological force by a
lawful motion, stretching one--two million years, with an ever-increasing in
its manifestation rate.  An ever-growing set of inert natural bodies, \emph{new
for the biosphere,} and new, great natural phenomena are being created by this
means in the biosphere at a speed comparable to that of reproduction,
expressible by a geometric progression in the course of time.

The biosphere is changing drastically in front of our eyes.  And there can
harldy be any doubt that its transformation, manifested in this way, by
scientific thought through organized human labor is not an arbitrary
phenomenon,\footnoteRus{случайное явление, \ie\ also chance phenomenon, or
accidental phenomenon} depending on the will of man, but is a tempestuous
\emph{natural process,}\footnoteRus{стихийный \emph{природный процесс}} whose
roots lie deep, and had been prepared by an evolutionary process whose duration
is calculated in the hundreds of millions of years.

Man must understand, as only a scientific, but not a philosophical or a
religious conception of the world can encompass this, that \emph{he is not an
arbitrary, independent of its surroundings}---biosphere and
noosphere---freely-acting natural phenomenon.  He comprises an unavoidable
manifestation of a great natural process, lawfully stretching throughout the
course of, at least, two billion years.

In the present times, under the influence of the surrounding horrors of life,
along with an unprecedented flowering of scientific thought, it has become
necessary to hear of the approach of barbarity, of the breakdown of
civilization, of the self-annihilation of mankind.  These attitudes, and these
reasonings appear to me to be consequences of insufficiently deep penetration
into our surroundings.  Scientific thought not having entered everyday life
yet, we are still living under the strong influence of philosophical and
religious habits, not corresponding to the reality of contemporary knowledge,
which we still have not grown out of.

Scientific knowledge, being manifested as a geological force creating the
noosphere, cannot lead to results, which contradict that geological process,
whose creation it is.  This is not an arbitrary phenomenon---its roots are
extremely deep.


\Section % 10
% The related process of cephalization, and the intensification of evolution.
% The critical periods of geological history, which reflect the periods of
% evolutionary intensificaton.
This process is connected with the creation of the human brain.  It was
detected in the history of science in the form of an empirical generalization
by the profound American naturalist, great geologist, zoologist,
paleontologist, and mineralogist J.~D.\ Dana [1813--1895] in New Haven.  He
published his conclusion almost 80 years ago.  Strangely, this generalization
has not entered daily life to this day, has been almost forgotten, and has not
undergone the necessary development to this day.  I will return to this later.
Here I will note that Dana presented his empirical generalization in the
language of philosophy and theology, and it, it seems, was connected to
conceptions, scientifically unacceptable today.

Speaking in contemporary scientific language, Dana noted that a more and more
advanced---central nervous system---\emph{brain} than that which existed
earlier on our planet is manifested [in] some parts of its inhabitants in the
course of geological time.\nocite{dana1866classification}  This process, called
\emph{encephalization} by him, never goes backward, [even though] it ceases
many times, sometimes for many millions of years.  The process is, therefore,
expressed by a polar temporal vector, whose direction does not change.  We
shall see that the geometrical state of space\footnoteRus{геометрическое
состояние пространства} occupied by living matter is also characterized by
polar vectors, that there is no place for straight lines in it.

The evolution of the biosphere is connected with \emph{the intensification of
the evolutionary process} of living matter.

We now know that critical periods in the history of the terrestrial crust are
emerging, in which geological activity, in the most diverse of its
manifestations, is increasing in its rate.  This increase is, of course,
unnoticeable in historical time, and can be noted scientifically only on the
scale of geological time.

These periods can be considered \emph{critical} in the history of the planet,
and everything is indicating that they are occasioned by deep, from the
standpoint of the Earth's crust, processes, apparently exceeding its
boundaries.  A simultaneous increase in volcanic, orogenic, glacial phenomena,
marine transgressions, and other geological processes simultaneously
encompassing a great part of the biosphere throughout its whole extent has been
observed.\footnoteEd{
	More precise stratigraphic studies, done in various parts of our planet
	during the 45-year post-war period require us to modify our conception
	of `critical ages' in the history of the Earth somewhat.  Orogenic
	phenomena, as well as marine transgressions, turned out to have
	occurred at greatly different times on different continents, and even
	in different parts of the same massive continent.  [See
	\cite{yanshin1966tektonika, yanshin1973nazyvaemyh}.]  However, there
	undoubtedly were outbreaks of volcanic activity on the territories of
	contemporary continents in the history of the Earth.  Judging from the
	estimates produced by A.~B.\ Ronov\footnoteRus{А.~Б.\ Роновым} of the
	mass of volcanic products, they took place throughout the last 600
	million years in the middle Devonian, at the end of the
	Carboniferous--beginning of the Permian, at the end of the Triassic,
	and to a less significant degree in the middle Cretaceous, and in the
	Neogene.  Each such outbreak of volcanism led to a planetary change in
	the composition of the atmosphere---to an increase in its $CO_2$
	content, and to a decrease of the oxygen content, which brought, on the
	one hand, a decrease of temperature, leading to the formation of polar
	ice caps, and, on the other,---an intense development of vegetation,
	and the return of oxygen to the atmosphere, as a result of the
	processes of photosynthesis.  [See \cite{budyko1974klimat}] Apparently,
	``most important, and great changes in the structure of living matter''
	were created in these periods, \ie\ they were `critical' in the sense
	in which V.~I.\ Vernadsky used this word.  [---Ed.]
}  The evolutionary process coincides in its intensification, in its greatest
changes with these periods.  Most important and great changes in the structure
of living matter, which are a clear expression of the depth of the geological
significance of this plastic reflection of living matter in the resulting
changes of the planet, were created in these periods.

There is no theory, precise scientific explanation of this main phenomenon in
the history of the planet.  It emerged empirically, and
unconsciously---penetrated science unnoticed, and its history has remained
unwritten.  A major role in it played American geologists, specifically, J.~D.\ 
Dana.  It has pervaded the scientific thought of our century.

It is, however, possible, and necessary to approach it with measure and number.
The geological length of its duration can be measured, and, in this way, the
change in the rate of geological processes can be characterized numerically.
This is one of the immediate tasks of radiogeology.


\Section % 11
% The new evolutionary state of the biosphere, associated with the scientific
% thought of social mankind.
While this remains uncompleted, we must note, and take into account that the
process of evolution of the biosphere, its transition into \emph{the
noosphere}, clearly manifests an acceleration in the rate of geological
processes.  The changes, which are presently manifesting themselves in the
biosphere through the course of [the last] several \emph{thousand years} in
connection with the growth of scientific thought and the social activity of
mankind, have never existed in the history of the biosphere before.

Such, at the very least, are the conceptions which we can now derive from the
study of the course of evolution of organisms during geological time.
Decamyriads\footnoteRus{декамириада} are much less than a historical time's
second for geological time.  Consequently, a thousand years on a geological
scale would be more than 300 million years of geological time.  This does not
contradict [the existence of the periods of]\footnoteTransl{
	Interpolated from implied meaning.
} the great changes of the biosphere which took place, for example, in the
Cambrian, when calcareous skeletal parts emerged in microscopic marine
organisms, or [in] the Paleocene, when the fauna of mammals grew.\footnoteEd{
	Numerous findings of small mammals are now known from the deposits of
	different horizons of the Upper, and the top strata of the Lower
	Cretaceous, and the most ancient remains of primitive mammals have been
	observed already in Triassic deposits.  However, the intensive
	evolutionary development of this class of vertebrates began after the
	dying out of the dinosaurs in the Paleocene, by which the boundary
	between the Cretaceous and the Paleogene in the history of the Earth is
	determined to a large degree. [---Ed.]
}  We must not fail to keep in mind that the time we are living through
corresponds, geologically, to such a critical period, since the ice age has
% FIXME: Does Vernadsky mean an ice age cycle, rather than just an ice age by
% `ледниковый период'?
still not ended---the rate of change is, nevertheless, so slow that man could
not notice it.

Man and mankind, his kingdom in the biosphere lie completely in this period,
and do not exceed its boundaries.

A picture of the evolution of the biosphere since the Algonkian, and, more
sharply, since the Cambrian, over 500--800 million years can be given.  The
biosphere transitioned into a new evolutionary state\footnoteRus{эволюционное
состояние} more than once.  New geological manifestations, which had never
existed before, emerged.  This occurred, for example, in the Cambrian, when
large organisms with calcium skeletons came into existence, and in the Tertiary
(or, possibly, at the end of the Cretaceous), 15--80 million years ago, when
our forests and steppes were coming into existence, and the life of large
mammals developed.  We have also been living through this presently, for the
past 10--20 thousand years, when man, having developed scientific thought in a
social environment, has been creating a new geological force in the biosphere,
unprecedented in it.  The biosphere has transitioned into, or, more precisely,
is transitioning into \emph{a new evolutionary state---into the noosphere}---is
being transformed by the scientific thought of social mankind.


\Section % 12
% The irreversibility of evolution, an expression of the Curie-Pesteur
% discoveries principles of handedness and dissymmetry, and, actually, of
% Vernadsky's polar and enantiomorphic state of space.
The irreversibility of the evolutionary process is a manifestation of the
characteristic distinction of living matter in the geological history of the
planet from its inert naturally-occurring bodies and processes.  It can be seen
that this irreversibility is connected to the special properites of the space
occupied by the bodies of living organisms, to its special geometric structure,
as P.~Curie said, with its special \emph{state of
space}.\footnoteRus{\emph{состоянием пространства}}  L.~Pasteur first
understood the fundamental significance of this phenomenon, which he
inadequately called dissymmetry, in 1862.\footnote{
	The principle was formulated by P.~Curie (1859--1906), but was
	understood and expressed completely clearly and intuitively by
	L.~Pasteur (1822--1895).  I have delimited it here as a special
	principle (\cite{pasteur1922oeuvres, curie1908oeuvres}).
}  He studied this phenomenon in another aspect, in the inequality between
left and right phenomena in the organism, in the existence of left-handedness
and right-handedness\footnoteRus{правизны и левизны} for it.\footnote{
	It is striking that the phenomenon of `left-handedness' and
	`right-handedness' remained outside of philosophical and mathematical
	thought, even though individual great philosophers and mathematicians,
	like Kant and Gauss, approached it.  Pasteur was a complete innovator
	in thought, and it is extremely important that he arrived at this
	phenomenon, and to the recognition of its significance proceeding from
	experiment and observation.  Curie proceeded from Pasteur's ideas, but
	developed them from a physical standpoint.  On the significance of
	these ideas for life see \cite{vernadsky1940biogeohimicheskie} [A large
	part of them was published in the book
	\cite[22--271]{vernadsky1992trudy}]; \cite{vernadsky1934problemy1}
	[\cite{vernadsky1980problemy}].
}  Right-handedness and left-handedness can be manifested geometrically only in
a space, in which vectors are polar and enantiomorphic.  The lack of straight
lines, and the strongly expressed curvature of the forms of life is,
apparently, connected to this geometrical property.  I will return to this
question further on, but I consider it necessary to note presently that we are,
apparently, dealing with a space, not corresponding to Euclidean space, but to
one of the forms of Riemannian space, inside organisms.

We are presently justified to admit the manifestation of the geometrical
properties corresponding to all three forms of geometry---Euclidean,
Lobachevskian and Riemannian---in the space in which we are living.  Further
investigation will show whether such a conclusion, logically completely
uncontradictable, is correct.\footnote{
	Mathematical thought has admitted the equal permissibility of the
	search for the manifestations of non-Euclidean geometry in the reality
	surrounding us long ago.  Perhaps, the thought of this was clear to
	Euclid himself when he separated the parallel postulate from his
	axioms.  Lobachevsky (1793--1856) was striving to prove the existence
	of the triangles introduced by him, proceeding from the rejection of
	this postulate, for cosmic space.  It seems to me that H.~Poincaré
	(\cite[3, 66]{poincare1902science}) most prominently emphasized the
	possibility of searches for the manifestations of non-Euclidean
	geometry in our physical environment.  This question raised no doubt
	with the ferment of thought,\footnoteRus{брожении мысли} occasioned by
	A.~Einstein (\cite{einstein1921geometrie}).  It can be objected that in
	these cases it was admitted, as it were, \lphr{tacito consensu} (by a
	tacit agreement) that geometry, of this form or another, is the same in
	all reality, while, as in the given case, we are dealing with a
	geometrical heterogeneity of the space in our reality.  The space of
	life is different from the space of inert matter.  I cannot see any
	basis for presuming such an admission contradictory to the foundations
	of our exact knowledge.
}  Unfortunately, the great amount of empirical observations, relevant here and
scientifically established, has not been assimilated in its significance by
biologists, and has not entered into their scientific world outlook.
Meanwhile, as P.~Curie showed, such a special state of space cannot occur in
the usual space without special circumstances; a dissymetrical phenomenon,
speaking his language, must always result from such a dyssimetrical cause.  The
fundamental empirical generalization that living originates only from living,
and an organism is born from an organism corresponds to this.  This is
manifested geologically in the fact that we observe an impassable boundary
between living and inert naturally-occurring bodies and processes in the
biosphere, which is not observable in any other terrestrial envelope.  There
are two sharply materially [and] energetically distinct media, mutually
penetrating and exchanging their constituent atoms, connected with the biogenic
flow of chemical elements, in it.  I will return to this phenomenon in more
detail further on.


\Section % 13
% The current period of an extraordinary manifestation of living matter through
% scientific thought.
We are currently living through an extraordinary manifestation of living matter
in the biosphere, genetically connected with the emergence of \lphr{Homo
sapiens} thousands of years ago, the creation, in this way, of a new geological
force, \emph{scientific thought,} dramatically increasing the influence of
living matter on the evolution of the biosphere.  Completely encompassed by
living matter, the biosphere is increasing the geological force of living
matter to an, apparently, unlimited degree, and, being transformed by the
scientific thought of \lphr{Homo sapiens,} is transitioning into one of its new
states---\emph{into the noosphere.}

As a manifestation of living matter, scientific thought \emph{cannot be} in
essence a reversible phenomenon---it can stop in the course of its motion, but,
once created and manifested in the evolution of the biosphere, it carries in
itself the ability of unlimited development in the course of time.  The course
of scientific thought in this respect, for example in the creation of machines,
is, as has been remarked long ago, completely analogous to the course of the
reproduction of organisms.

There is no irreversibility in the inert medium of the biosphere.  Reversible
cyclical physico-chemical, and geochemical processes strongly predominate in
it.  Living matter enters in them with its physico-chemical manifestations of
dissonance.\footnoteEd{
	The Earth as a whole has an irreversible development, as well, as is
	shown by the work with radioactive determination of the age of the
	rocks of the early Precambrian.  Biological evolution is strongly
	distinguished by a different rate of development
	(\cite{yanshin1988evolyuciya}).  [---Ed]
}

The growth of scientific thought, closely connected with the growth of man's
population of the biosphere---his reproduction, and his cultures of living
matter in the biosphere,---must be limited by the foreign living matter of the
environment, and must exert a \emph{pressure} on it.  For this growth is
connected to the quantity of rapidly increasing living matter, directly and
indirectly participating in scientific work.

This growth and the pressure connected to it is ever increasing, due to the
fact that the activity of the mass of created machines, whose increase in the
noosphere is subject to the same laws as those of the reproduction of living
matter itself, \ie\ is expressed by a geometrical progression, is dramatically
manifested in this work.

As the reproduction of organisms is manifested in the \emph{pressure} of living
matter in the biosphere, the course of the geological manifestation of
scientific thought puts pressure, by the instruments created by it, on the
inert medium of the biosphere containing it, creating the noosphere, the
kingdom of reason.

The history of scientific thought, of scientific knowledge, of its historical
course is being manifested in a new aspect, which has not been sufficiently
recognized to this day.  It must never be viewed as simply the history of one
of the humanities.  This history is, at the same time, \emph{the history of the
creation a new geological force---scientific thought,---in the biosphere,} not
present in the biosphere before.  This is the history of manifestation of a new
geological factor, of a new expression of the biosphere's state of
ogranization, forming tempestuously, as a natural phenomenon during the last
few tens of thousands of years.  It is not arbitrary, as every natural
phenomenon, it is lawful, as the paleontological process, creating the brain of
\lphr{Homo sapiens} and that social environment, in which scientific thought, a
new geological, consciously directed, force is being created as a consequence
of this environment, as a natural phenomenon connected with it, is lawful in
the course of time.

But the history of scientific knowledge, even as a history of one of the
humanities, is still unrecognized and unwritten.  There is not a single attempt
to do that.  It has just begun to exceed the limits of `biblical' time for us
only in the recent years, the existence of \emph{a single center} of its
emergence, somewhere in the region of the future Mediterranean culture eighty
thousand years ago, has started to become clear.  We are beginning to detect,
to establish unexpected for us, completely forgotten scientific facts lived
through by mankind, only with great gaps from cultural remains, attempting to
encompass them by new empirical generalizations.\footnote{
	The rapid change in our knowledge thanks to archeological excavations
	allows us to hope for very great changes in the near future.
}

\Chapter{%
Manifestation of the historical moment mankind is currently living through as a
geological process.  Evolution of the species of living matter and evolution of
the biosphere into the noosphere.  This evolution cannot be stopped by the
course of global human history.  Scientific thought and mankind's daily lives
as expressions of it.}\label{ch:2}

\Section % 14
We are not yet conscious of, we are not yet living the realization of the
full consequences of the astonishing, unprecedented times that mankind has
entered during the 20th century.

We are living at the threshold of an extremely important, fundamentally new
epoch in the existence of mankind, in mankind's history on our planet.

Mankind has, for the first time, encompassed the whole surface envelope of the
planet---the whole biosphere, all parts of the planet connected to life---with
human life, with human culture.

We are present at, and are actively participating in the creation of a new
\emph{geological factor} in the biosphere, unprecedented in its power and in
its unity.

It has been scientifically established for the last 20--30 thousand years, but
has been clearly manifested at an ever increasing rate only during the last
millenium.

The envelopment of the whole surface of the biosphere by a unified social
species of the animal kingdom---by \emph{mankind}---has been completed after
many hundreds of thousands of years of unstoppable, tempestuous striving for
it.  There is no corner on Earth inaccessible to mankind.  There is no limit to
our possible population growth.  Man, through scientific thought and through
his life, socially organized into states, and guided by technology, is creating
a new \emph{biogenic force} in the biosphere, which is guiding his population
growth and creating favorable conditions for his population in parts of the
biosphere, earlier impenetrable to human life, and even in places where there
was no life before.

Theoretically, we cannot foresee a limit to mankind's potential, if we only
take into account the effect of generations; every geological factor is fully
manifested in the biosphere only in the effect of generations of living beings,
only in geological time.  With the rapidly increasing precision of scientific
work---in this case, of the methodology of scientific observation,---we can now
clearly establish, and study the increase of this new, principally currently
emerging, geological force in historical time.

Mankind is a unified whole, and even if that is recognized by the vast
majority, this unity manifests itself in forms of human life, which actually
deepen and strengthen it without being noticed by man, impetuously, [as a
result of] an unconscious striving for it.  Human life, with all of its
variety, has become indivisible, unified.  An event, ocurring in a forsaken
corner on land or in the ocean, is reflected, and has consequences, major or
minor, in a multitude of other places, all over the Earth.  The telegraph,
telephone, radio, airplanes, aerostats\footnoteTransl{An aerostat is an object
that can stay stationary in air, bacause it is lighter than it, such as a
baloon or a dirigible.} encompass the globe.  Communication is ever easier and
faster.  Its organization increases, turbulently grows, every year.

We can clearly see that this is the beginning of a tempestuous movement, of a
natural phenomenon, which cannot be stopped by the accidents of human history.
Here the relation between historical processes and the paleontological history
of the manifestation of Homo sapiens is expressed, maybe for the first time.
That process---\emph{the complete colonization of the biosphere} by
mankind---arises from the course of the history of scientific thought, which is
inseparably connected with the speed of communication, with the achievements of
transportation technology, with the ability of thoughts to be communicated
\emph{instantaneously}, and to be discussed everywhere on the planet
simultaneously.

The fight, which is being carried out against this main historical current, is
forcing even its ideological opponents to obey it.  Government formations,
ideologically rejecting the equality and unity of all people, are attempting,
lacking no resources, to halt its impetuous manifestation; but it can hardly be
doubted that these utopian dreams would fail to last.  This transformation will
inevitably come to pass in the course of time, sooner or later, since the
creation of the noosphere out of the biosphere is a natural phenomenon,
fundamentally deeper and more powerful than human history.  It necessitates the
manifestation of mankind as a unified whole.  This is its inevitable
requirement.

Ours is a new stage in the history of the planet, which does not allow
comparison with past history without corrections.  It is so, because this stage
is creating fundamental \emph{novelty} in the history of the whole Earth, and
not just in the history of mankind.

Man has actually recognized for the first time that he is a citizen of the
\emph{planet} and that he can---must---think and act in a new aspect, not only
in the aspect of individual personalities, nuclear or extended families,
nations or their unions, but also in a \emph{planetary aspect}.  He, like
everything living, can think and act in a planetary aspect only in the region
of life---in \emph{the biosphere}, in a certain earth envelope, with which he
is inseparably and lawfully connected, and outside of which he cannot go.  His
existence is a function of it.  He carries it everywhere with himself.  And he
inevitably changes it lawfully and unceasingly.


\Section % 15
Simultaneously with mankind's complete envelopment of the surface of the
biosphere---with its complete colonization,---which is closely connected
with the achievements of scientfic thought, i.e. with the course of scientific
thought in time, a scientific generalization, which scientifically reveals the
character of the historical moment mankind is currently living through in a new
way, has been formed in \emph{geology}.

Mankind's geological role has been cast anew in the understanding of
geologists.  True, the recognition of the geological significance of our social
life has been expressed in a less clear form long ago, much earlier in the
history of scientific thought.  However, at the beginning of our century C.
Schuchert [1858--1942] in New
Haven,\footcite[80]{schuchert1933geology} and A.~P.\ Pavlov (1854--1929) in
Moscow\foreignlanguage{russian}{\footcite[с.~105 и
сл.]{pavlov1936geologicheskaya}} independently accounted, geologically anew,
for the long-known change which the emergence of human civilization introduces
into the environment, onto the face of the Earth.  They considered it possible
to take this manifestation of Homo sapiens as the basis for distinguishing
\emph{a new geological epoch,} along with the tectonic and orogenic data which
usually determine such divisions.

They correctly tried to split the Pleistocene Epoch, defining its end by the
beginning of the manifestation of mankind (during the recent hundred-somethng
thousand years---say a few decamyriads ago), and separating the latter in its
own geological epoch: \emph{psychozoic,} according to Schuchert;
\emph{anthropogenic,} according to A.~P.\ Pavlov.

Actually Ch.\ Schuchert and A.~P.\ Pavlov deepened and made more precise,
brought into the established in modern geology divisions of the history of the
Earth, a conculsion, which was made much before them, and which did not
contradict the empirical scientific work.  This conclusion was clearly
recognized by one of the creators of contemporary geology, L. Agassiz
(1807--1873), based on the paleontological history of \emph{life}.  He
established the special geological \emph{epoch of mankind} already in 1851.

However, Agassiz relied not on geological facts, but rather, to a great extent,
on the common religious conviction so strong during the age of natural science
before Darwin; he started from the special position of man in the
universe.\footnote{Agassiz expressed that idea in a polemical work directed
against Darwinism (\fullcite{agassiz1859essay}).  It is possible that this is
related to why the work did not reach, [despite] the many important reflections
in it, the influence it could have had.}

The geology in the middle of the 19th century, and the geology at the beginning
of the 20th century are incomparable in their power and scientific
justification, and the epoch of mankind of Agassiz is not scientifically
comparable with the epoch of Schuchert-Pavlov.

Already earlier, when geology was just being created and its basic concepts did
not yet exist, G.\ Buffon (1707--1788) notably expressed that same geological
epoch of mankind at the end of the 18th century.  He proceeded from the ideas
of the philosophy of the Enlightenment, advancing the significance of reason in
the conception of the universe.

The definite difference between these homonymous concepts is clear from the
fact that Agassiz assumed the geological age of the World to be the biblical
duration of the existence of the Earth---six--seven thousand years,---Buffon
thought about an age of more that 127 thousand years, Schuchert and Pavlov---of
more than a billion years.


\Section % 16
We have already met with similar conceptions in philosophy long ago.
Conceptions, which have been reached in another way---not by way of precise
scientific observation and experimentation, like that of C. Schuchert, A. P.
Pavlov, L. Agassiz (and J. Dana, who knew about the generalizations of
Agassiz), but by way of philosophical searches and intuition.

The philosophical worldview creates, in general, as well as in particular, that
environment, in which scientific thought takes place and develops.  To a
significant extent, it determines and gives rise to scientific thought, itself
being changed by its achievements.

The philosophers relied on free, it seemed to them, in their expression ideas,
on the searches of confused human thought, of human consciousness, which
wouldn't reconcile with reality.  However, man unavoidably built his ideal
world in the brutal framework of surrounding nature, the environment of his
life, the biosphere, with which he has a deep connection, independent of his
will, which he did not, and still does not, understand.

We find, in the history of philosophy, already many centuries before our age,
intuitions and constructs, which could be connected to scientific empirical
conclusions, if we translate the thoughts---intuitions---that have reached us
into the realm of real scientific facts of our time.  We lose their roots in
the past.  A few of the philosophical searches in India, many centuries
ago,---the philosophy of the Upanishads---can be interpreted in such a way, if
we translate them into the realm of 20th century science.\footnote{The
philosophy of The East, mainly of India, in connection with the new creative
work there, taking place under the influence of the introduction of Western
science in Indian culture, is of much greater interest for life sciences than
Western philosophy, which is deeply permeated---even in its materialistic
parts---by deep echoes of Judeo-Christian religious searches.}

Analogous conceptions existed in another, smaller, cultural area, partly
overlapping, but later, which was isolated from the Indian one for a
significant part of the time: in the circle of the Helenic Mediterranean
civilization.  We can trace the germs of these conceptions going back almost
two and a half thousand years ago.  The significance of science and scientists
for the government of the polis in political and social thought is clearly
manifested in Helenic thought, and is notably expressed in the concept of the
sate, [given by] Plato [427--347].

It cannot, it seems, be denied, but the condition of the sources, reaching us
in fragments, also does not allow us to confirm precisely, that after Aristotle
[384--322] these ideas were still alive during the Helenic age of Alexander the
Great [356--323], when, a few centuries after the destruction of the Persian
kingdom, a close exchange of ideas and knowledge between Helenic and Indian
civilization was established.  A connection between them and Chaldean
scientific thought, which went back a few millenia before Helenic and Indian
thought, was established at the same time.  The history of scientific work and
thought during this remarkable age is just beginning to come to light.

Better known is the influence of the Helenic political and social ideas.  We
can trace their historical influence exactly in the historical process of
modern science and of the civilization of the European West, which replaced the
theocratic ideological structure of the Middle Ages.  We can see their growth
in pactice, and with clarity only during the 16th--17th centuries, in the
conceptions and constructs of F. Bacon (1561--1626), who prominently advanced
the idea of the power of man over nature as the aim of modern science.

In the 18th century, in 1780, G.\ Buffon posed the manifestation of man's
control of nature \emph{as part of the history of the planet} not as an idea,
but as an observable natural phenomenon.  He relied on the hypothetical
reconstruction of the planet's past, connected with philosophical intuition and
theory, rather than on precisely observed facts---but he was looking for them.
His ideas were adopted by philosophical and political thought, and,
undoubtedly, exerted their influence on the course of scientific thought.
Geologists from the end of the 18th--beginning of the 19th century often relied
on them in their current scientific work.


\Section % 17
The scientific constructs of Schuchert and Pavlov and all the scientific
work which---to a significant degree unconsciously---preceded them are
essentially distinct from these philosophical constructs, which, however (this
can be established historically), undoubtedly influence the course of
geological thought, though unable to give it a firm basis.

It is clear from the generalizations of Schuchert and Pavlov that the main
influence of human thought as a geological factor is expressed in its
scientific manifestation: it mainly builds and guides the technical work of
mankind, which is transforming the biosphere.

Both of the indicated geologists were able to make their generalizations,
above all, because mankind was able to colonize the whole planet in their
time.  No organism except him, save for microscopic species and, possibly, a
few graminoids, has encompassed such an area in populating the planet.
However, mankind has accomplished this in a different way.  He thought
scientifically and transformed the biosphere through labor, adapted it to
himself and himself created the conditions for the manifestation of his
characteristic biogeochemical energy of reproduction.  Such population of the
whole planet became clear at the beginning of the 20th century, and it could
be considered a fact since about the first quarter of that century, which is
being confirmed every year in front of our eyes.  It became possible only
thanks to the drastic change of the conditions of life connected with the
emergence of a new ideology, with the drastic change in the tasks of
government life, with the scientific growth of technology, which were being
carried out at the very same time.

As J. Ortega y Gasset\footnote{\fullcite{ortegaygasset1932revolt-p19}}
correctly remarked, the 19th century in Europe, and over the whole world since
its second half, was a historical period when the significance of the vital
interests of the masses of population occupied first place in practice and
ideology in their consciousness and in the consciousness of government people
for the first time in wold history.  It was dramatically manifested in everyday
life for the first time.  A new ideology was based on the consciousness of the
population masses stepping onto the historical stage as a social force for the
first time.  It is beginning to encompass all mankind---every language without
exception---at a rapidly increasing rate.

It will show in its real significance only in the course of time.

The social-political ideological shift was dramatically manifested in the 20th
century mainly thanks to scientific work, thanks to the scientific
determination and clarification of the social tasks of mankind, and of the
form of his organization.


\Section % 18
The question of the better organization of life and of the means by which
it could be accomplished has been raised numerous times during the
multi-thousand-year historical tragedy full of blood, suffering, crime,
destitution, hardship, which we call world history.  Man has not accepted the
conditions of his life.

The exit from these searches has been resolved differently, and we can see
numerous (and how many have disappeared without trace!)
searches---philosophical, religious, artistic and scientific.  For millenia
they have been, and are being created in every corner where human society has
existed.

The world history of mankind has been lived and recreated for a significant
part of the human population, and the places and times full of suffering,
evil, slaughter, hunger, and destitution for the majority have been an
unsolvable mystery from a \emph{human} point of view of sensibility and
goodness.  In general, innumerable philosophical and religious attempts during
the course of millenia have not reached a unified explanation.

All solutions reached in such a way transfer and have transferred the question
in a different plane---from the domain of brutal reality, into the domain of
ideal constructs. Various forms of countless religious-philosophcal solutions,
which are indeed related to the notion of individual immortality, in one or
another form, in the literal meaning of the word, or in its future resurrection
in new conditions, where evil, suffering and disasters would not exist, or
where these would be distributed justly, have been found.  The notion of
metempsychosis, solving the question not from a personal standpoint, but from
the standpoint of all living matter, is the deepest.  It, having emerged a few
millenia ago, is still alive and vivid for many hundreds of millions of people
to this day.  And there is, perhaps, nothing it contradicts contemporary
scientific notions in.  The course of scientifc thought has nowhere run up
against the conclusions from this notion.

All of these notions---with all of their distance, sometimes, from precise
scientific knowledge---are a powerful social factor over the course of
millenia, strongly reflected in the process of the evolution of the biosphere
into the noosphere, far from being, however, decisive, or somehow distinguished
from other factors in its creation at the same time.  In the course of tens of
thousands of years, they have, in this aspect, sometimes played the main role,
have sometimes disappeared among others, have moved into the background, could
have been left neglected.


\Section % 19
Because this same historical process of world history is reflected in the
nature surrounding man in another way[;]\footnoteTransl{%
	The source Russian version has a full stop here.%
} it is possible and necessary to approach it purely scientifically, leaving
aside any notions which do not result from scientific facts.

Archeologists, geologists, and biologists are now having such an approach to
the study of world history, leaving without consideration of any of the
millenia-old notions of philosophy and religion, not taking them into account,
creating a new scientific understanding of the historical process of man's
life.  Geologists, deepening the study of the history of our planet, of the
Pleistocene, of the Ice Age, have collected a vast amount of scientific facts,
manifesting the reflection of the life of human societies---in the end, of
civilized mankind---on the geological processes of our planet, in fact, of the
biosphere.  Without its evaluation from the standpoint of good and evil,
without regard for the ethical or philosophical aspect, scientific work,
scientific thought is establishing a new fact of primary geological
significance in the history of the planet.  This fact consists of the detection
of a new\emph{ Psychozoic }or\emph{ Anthropogenic geological Age, } created by
the historical process.  In fact, it is defined planetologically by the
emergence of mankind.

None of the countless---geological, philosophical, or religious---notions of the
significance of mankind, and the significance of human history play any
[significant] role in this scientific generalization.  They can be left aside
without any concerns.  Science does not have to take them into account.


\Section % 20
Approaching the analysis of this scientific generalization, we should note that
its duration can be estimated as millions of years, while the historical
process of human societies encompasses a few decamyriads, hundreds of thousands
of years, of it.

It is necessary, most of all, to stress a few preconditions, which determine
this generalization.

First, is \emph{the unity and equality, in essence, in principle, of all
people}, of all races.  This is expressed biologically in the detection of all
people in the geological process \emph{as a unified whole} with respect to the
rest of the living population of the planet.

And this is despite the possibility, and, even, probability of the emergence of
the different human races from different species of the genus \lphr{Homo}.
This difference likely does not reach deeper, to the various animal
predecessors of the genus \lphr{Homo}.  We cannot, however, deny it.  Such
unity with respect to all other life has been, in general, maintained
throughout all of world history, even though it was absent, or almost absent at
times, and in places in special cases.  We are encountering such manifestations
still today, but the general tempestuous process is not changed by this.

The geological significance of mankind was manifested for the first time in
connection with this.  Apparently, already hundreds of millenia ago, when man
acquired control over fire and began making the first instruments, he laid the
foundation of his advantage over the higher animals, the fight with which
occupied a major part of his history, and was, theoretically, finally ended a
few centuries ago with the discovery of firearms.  Man must take special care
in the 20\textsuperscript{th}$\;$c.\ not to allow the extinction of all
animals---large mammals and reptiles,---which he would like to preserve because
of some or other considerations.  Many tens of millenia earlier, however, close
to his emergence, he was that force, new on our planet, which occupied an
important place along with other earlier species, in bringing the extinction of
species of large animals.  It is quite possible that he did not differ much
from numerous other gregarious predators at that time.


\Section % 21
Much more important, from a geological point of view, was another shift, slowly
taking place tens of thousands of years ago---the domestication of herd animals
and the cultivation of cultured plant races.  Man started changind the living
world around him, and creating for himself a new, previously non-existent
living nature by this means.  The great significance of this was manifested in
another way---in the fact that he saved himself from hunger in a new way, known
to only a limited degree among animals,---the conscious, creative safeguard
against hunger---and, consequently, created the possibility for his unlimited
reproduction.

At that time, perhaps, ten--twenty thousand years ago, thanks to this
possibility, the possibility for the formation of large settlements (towns and
villages), and, consequently, the possibility for the formation of government
structures, completely essentially different from those special forms which
arise from blood relations, was first established.  The idea of the unity of
mankind received here in reality, although, obviously, unconsciously, even
greater possibilities for its development.

Thanks to the discovery of fire, man was able to survive the Ice Age---those
great changes and variations of the climate and the state of the biosphere,
which are now being scientifically uncovered before us in the alterations with
the so-called interglacial periods---at least, three in number---in the
Northern Hemisphere.  He survived them, even though numerous other lage mammals
disappeared then from the face of the Earth.  It is possible that he aided
their extinction.

The Ice Age has not ended, and extends to the present time.  We are living in
an interglacial period---the warming is still continuing,---but man has adapted
to these conditions so well that he does not notice the Ice Age.  The
Scandinavian Glacier thawed in the location of St.\ Petersburg and Moscow a few
thousand years ago when man had already developed domestication and
agriculture.\footnoteEd{
	The time of the maximum of the last glaciation is determined today to
	be 18--20 thousand years ago by the method of carbon dating.  It did
	not reach Moscow, but only the Valdai Hills; the ice cover thawed about
	10--12 thousand years ago in the outskirts of Leningrad.
	\parenNoteAuth{Ed.}
}

Hundreds of thousands of generations passed in the history of mankind during
the Ice Age.

However, we can hardly doubt today that man (probably, not the genus Homo)
existed already much earlier---at latest, at the end of the Pliocene, a few
million years aro.  The Piltdown Man in Southern England at the end of the
Pliocene, morphologically different from contemporary man, already possessed
stone implements, and, obviously, unpreserved implements out of wood, and,
possibly, bone.  His brain apparatus was as developed as in contemporary
man.\footnoteEd{
	The skull from the Piltdown cave, constructed from fragmentary remains
	in 1912 by Charles Dawson, was fabricated either by him, or by other
	irresponsible anthropologists.  It is a skull of an entirely
	contemporary person with jaws of a hominid ape.
	(\cite{howell1965early})
	\parenNoteAuth{Ed.}
}
The Sinanthropus of Northern China, living, apparently, at the beginning of the
Post-Pliocene in an area where the glacier, apparently, did not reach,
controlled fire and possessed implements.\footnoteEd{
	Sinanthropus lived 350--400 thousand years ago, \ie\ in the middle of
	the Pleistocene, somewhat later than V.\ I.\ Vernadsky thought.
	However, his supposition that the genus Homo existed already ``a few
	million years ago'', turned out to be correct.  The famous excavations
	of Dr.\ L.\ Leakey in the Olduvai Gorge on the border of Kenya and
	Tanzania, widely covered in scientific and popular science journals,
	showed that primitive man in Eastern Africa, classified as the peculiar
	species of Homo habilis (handyman), undoubtedly lived 1,800--1,900
	thousand years ago.  The later discoveries of R.\ Leakey on the eastern
	shore of Lake Rudolph led to the wide-spread oppinion that man lived
	already 3 million years ago in Eastern Africa, although that number is
	not credible, since the fragmentary remains of the skull were found in
	scree, and it is not known what layer they originate from.  The
	contemporary species Homo sapiens (wise man) emerged 40--45 thousand
	years ago not in Africa, but in the fairly northern latitudes of
	Europe and Asia, probably not without the influence of, and adaptation
	to the extreme conditionns of the Ice Age.  (See
	\cite{ivanova1965geologicheskiy}; also in German:
	\cite{ivanova1972geologische}.)
	\parenNoteAuth{Ed.}
}

It is possible that A.\ P.\ Pavlov was quite right when he supposed that the
Ice Age, the first glaciation of the Northern Hemisphere, began at the end of
the Pliocene, and at that time a new organism, possessing an exceptional
central nervous system, which led, in the end, to the development of cognition,
and is now being manifested in the transition of the \emph{biosphere into the
noosphere,} emerged in the conditions approaching the severe ones of
glaciation.

Apparently, all morphologically different types of man, the different genera
and species were already communicating with each other, were distinct from the
general mass of living matter from the beginning, possessed creative work of a
drastically different character than that of surrounding life, and could
interbreed with each other.  \emph{The unity of mankind was developing
tempestuously} in this way.  Apparently, Osborn\footnote{\cite{osborn1910age}}
was right that man on the border between the Pliocene and the Post-Pliocene,
still lacking permanent settlements, possessed great mobility, traveled from
place to place, was recognizing and manifesting his strong
distinctness---strove toward independence from his surroundings [environment].

\Section % 22

. . .

\Section % 23

. . .

\Section % 24

. . .

\Section % 25

. . .

\Section % 26

. . .

\Section % 27

. . .

\Section % 28

. . .

\Section % 29

. . .

\Section % 30

. . .

\Section % 31

. . .

\Section % 32

. . .

\Section % 33

. . .

\Section % 34
Science, therefore, is by no means a logical construct, a truth-seeking
apparatus.  Scientific truth can never be known by logic, but rather only by
living.  \emph{Action }is a characteristic of scientific thought.  Scientific
thought---scientific work---scientific knowledge occurs in the thick of life,
from which it is inseparable, and by its very existence gives rise to its own
active manifestations, which themselves are not only means of disseminating
scientific knowledge, but also create the countless forms of its detection,
give rise to countless major and minor sources of the growth of scientific
knowledge.

The human individual, even in the time of our state of organization of science,
is, thus, far from always the creator of scientific ideas and scientific
knowledge; the research scientist, living a life of purely scientific work, of
large or small extent, is \emph{one} of the creators of scientific knowledge.
Individual people, connected to scientifically important, but often foreign to
science, considerations, revealing scientific facts and scientific
generalizations, sometimes fundamental and decisive, hypotheses and theories
widely used in science, come forward accidentally, \ie\ by the means of
everyday life, out of the thick of life along with the scientist.

Such scientific work and scientific searches, proceeding from actions outside
the scientific, consciously organized work of mankind, is the active-scientific
manifestation of the living of the human cognitive environment at a given time,
a manifestation of life's scientific environment.  The part of the scientific
structure of the new scientific thought, introduced in science in this way, is
by its mass, and by its importance for the outcome of history comparable, it
seems to me, to what is introduced into science by the scientists consciously
working on it, to what is revealed by the consciously organized scientific
work.  Without the simultaneous existence of scientific organization and a
scientific environment, this ubiquitous form of mankind's scientific work,
tempestuously unconscious, disappears and is forgotten to a large extent, as
this occurred in the regions of Mediterranean civilization over the course of
long centuries in the Christianized Roman Empire, in Persian, Arabic, Berber,
Germanic, Slavic, and Celtic societies of Western Europe, in connection with
the national breakdown of the government formations existing in them during the
$4^\mathrm{th}$--$12^\mathrm{th}$ c.\ AD, and, often, later.  Science loses its
achievements in the course of time, and tempestuously comes back to them.

The history of science, and the history of mankind, reveals such events at
every step.  The flourishing of Hellenic science left aside, and did not make
use of, used late (after millennia) such achievements of everyday Chaldean
science, as, for example, Babylonian algebra.


\Section % 35
This means---the introduction of scientific discoveries, foreign to \emph{the
scientific searches of the individual personality}, to which life gives rise
everywhere, and their incorporation into the organized manifestation of the
scientific work of scientists, the scientific apparatus of a given
time,---however, is not the only means by which the living environment impacts
science.

This, in and of itself collective, \altStylePhr{and,} from a scientific point
of view, unconscious work,\footnote{%
	Uncoscious in the sense that the scientific result, or phenomenon of
	life, which creates the scientifically important or necessary fact (or
	generalization), did not have \emph{that goal} at its creation or
	manifestation.
} \emph{in the course of historical time} and through the
changes occurring in this way, creates the new and important, which can be
registered and can become the result of scientific achievements of primary
importance, as, for example, were the circumnavigation of the Earth, the
discovery of America, the fall of the Persian Kingdom (destroyed by Alexander
the Great), as well as the Chinese kingdoms and Central Asian cultural centers,
the defeat of Genghis Khan, the victory of the Christian church and religion,
the emergence of Mohammedanism and its religious-political identification, as
well as other major and minor events of political life.

No less, but, often, rather more powerful have been those changes, which have
occurred in economic life, in agriculture, or in individual manifestations of
success in everyday life, like, for example, the introduction of the camel
(dromedary) in the desert and semidesert areas of Northern Africa,\footnote{%
	\cite[p.~178]{julien1931histoire}.  See
	\cite{gsell1926memoires};\footnoteTransl{%
		According to Wikipedia[!] the Académie des Inscriptions et
		Belles-Lettres was founded by Jean-Baptiste Colbert, and Jean
		Sylvain Bailly was its member.
	} \cite[p.~181]{gautier1927siecles} for the significance of this
	phenomenon.
} and the discovery of printing in the Rhenish countries in Europe.\footnote{%
	We must never forget that the printing press was discovered in Korea a
	few centuries before Coster and Gutenberg, and was widely used in the
	Chinese kingdom.  There, however, the factor which gave it a living
	power did not exist: active scientific work was lacking in Korea and
	China at the time.
}

Along with these tempestuous phenomena, whose consequences for scientific
thought were not considered at their creation by mankind, to an equal, and
sometimes, perhaps, greater degree, scientific thought itself---the scientific
discoveries of individual thinkers and scientists, which change mankind's
world-view, like Copernicus, Newton, Linnaeus, Darwin, Pasteur, P.\ Curie---is
acting in the biosphere.  In some cases this was done consciously, in
others---unexpectedly for the scientist oneself, as occurred before our
eyes with A.\ Becquerel [1852--1908], discovering radioactivity in
1896,\footnote{%
	Becquerel himself thought that he took up Uranium only because it was
	studied by his father and grandfather (\autoref{sec:55}).
} or with H.\ Ørsted [1777--1851], detecting electromagnetism,\footnote{%
	Ørsted discovered electromagnetism in 1820.
	(\cite{oersted1920discovery}.)
} or with L.\ Galvani [1737--1798], discovering the galvanic
current.\footnote{%
	The phenomenon discovered by Galvani was correctly explained by Volta.
	Galvani's explanation was incorrect, but ``galvanism,'' with
	incalculable consequences before the study of electricity, was
	discovered by him.  (See \cite{alibert1801eloge} about him.)
}

Maxwell, Lavoisier, Ampere, Faraday, Darwin, Dokuchaev, Mendeleev and many
others encompassed great scientific revelations, worked creatively to bring
them into being in full consciousness of their fundamental significance for
life, but unexpected for their contemporaries.\footnote{%
	It is interesting that the significance of these discoveries in their
	application to life was admitted decades after the deaths of Maxwell,
	Lavoisier, Faraday, Mendeleev, \altStylePhr{and} Ampere.
}

Their thought---consciously for them---influenced the thick of life; here the
applied creations arising in this way, in a new form, unexpectedly and
unsurmisedly for their contemporaries, often after the deaths of their
creators, were reflected anew in scientific work, overturned mankind's everyday
life, \altStylePhr{and} created new, unexpected sources of scientific
knowledge.

Along with them, in the same way, through the thick of life, through the
environment, inventors, among them, often, people with little scientific
literacy---from all social classes and circles, often people having no
connection with or interest in the search for scientific truth,---are creating
a new, analogous cycle of scientific problems.\footnote{%
	R.\ Arkwright\dots\ [Arkwright, Richard (1732--1792)---English
	mechanic, inventor of the spinning frame. \noteAuth{Ed.}]; Zénobe
	Théophile Gramme\dots\ [Gramme (1826--1901)---Belgian electrical
	engineer, one of the inventors of the dynamo. \noteAuth{Ed.}]
}


\Section % 36
From everything said so far we can see that \emph{it is possible to make}
conclusions of great scientific significance, namely:
\begin{enumerate}
  \item The course of scientific work is that force by which man changes the
  	biosphere in which he lives.
  \item This manifestation of the biosphere's changing is an inevitable,
  	concomitant phenomenon to the growth of scientific thought.
  \item This change of the biosphere occurs independently of human will,
  	tempestuously, as a naturally-occurring phenomenon.
  \item And since the environment of life is an organized envelope of the
  	planet---the biosphere,---the introduction, in the course of its
	geologically long existence, of a new factor of change---the scientific
	work of mankind---in it is the natural process of the transition of the
	biosphere into a new phase, into a new state---into the noosphere.
  \item We can see this more clearly in the historical moment we are living
  	through than could be seen earlier.  ``Nature's law'' is being revealed
	before us now.  New sciences---geochemistry and biogeochemistry---are
	making the expression of a few important characteristics of the process
	mathematically possible.
\end{enumerate}


\Section % 37

. . .

\Section % 38

. . .

\Section % 39

. . .

\Section % 40

. . .

\Section % 41

. . .

\Section % 42

. . .

\Section % 43

. . .

\Section % 44

. . .

\Section % 45

. . .

\Section % 46

. . .

\Chapter{%
The movement of scientific thought in the 20th century, and its significance in
the geological history of the biosphere.  Its main characteristics: explosion
of scientific work, change in the understanding of the fundamentals of reality,
ecumenicism, and efficient, social manifestation of science.}


\Section \label{sec:47}
What is presently occurring in the scientific movement can only be compared
with that scientific movement from the past of science, which was connected
with the birth of Greek philosophy and science in the 6th--7th c.~BC.

Unfortunately, so far we cannot clearly imagine that accumulation of scientific
knowledge which the ancient Greeks had amassed at the time when scientific
thought manifested itself in their environment, and when it, for the first
time, acquired a scientific-philosophical structure, outside of religious,
cosmogonic and poetical constructs---when the scientific method was created for
the first time in the Hellenic city civilization of the polis---logic and
theoretical mathematics applied to life, when the search for scientific truth
became a reality, as a goal for itself in the life of the individual in a
social environment.

The circumstances of this, as history has shown, momentous event in mankind's
life, and in the evolution of the biosphere are, to a large extent, mysterious
and the history of scientific knowledge is being clarified slowly, but
nevertheless ever deeper.  Clear is only a general sketch of the accumulation
of scientific knowledge of the Hellenic environment at that time, the
achievements of the thinkers of Hellenic science, who lived at the time, and
what they received from the previous generations of Hellenic civilization.  We
are slowly beginning to understand this.  This is on the one hand.

And on the other hand, the conceptions about what the Greeks received from
great civilizations preceding them---Asia Minor, Cretan, Chaldean
(Messopotamian), Ancient Egypt, India---are now starting to drastically
change.

Unfortunately, only a \emph{miniscule part} of Hellenic scientific literature
has reached us.  The major researchers have left no trace in the literature
accessible to us, or only fragmentary indications of their scientific work has
reached us.

True, a large part of the complete works of Plato has reached us, as well as a
significant part of Aristotle's scientific works, however, many of the latter's
works, fundamental from the standpoint of the scientific search, have been
lost.  Especially unfortunate, from this standpoint, is the loss of the works
of major scientists, in whose output scientific thought and the scientific
method entered the age of flourishing and synthesis of Hellenic
science---Alcmaeon (500~BC), Leucippus (430~BC), Democritus (420--370~BC),
Hippocrates of Chios (450--430~BC), Philolaus (5th century BC) and many others,
from whom only miniscule fragments, or nothing but names have remained.

The loss of the first attempts at histories of scientific work and thought,
which were written closest to the centuries of its manifestation, may be even
more unfortunate.  Partly distorted, and in an incomplete form, this work has
reached us in the form of nameless essentials, sometimes adapted and skewed in
the course of the many centuries after their publication.  But the originals of
Xenocrates' (397--314) history of Geometry, Eudemus of Rhodes' (circa~320)
history of science, Theophrastus' (372--288) historical books, and others have
been lost in the historical course of Greko-Roman civilization by the time of
our age---during the centuries closest to it, almost a thousand years ago.

In essence, the basic fund of Helenic science---what I call a \emph{scientific
apparatus}\footnote{
	\foreignlanguage{russian}{\fullcite[9--10]{vernadsky1939problemy2}
	(Problems of Biogeochemistry II)}
}---has reached us in miniscule fragments, passing, on top of it, through many
centuries, in the remains of Aristotle's and Theophrastus's works on the
history of natural sciences, as well as in the works of Greek mathematicians.
Nevertheless, it exerted tremendous influence on the Renaissance and on the
creation of Western European science in the 15th--17th centuries.  Our modern
science has been created, to a significant extent, relying on and starting from
this fund's achievements, developing the ideas and knowledge laid out in it.
Broken for centuries, that already during the time of the Roman Empire, the
threads were restored in the 17th century.


\Section % 48
The recent course of the history of science requires us to change
our conceptions of that pre-Hellenic heritage, from which Hellenic science
sprouted, as I already indicated (§42).

The Greeks have everywhere pointed to the great knowledge, which they had
received from Egypt, Chaldea, the East.  We must now admit that they were
correct.  Science had already existed before them---the science of the
``Chaldeans'', reaching back beyond millenia BC, is only now being uncovered
before us---in fragments, proving beyond any doubt its long unsuspected, until
our time, force (§42).

It is now becoming clear that we must attribute a much more real significance,
than has been recently done, to the numerous indications by ancient scientists
and writers of the fact that the creators of Hellenic science and philosophy
took into consideration, proceeded in their creative work from the achievements
of scientists and thinkers from Egypt, Chaldea, Arian and non-Arian
civilizations of the East.

Babylonian scientists worked together with Greek ones in the course of several
centuries.  At the same time, the new flourishing of Babylonian astronomy
occurred in the centuries closest to our age.  Gradually, in the course of
several generations, they merged into the Hellenic cultural environment and
equally suffered the unfavorable for science circumstances of that time (§40).
Undoubtedly, the knowledge received from the scientists of that time was used
by the Greeks during the period of this dialogue.

Undoubtedly, what was harnessed and used by them was very significant by that
time---especially if we consider the multimillenial experience and the
multimillenial tradition of seafaring, engineering, agriculture, irrigation
works, military art, government organization and everyday life.

For centuries Greek science worked in direct contact with Chaldean and Egyptian
science, was merging with them.  Though it is possible that creative thought in
Egyptian science died out during that time---this didn't happen with Chaldean
science (§42).

Hellenic science, in the age of its birth, is a direct continuation of the
intense creative thought of pre-Hellenic science.  This fact is acknowledged,
but still not assimilated, in the history of science.

The ``miracle'' of Hellenic civilization---a historical process, whose results
are clear, but whose course cannot be precisely traced---was a historical
process like others.  It had a solid basis in the past.  Only its result in its
achievement---the rate at which it was achieved---turned out to be singular in
time, and exceptional in its consequences in the noosphere.


\Section % 49

. . .

\part{On Scientific Truths}

%\Chapter{}
\refstepcounter{chapter}

\include{chapter-05}
\part{New Scientific Knowledge and the Transition of the Biosphere into the
Noosphere}

%\Chapter{}
\refstepcounter{chapter}

\Chapter{%
The structure of scientific knowledge as a manifestation of the noosphere, the
geologically new state of the biosphere resulting from this knowledge.  The
historical course of the planetary manifestation of Homo sapiens by means of
its creation of a new form of cultural biogeochemical energy, and the noosphere
associated with it.}

100. The sciences of the biosphere and its objects, i.e. all humanities
without exception, natural sciences, in the term's own meaning, (botany,
zoology, geology, mineralogy, etc.), all engineering sciences---applied
sciences in the general meaning of the term---are areas of knowledge, which are
maximally accessible to mankind's scientific thought.  Here millions of
millions of incessantly scientifically established and systematized facts,
which are the results of organized scientific work, are concentrated, and are
unstoppably increasing, quickly and consciously, with every generation,
beginning with the 15th--17th centuries.

In particular, the scientific disciplines of the constitution of means of
scientific knowledge, inseparable from the biosphere, can be viewed
scientifically as a geological factor, as a manifestation of the biosphere's
organization.  These are sciences ``of the spiritual'' work of the human
individual in one's social environment, sciences of the brain and organs of
sense, the problems of psychology and logic.  They give rise to the search for
the fundamental laws of human scientific knowledge, that power which has, in
our geological age, transformed the biosphere encompassed by mankind into a
natural body, new in its geological and biological processes---into a new
state, into the noosphere,\footnote{
	\foreignlanguage{french}{\fullcite{leroy1928origines-p37-57}}
} to whose consideration I shall return below.\footnote{
	See \foreignlanguage{russian}{\cite[Гл.~21]{vernadsky1987himicheskoe}}}

Its emergence in the history of the planet, beginning intensively (on the
scale of historical time) a few tens of thousands of years ago, is an event of
great importance in the history of our planet, connected, in the first place,
with the growth of sciences about the biosphere, and is, obviously, not
accidental.\footnote{
	I will return to this process later.  Here I only note Le Roy's thought
	(1928): \foreignquote{french}{Deux grands faits, devant l'esquels tous
	les autres samblent presque svanouir, dominent dans l'histoire passe de
	la Terre: la vitalisation de la matire, puis l'hominisation de la
	vie.}---Op. cit., p.47.  \enquote{Two major facts, in comparison to
	which all others seem almost unnoticeable, predominate in the history
	of the Earth: the vitalization of matter, and the humanization of life.
	The first one is hypothetical, but the beginning of the second is
	clearly visible.}}

We can say that, in this manner, the biosphere is the main area of scientific
knowledge, even if we are only now beginning to differentiate it scientifically
from our surrounding reality.


101. It is clear from what has been said, that the biosphere corresponds to
that, which in the thought of naturalists and in most of philosophical
thought, in the cases where they were not concerned with the Cosmos as a
whole but remained within the limits of the Earth, corresponds to Nature as
usually understood, the Nature of the naturalist in particular.

However, this nature is not amorphous and shapeless, as it has been considered
for centuries, but has definite, very precisely delineated
structure,\footnote{
	This ``structure'' is very peculiar.  It is not a mechanism or anything
	motionless.  It is dynamic, always variable, moving, changing at every
	moment, and never returning to a previous type of equilibrium.  It is
	closest to a living organism, differing, however, from it in the
	physical-geometrical state of its space.  The space of the biosphere is
	physically-geometrically inhomogeneous.  I think that it is convenient
	to define this structure by means of a special concept of organization.
	See \fullcite{vernadsky1934problemy1, vernadsky1980problemy}.
} which must, as such,
be reflected, and considered in all conclusions and results concerning Nature.

It is especially important in scientific research that this is not forgotten
and that it is taken into account, since unconsciously, opposing the human
individual to Nature, the scientist and thinker gives in to the greatness of
Nature above the human individual.

But life in all of its manifestations, the manifestation of the human
individual included, radically changes the biosphere in such a degree that not
only the agglomeration of indivisible units of life, but, in a few problems,
also the single human individual in the noosphere could not be left without
attention in the biosphere.


102. Living nature\footnoteTransl{
	A literal translation of the Russian expression for the living part of
	nature.
} is a main characteristic of the manifestation of the biosphere, it is the
very distinction of the biosphere from the other earth envelopes.  The
structure of the biosphere is characterized, first of all, and most of all, by
life.

We shall see further on (§135) that between the physical-geometrical
properties of living organisms---they are manifested in the form of their
agglomerations in the biosphere---living matter, and those properties of inert
matter, which constitutes the dominant part of the biosphere by weight and by
number of atoms, there is in several respects an impassible gulf.  Living
matter is a carrier and creator of free energy absent from any other earth
envelope on such a scale.  This free energy---biogeochemical
energy\footnote{
	The concept of biogeochemical energy was introduced by me in 1925 in a
	still-unpublished report to the R.~Rosenthal fund in Paris. (The fund
	does not exist any more.)  This fund gave me the ability to work
	without interruption for two years.  The concept has been presented by
	me in print in numerous articles and books:
	\begin{itemize}
	  \item \foreignlanguage{russian}{\cite[30--48]{vernadsky1926biosfera}};
	  \item \foreignlanguage{french}{\cite{vernadsky1926etudes1,
		  vernadsky1927etudes2}};
	  \item \foreignlanguage{russian}{\cite{vernadsky1926razmnozhenii1},
		  \cite{vernadsky1926razmnozhenii2}};
	  \item \foreignlanguage{french}{\cite{vernadsky1926multiplication1,
		  vernadsky1926multiplication2}};
	  \item \foreignlanguage{russian}{\cite{vernadsky1927bakteriofag}}.
	\end{itemize}
	[Ed.:] For the R.~Rosethal fund's report
	\foreignlanguage{russian}{\rtitle{Живое вещество в биосфере}} see:
	\foreignlanguage{russian}{\cite[555--602]{vernadsky1994zhivoe}}
}---encompasses the whole biosphere and generally determines all of its
history.  It gives rise to and sharply changes the intensity of the migration
of chemical elements constituting the biosphere, and determines their
geological significance.

A new form of this energy, even greater in its intensity and complexity, has
been created and has been quickly increasing in its significance in the domain
of living matter during the last ten thousand years.  This new form of energy,
connected with the activity of human societies, of the genus Homo and others
(Hominidae) close to it, preserves the manifestation of the usual
biogeochemical energy, but at the same time gives rise to a new kind of
migration of chemical elements, leaving, in its variety and power, the usual
biogeochemical energy of living matter on the planet far behind.

This new form of biogeochemical energy, which can be called energy of human
culture, or cultural biogeochemical energy, is the form of biogeochemical
energy, which is presently creating the noosphere.  Later on I shall return to
a more detailed presentation of our knowledge of the noosphere and its
analysis.  But it is now necessary to sketch its manifestation on the planet.

This form of biogeochemical energy is characteristic not only of Homo sapiens,
but also of all other living
organisms.\footnote{
	\foreignlanguage{russian}{\cite[30--48]{vernadsky1926biosfera}}.  See
	\foreignlanguage{russian}{\cites[330--341]{vernadsky1994zhivoe}{vernadsky1926razmnozhenii1}{vernadsky1926razmnozhenii2}}.
	Published under the title \foreignlanguage{russian}{\rtitle{О
	размножении организмов и его значении в строении биосферы}} in the book
	\foreignlanguage{russian}{\cite[75--101]{vernadsky1992trudy}}.
}  It is, however, negligible in them in comparison to the usual biogeochemical
energy, and has a hardly noticeable effect on the balance of nature, and that
only in geological time.  It is connected to the psychological activity of
organisms, to the development of the brain in the highly developed
manifestations of life, and is expressed in a form resulting in the
transformation of the biosphere into a noosphere only with the emergence of the
human mind.

Its manifestation in mankind's predecessors has been produced, apparently,
over hundreds of millions of years, but it could be expressed in the form of a
geological force only in our time, when Homo sapiens has encompassed with our
life and cultural work the whole biosphere.


103. The biogeochemical energy of living matter is determined, above all, by
the reproduction of organisms, and by their inevitable tendency, determined by
the energetics of the planet, toward a minimum of free energy---it is
determined by the fundamental laws of thermodynamics, corresponding to the
existence and stability of the planet.

It is expressed in the respiration and feeding of organisms---``laws of
nature'', which have not been discovered in their mathematical expression to
this day, but the task of searching for whose expression was clearly laid out
already in 1782 by C. Wolf at the St. Petersburg Academy of
Sciences\footnoteRus{Петербургской Академии наук} at the
time.\foreignlanguage{russian}{\footcite[50]{vernadsky1954sochineniya}}

Obviously, this biogeochemical energy, in this form, is characteristic of Homo
sapiens, as well.  It is, as with all other living organisms, a species
characteristic,\footnote{
	On the species charactestic see \cite{vernadsky1930considerations}.
} and seems unchangeable to us in the course of historical time.  The other
form, of ``cultural'', biogeochemical energy is also unchanging, or hardly
changing for other organisms.  This other form is expressed in the everyday and
in the technical conditions of organisms' life---in their movement, in their
daily activity and construction of dwellings, in the transportation of their
surrounding matter, etc.  It, as I have already indicated, comprises a
negligible fraction of their biogeochemical energy.

With mankind, this form of biogeochemical energy, associated with the human
mind, grows and increases in the course of time, quickly taking first place.
This growth is possibly related to the growth of the mind itself---apparently,
a very slow process (if it, in fact, occurs at all)---but mainly---with the
increase of the precision and depth of its use, associated with the conscious
change of the social setting, and, particularly, with the growth of scientific
knowledge.

I shall proceed from the fact that the skeletons of Homo sapiens, including
the skull, over a hundred millenia gives us no basis for viewing them as
belonging to another species of man.  This is admissible only under the
condition that the brain of Paleolithic man does not differ in any significant
degree in its structure from the brain of contemporary man.  At the same time,
there is no doubt that the mind of that man from the Paleolithic for this
species of Homo cannot bear comparison to the mind of contemporary man.
Thence it follows that the mind is a complex social structure, built, for the
man of our times, just as for the Paleolithic man, upon the same nervous
substrate, but in a different social setting, which is being composed through
time (space-time, in essence).

Its change is the basic element, leading, in the end, to the transformation of
the biosphere into a noosphere in the obvious manner, above all---through the
creation and growth of the scientific understanding of our surroundings.


104. The emergence of cultural biogeochemical energy on our planet is a major
factor in its geological history.  This had been prepared for through all
geological time.  The main, decisive process here is the maximum manifestation
of the human mind.  But this is, in essence, inseparable from all
biogeochemical energy of living matter.

The life of the migration of atoms in the living process connects in a unified
whole all migrations of atoms of the biosphere's inert matter.

Organisms are alive only while the material and energetic exchange between
them and their surrounding biosphere is uninterrupted.\footnote{
	The complete absence of exchange for the latent forms of life cannot be
	considered proven, yet.  It is extremely slow---and, possibly, in a few
	cases there is no migration of atoms indeed---it could become
	noticeable only in geological time.
}  Colossal definite chemical cyclical processes of atomic migration, in which
living organisms enter as a lawful, inseparable, often main part of the
process, are being clarified in the biosphere.  These processes are constant in
geological time and, for example, the migration of magnesium atoms incorporated
in chlorophyll stretches uninterruptedly for at least two billion years through
innumerable, genetically related generations of green organisms.  Living
organisms, uninterruptedly and inseparably connected to the biosphere by such
atomic migrations, comprise a lawful part of its structure.

This must never be forgotten in the scientific study of life and in scientific
statements about any of its manifestations in Nature.  We cannot overlook the
fact that an uninterrupted connection---material and energetic of the living
organism with the biosphere, a completely definite connection, ``geologically
eternal'', which can be scientifically expressed precisely---is always present
in our every scientific approach to life and must be reflected in all of our
logical conclusions and results about it.

In moving to the study of the geochemistry of the biosphere we must, first of
all, precisely estimate the logical significance of this connection,
unavoidably entering all of our constructs related to life.  It does not
depend on our will, and cannot be excluded from our experiments and
observations, but must always be taken into account as something fundamental,
inherent in life.

The biosphere must, in this manner, be reflected in all of our scientific
statements without exception.  It must be manifest in every scientific
experiment and scientific observation---and in every thought of the human
individual, in every speculation, from which the human individual---even
thought---cannot escape.

Therefore, the human mind can be maximally expressed only with the maximum
development of the basic form of the biogeochemical energy of mankind, i.e.
with its maximum reproduction.


105. The potential for covering the surface of the whole planet by means of
reproduction of an organism of a single species is characteristic of all
organisms, since the law for reproduction is expressed in the same form for
all of them, in the form of a geometrical progression.  I have already
indicated the major significance of this phenomenon long ago,\footnote{
	See \cites[37--38]{vernadsky1926biosfera}[335, 413--424]{vernadsky1994zhivoe}{vernadsky1926etudes1}[59--83]{vernadsky1940biogeohimicheskie}[75--101]{vernadsky1992trudy}.
} and I will return to it at the appropriate place in this book.

The phenomenon of covering the whole surface of the planet by a given single
species can be seen widely developed for aquatic life in the microscopic
plankton of lakes and rivers, and for a few forms of---essentially also
aquatic---microbes, from the surface layers of the planet, propagating through
the troposphere.  Among larger organisms we observe this in almost full
measure in a few plants.

This has begun to be manifest for mankind in our times.  The whole globe and
all the seas have been encompassed by him in the 20th century.  Thanks to the
success of communications, man can be in constant communication with the whole
world, cannot be solitary and get himself lost in the grandiosity of the
earth's nature anywhere.

Presently, the number of the human population on Earth has reached
unprecedented height, nearing two billion people, despite the fact that murder
in the form of war, hunger, malnourishment, constantly affecting hundreds of
millions of people, extremely diminishes the course of the process.
Negligible time from the geological point of view would be necessary, hardly
more than a few hundred years, to end these relics of barbarism.  This could
be freely done even now; the ability to end this condition is presently in the
hands of mankind, and the reasonable will will inevitably go down that path,
because it corresponds to the natural tendency of the geological process.  It
should be so all the more, since the means to do it are increasing rapidly and
almost tempestuously.  The real significance of population masses, suffering
the most from this, is irrepressibly increasing.

The number of people inhabiting the planet began increasing, say, about 15--20
thousand years ago when mankind became less influenced by food shortage in
relation to the discovery of agriculture.  Apparently it was then, say, about
10--8 thousand years ago that the first population explosion
occurred.\footnote{\cite{childe1937man-p78-79}} G.~F.\ Nikolai (in
1918--1919)\footnote{\cite{nikolai1919biologie-p54}.} attempted to estimate the
actual population increase of mankind and the development of agriculture
numerically, the actual population of the planet by mankind.  According to his
calculations, taking the total territory of the Earth, there are 11.4 people
per square kilometer, which constitutes $2.10^{-4}\%$ of the possible
population.  Considering the amount of energy received from the Sun,
agriculture allows 150 people to be sustained per $1\,{\mathrm{km}}^2$, i.e.\
for the whole Earth (land area) it must be $22.5\cdot 10^9$ units, i.e. 22--24
times more than live presently.\footnote{\cite{nikolai1919biologie-p60}.}  But
mankind acquires energy for sustenance and for living not only through
agricultural labor.  Considering this possibility, Nikolai, for example,
estimated that the Earth in the historical age started in our time, using new
energy sources, could be populated by three hexillion people ($3\cdot
10^{16}$), i.e. more than tens of millions of times more than the present
number of mankind.  These numbers must be highly increased at the present
moment, when more than 20 years have passed since Nikolai's calculations, since
mankind can, in practice, presently use sources of energy, which Nikolai could
not imagine in 1917--1919---energy, connected to the atomic nucleus.  Must now
say, more simply, that the source of energy, which is encompassed by the human
mind in the energetic age of mankind, which we are entering---is practically
unlimited.  Hence, it is clear that the cultural biogeochemical energy (§17)
shares the same characteristic.  According to Nikolai's calculations, machines
increased mankind's energy more than ten times in his time.  We cannot
presently give a more precise calculation; however, recent accounts of the
American Geological Committee\footnoteRus{американского Геологического
комитета} indicate that water power, presently in use all around the world,
reached 60 million horsepowers at the end of 1936: it increased by 160 per cent
in 16 years, mainly in North America.\footnote{\fullcite{blair1938water}.}
Thanks to that, we must already increase Nikolai's calculations more than one
and a half times.

In essence, all of these calculations about the future, expressed in a
numerical form, have no significance, since our knowledge of the energy
accessible to mankind is, we can say, rudimentary.  Of course, the energy
accessible to mankind is not an infinite amount, since it is determined by the
size of the biosphere.  The limit to the cultural biogeochemical energy is
also determined by this.

We shall see (§138) that there is also a limit to the basic biogeochemical
energy of mankind---the speed of expansion of life, the limit of mankind's
reproduction.

The speed of reproduction\footnote{
	On the speed of expansion of life see below.  See
	\foreignlanguage{french}{\cite{vernadsky1926etudes1}};
	\foreignlanguage{russian}{\cite[413--424, 437--444]{vernadsky1994zhivoe}};
	\foreignlanguage{russian}{\cite[118--125]{vernadsky1940biogeohimicheskie}};
	\foreignlanguage{russian}{\cite[Гл.~20]{vernadsky1965himicheskoe}.}
}---the magnitude $V$ considered, in essence, by Nikolai, is based on the
actually observed population of the planet by mankind in unfavorable for his
life conditions.  We shall also see, further on, that there are still unknown
to us phenomena in the biosphere, which lead to a stationary maximum quantity
of living units per hectare which can exist in a given geological age in a
given condition of the biocenosis.


106. We can record the human population on the planet with any precision only
since the beginning of the 19th century.  It is still calculated with a high
percentage of possible error.  Our knowledge has considerably increased during
the last 137 years, but can still not be considered having reached the
precision which contemporary science may require.  For earlier times the
numbers are only provisional.  Still, they are helping us in the understanding
of the occurring process.

The following data may have significance for us in that aspect.

The number of people in the Paleolithic likely reached a few million.  It is
possible that it began with one family.  However, the opposite view is also
possible.\footnote{See E.~Le Roy. [The author's note has not been found.
---Ed.]}

In the Neolithic we are likely dealing with tens of millions on the whole
surface of the Earth.  It is possible that even in historical time it did not
reach a hundred million, or that it did not exceed that number by
much.\footnote{
	\cite{weinberg1922dvuhdesyatitysyachiletiyu-p21} (assumes a
	population of 80 million at the beginning of our age).}

G.~F.\ Nikolai supposed that the human population of the planet increases by 12
million people annually for 1919, i.e. increases by, say, 30 thousand a day.
According to the critical report of the Kulischers
(1932)\footnote{\cite{kulischer1932kriegs-p135}.} the world population was 850
million in 1800 (A.~Fischer takes it to be 775 million).  We can assume its
number for the white race to be 30 million in 1000, 210 million in 1800, 645
million in 1915.  For the whole population in 1900, according to the
Kulischers---about 1,700 million, but according to A.  Hettner
(1929)\footnote{\cite{hettner1929gang-p196}}---1,564 million, and 1,856 million
in 1925, according to the same.

That number has evidently reached about two billion, more or less, at present.
The population of our country (about 160 million) comprises about 8\% of the
world population.  The world population is rapidly increasing, and, evidently,
the percentage of our population is increasing, since its growth is greater
than the average population growth.  In general, we should expect to
significantly exceed 2 billion by the end of the century.


107. The reproduction of organisms, i.e. the manifestation of biogeochemical
energy of the first type, without which there is no life, is inseparable from
man.  However, at his very differentiation from the mass of life on the
planet, man had already mastered the use of tools, even if they were very
primitive, which allowed him to increase his muscle power, and were the first
manifestation of contemporary machines, which distinguished him from the other
living organisms.  The energy by which they were powered, however, was
produced through the feeding and breathing of man's very organism.  It has
probably been hundreds of thousands of years already since man---genus
Homo,---and his predecessors mastered the use of wooden, bone, and stone
tools.  The skill of making and using those tools was being developed slowly,
in the course of many generations, skill---the mind in its first
manifestation---was being perfected.

Such tools can be observed already in the most ancient Paleolithic, 250
thousand--500 thousand years ago.

A significant part of the biosphere was living through critical times during
that period.  Apparently, a radical change---in its water and heat
regime---began already in the Pliocene, an ice age began and was developing
throughout the whole period.  We are, apparently, still living during the
dying out of its last manifestation, whether temporary or permanent is still
unknown.  We can see strong oscillations in the climate during these half a
million years; relatively warm periods---continuing for tens and hundreds of
thousands of years---replaced in the northern and southern hemispheres
periods, during which masses of ice which reached depth of up to a kilometer,
for example, in the vicinity of Moscow, moved slowly---on the historical
scale.  They disappeared a thousand and seven\footnoteTransl{
	The other English translation has seven thousand here, and notes ``Now
	we know that in the environs of Leningrad the ice has disappeared about
	12 thousand years ago.''
} years ago in the Leningrad region, and are still occupying Greenland and
Antarctica.  Apparently, Homo sapiens, or his closest predecessors, formed not
long before the onset of the ice age, or during one of its warm periods.  Man
survived the coldness during that time with hardship.  That was possible thanks
to a great discovery in the Paleolithic---the mastery of fire.

This discovery was made in one--two, possibly a few more, places and slowly
spread among the population of the Earth.  Apparently, we have a general
process of great discoveries here, where not the mass activity of mankind,
smoothing out and amending particularities, but rather the manifestation of
the separate human individuality plays a role.  We can trace that in the more
recent time and in very many cases, as we shall see later (§134).

The discovery of fire is the first case of a living organism mastering and
harnessing a force of nature.\footnote{\cite{childe1937man-p56}. Cp.:
	\cite{frazer1930myths}.}

This discovery is the foundation, as we shall now see, of all the following
increase of mankind, and of our present power.

This increase, however, took place extremely slowly, and it is hard for us to
imagine the conditions, under which it could occur.  Fire was already known to
the ancestors of the genus, or to the predecessors of that species of Hominid,
who is building the noosphere.  The latest discovery in China reveals the
cultural remains of Sinanthropus, which indicate his wide use of fire,
apparently, long before the last glaciation of Europe, a hundred thousand
years before our time.  We presently have no data of any credibility about how
that discovery was made by him.  Sinanthropus already possessed a mind, had
primitive tools, used speech, performed burial rites.  This was already a
human, but foreign to us in many morphological characteristics.  Also, the
possibility that he is one of the predecessors of the contemporary human
population of China has not been eliminated.\footnote{
	On Sinanthropus's technology, and on his use of fire see
	\cite{bogaevsky1936tehnika-p26-27}.  Pithecanthropus, who lived
	earlier, at the very beginning of the Pleistocene, hardly more than 550
	thousand years ago, also possessed fire.  Ср.:
	\cite{bogaevsky1936tehnika-p11.67}.  The use of fire by Pithecanthropus
	cannot be considered proven, yet, but is very likely.}


108. The discovery of fire is all the more remarkable because the
manifestation of fire and light emission in the biosphere had been a
relatively rare phenomenon before mankind, and had manifested mainly when
taking up a large space, in the form of cold light, in such forms as airglow,
aurora borealis, sheet lightning, stars and planets, noctilucent clouds.  The
Sun alone, the source of life, was simultaneously a bright manifestation of
light and heat, was lighting and heating the planet.

Living organisms had developed a manifestation of cold light long ago.  It
appeared in such large-scale phenomena as marine bioluminescence, usually
taking up hundreds of thousands of square kilometers, or the luminescence in
marine depths, whose significance is just beginning to be clarified.  Fire,
accompanied by high temperature, was manifested in local phenomena, rarely
taking up large spaces like volcanic eruptions.

But these colossal on the human scale phenomena, obviously, because of their
destructive force, could in no way have aided the discovery of fire.  Man had
to look for it in closer to him, and less scary and dangerous manifestations
of nature than volcanic eruptions, still exceeding mankind in their
manifestation of power.  We are only beginning to approach using them in
practice, in conditions which were inaccessible and unthinkable to Paleolithic
man.\footnote{
	Mankind has obtained superheated vapor at a ${140}^\circ C$ temperature
	as a source of power only in the 20th century with the aid of drilling
	in Larderello under Le Conte's initiative.  Still later, this method
	was greatly developed in Soffioni, in New Mexico, in Sonoma.  Parsons,
	before his death, worked on an implementable project to obtain an
	unlimited, from mankind's point of view, source of energy from the
	inner heat of the earth's crust with the aid of deep drilling.  The
	attempt to obtain energy from the cold depths of the ocean, which the
	French Academician Claude did not realize only because of criminal
	hooliganism in 1936, can be considered analogous.  Undoubtedly, we have
	in these phenomena a practically inexhaustible force in mankind's
	hands.}

He had to look for phenomena giving heat and fire in his surrounding everyday
phenomena of life; in his habitat---in the woods, steppes, among living
nature, with which he was in close (long forgotten by us) connection.  Here he
could encounter fire and heat in a safe form in numerous everyday phenomena.
These were, on the one hand, fires, the burning of living and dead matter.
They were the very sources of fire used by Paleolithic man.

He burned wood, plants, bones, that which produced fire around him without his
will.  This fire was due to two very different reasons before man's emergence.
On the one hand, lightning caused forest fires, or set dry grass on fire.
Mankind still suffers from fires caused this way.  The natural conditions in
the ice age, especially in interglacial ages, could have been even more
favorable for lightning phenomena.  There was, however, another cause which
produced fire independently of mankind.

That was the biological activity of lower organisms, which lead to fires in
dry steppes,\footnote{
	The spontaneous ignition of dry grass in the steppes, in pampas, in
	forests has sometimes been denied.  Presently the source of fires is
	almost always man, but there are cases which, it seems to me,
	undoubtedly indicate the possibility of spontaneous ignition in steppes
	under the direct action of the sun.  The cause remains unclear.  About
	such cases see \cite{popping1835reise-p398}.
	\cite{carpenter1920naturalist-p76-77}.
} to the burning of bituminous coal layers, to the burning of peat bogs, which
continued throughout a number of human generations and gave a convenient way of
obtaining fire.  We have direct indications of such bituminous coal fires in
Altai, in the Kuznetsk basin, where they occurred in the Pliocene and
post-Pliocene, but where they also occurred in historical time, and where we
still have to deal with them.  The causes of these fires are still not
completely clear, but all indications are that it is unlikely that we have
phenomena of purely chemical spontaneous combustion, i.e.  intensive oxidation
of coal fragments with oxygen from the atmosphere, or its spontaneous ignition
due to heat released during oxidation of sulphur compounds of iron in the
coal.\footnote{
	See \cite{usov1924sostav-p58, usov1933podzemnye-p34,
		obruchev1934podzemnye-p83-85}.
	J.~F.\ Hermann\footnoteRus{И.~Ф.\ Герман}, who discovered
	the Kuznetsk bituminous coal basin, already indicated these phenomena
	in 1796.  See \cite{hermann1793notice}.  Cp.
	\cite{jaworsky1933erdbrande, yavorski1932kamennougolnye}.}

The most probable source is the biochemical phenomena associated with the
biological activity of thermophilic bacteria.  We have the direct observations
of B. L. Isachenko\footnoteRus{Б. Л. Исаченко} and N. I.
Malchevskaya\footnoteRus{Н. И.  Мальчевская}\footnote{
	See \foreignlanguage{russian}{\cite{isachenko1936biogennoe}}.
} for peat bogs in recent times.

This phenomenon presently requires careful study.


109. Such regions of warm winter and summer, as well as places of outlets of
heat sources, were precious gifts of nature to Paleolithic man, who had to use
them just as they are used, or were used until recently by tribes and peoples
that we still find in a living Paleolithic stage.

Man at that time, with his great attentiveness and closeness to nature,
undoubtedly noticed such places, and must have been using them, especially in
glacial periods.

It is curious that we can observe the use of the same biochemical processes
among the instincts of animals.  This can be observed in the family of the
chickens, with the so-called incubator birds, or large-foots (Megapodiidae) of
Oceania and Australia, which make use of the heat of biological decay, i.e. of
a bacterial process, for the hatching of chicks form eggs, creating large
mounds of sand or dirt mixed with strongly rotting organic
remains.\footnote{
	See \foreignlanguage{russian}{\cite{brem1912zhizn-ptitsy}}.
}  These mounds can reach 4 meters in height, and the temperature in them
reaches no less than ${44}^\circ C$.  Apparently, these are the only birds
possessing such instincts.

It is possible that ants and termites increase the temperature of their
dwellings on purpose.

However, these are weak attempts, incomparable to that planetary revolution,
which mankind has produced.

Man has been using the products of life---dry plants---as a source of energy,
fire.  Numerous myths about its creation have been preserved and
created.\footnote{See \cite{frazer1930myths}.}  But most characteristic is the
fact that man used, for that purpose, methods which he hardly ever observed to
produce fire in the biosphere until his discovery.  The most ancient methods
were, apparently, the transformation of man's muscle power into heat (strong
friction of dry objects), and the making and catching of sparks from stones.  A
complex system for the preservation of fire was developed in the end in
everyday life a hundred, and more, thousand years ago.

The surface of the planet has been changed radically after this discovery.
Fireplaces shone, were extinguished and started everywhere, if only man lived
there.  Mankind was able, thanks to this, to survive the coldness of the
glacial period.

Man was producing fire among living nature, subjecting it to burning.  In this
way, by means of steppe and forest fires, he acquired a force which, in
comparison to that of his surrounding animal and plant world, put him above
the numerous other organisms and became a prototype of his future. Mankind has
mastered other sources of light and heat---electrical energy---only in our
time, in the 19th--20th centuries.  The planet started shining even more, and
we have found ourselves at the beginning of times, whose significance and
future still remain outside of our attention.


110. 


. . .



\part{The Sciences about Life in the System of Scientific Knowledge}

%\Chapter{}
\refstepcounter{chapter}

\include{chapter-09}
\Chapter{%
The biological sciences must come, together with the physical and the chemical
sciences, among the sciences encompassing the noosphere.}

151. But the contemporary state of biology and its excursions into philosophy
are also detrimental to philosophy.

The expectant attitude of the naturalist for the confirmation of philosophy
creates among philosophers the impression that precisely the
scientists\footnoteTransl{
	probably a typo, and should be: ``exact scientists''
	\parenNoteAuth{Pav}
}, proceeding from their data, accept the basic tenets of the philosophical
current of materialism about the lack of fundamental difference between living
and inert.  Vitalistic notions have remained so far in the past in the general
course of biological thought that their real significance hardly influences
large-scale work.  The overbearing majority of naturalists are far from them.

The philosophers-naturalists, whose significance in contemporary philosophical
thought, in its global scope, is minute, receive [from the exact scientists]
what seems like firm ground, and calm their doubts.  This impacts their
creative work, which slowly dies down, and degenerates into dry, formal
scholasticism, or into verbal talmudism, especially in such cases as our
country, where dialectical materialism is the state philosophy, and is
favoured by the mighty support of government power, and by intellectual and
practical impossibility of its free criticism and of the free development of
any other philosophical views.

However, official dialectical materialism itself, being one of the many forms
of this current of philosophical thought, does not possess such freedom,
either.  And has been, meanwhile, never systematically philosophically worked
out to the end, remaining full of unclarity and unthoughtfulness.  Its official
exposition has changed more than once during the past twenty years, previous
ones were declared heretical, and new ones were created.  Our philosophers of
strict discipline, in which they work, have been obliged to obey without
objection, under the threat of persecution and material hardship, these new
ones, and to publicly repudiate their previous teachings, admitting their
mistakes.  It is easy to imagine what result follows, and how fruitfully can
one work intellectually in such a severe real environment.  As a result, a
condition very reminiscent of the condition of the orthodox church under
despotism has arisen, with the gradual downfall of lively work, work in this
area of philosophy, the exit into safe areas of knowledge, the publication of
classics, forebears; a new degeneration of thought has arisen.


152. It seems to me that for these 20 years, except the republication of old
works, which were released in the pre-revolutionary period, not a single
independent, purely philosophical work has been published, and there are not
even histories, based on primary sources, of the creation of dialectical
materialism itself.\footnoteEd{This part of the phrase is crossed out by the
author in the manuscript.}  Such decline of philosophical thought in the area
of dialectical materialism in our country, and the seemingly extensive
possibilities of its manifestation, are a consequence of the adopted
understanding of the goals of philosophy, and of the decrease of deep
philosophical work, thanks to the belief among our philosophers that a
philosophical truth, which cannot be changed and subjected to doubt any
further, has been reached.

Such an idea is, essentially, foreign to both K. Marx and F. Engels, not to
mention Feuerbach.

It was developed on Russian soil in the middle of emigration, and grew into a
state ideological influence completely unconsciously, its consequences being
unexpected for many very prominent freely thinking communists, as well.

The fight of the intellectual circles turned, in the end, imperceptibly and
unsuspectedly, into a state philosophy of the winning interpretation of
dialectical materialism.

Thanks to the strengthening of one definite current, this has been manifested
more and more clearly during the past 10 years.

As a result, we see, or we have, instead, a mass of literature of a transient
character, rooting out conscious or unconscious errors and heresies, deviations
from the officially accepted state philosophy.  On top of that, the state
philosophy itself has changed in very important nuances in the accepted
interpretation of dialectical materialism.  Such a sad state of work in our
country in the area of dialectical materialism at the presence of huge material
resources, which had never existed for any other philosophy (except for
theological ones---Catholic and Muslim philosophies in the Middle Ages), would
unavoidably come in another way, as well, thanks to many peculiarities in the
structure of state philosophy in our country.  On the one hand, thanks to the
emigration of intellectual circles, whose significance was already indicated;
and, on the other, thanks to the complexity, independent of life in our
country, of the environment, in which dialectical materialism was being
created.


153. Dialectical materialism, in the form in which it is actually manifested in
the history of thought, was never presented coherently by its authors---Marx,
Engels, and Ulyanov-Lenin.  These were prominent thinkers, and no less
prominent political activists.  Characteristic of them are a large breadth of
scientific knowledge and scientific interests, unusual for political activists.
They stood at the level of their time, but at the same time were volitional
personalities, organizers of the popular masses.  They were actively opposed
to, and regarded strongly negatively religious searches, judging them,
historically, as a force hostile, in the end, to the interests of the popular
masses and to the freedom of scientific work.  However, they, at the same time,
attributed great significance to philosophical thought, whose primacy over
scientific thought did not raise any doubt to them.

Their philosophical ideology was most closely related to their political
activity, and left an imprint on their scientific searches and understanding.
They were primarily philosophers, spokesmen for aspirations, and
organizers of the actions of the popular masses, whose social well-being---on a
real planetary basis---was the goal and meaning of their lives.  We see, by the
example of these people, a real, great impact of the personality not only on
the course human history, but, through it, on the noosphere, as well.

Part of the polemical works which their authors---Marx, Engels, Lenin,
Stalin---never intended for such a task were laid in the foundation of the
Soviet state philosophy; their statements on practical and political questions
of life, in which philosophy sometimes occupied a secondary place.  Such were,
secondly, draft notebooks, extracted from the manuscripts remaining after their
deaths, often reports and overview summaries related to the reading of
philosophers, which were never historically, scientifically, critically
published.  They were published by the scientific apparatus and with the
obeisance of believing students, and, as always under such circumstances, are
full of contradictions, and, in some cases, such as the Engels's
\rtitle{Dialectics of Nature}, the authorship of all of Engels's statements
cannot be considered proven.  A few works of Marx, and, partly, Engels, have a
different character, but they are completely insufficient for the firm
establishment of a new philosophy.  Marx' and Engels' life work was in another
domain.  Marx was a prominent scientist, who in the \rtitle{Kapital} reached
his conclusions by an exact scientific pathway, but presented them in the
language of Hegelian philosophy, independently reworked by him and Engels,
which already during their lifetimes did not (in general) correspond to current
scientific methodology and scientific searches.  The prominent mind could
permit itself such a peculiar form of presentation.

Already during Marx's lifetime---at the publication of the last volumes of his
\rtitle{Das Kapital}---such a presentation was an obvious anachronism, and it
is an even greater one in our time.  In essence, of course, what is important
is not the form of presentation of the scientific work, but rather the actual
methodology, by which what is presented has been reached.  The form of Marx's
presentation misleads the reader into thinking that what is presented was
reached by a philosophical pathway.  It is, in reality, only presented that
way, but was, in fact, reached by the exact scientific method of the historian
and economist-thinker, who Marx was in his scientific work.

It turned into a complete anachronism, since it was transferred from the area
of political economy and history into the area of natural and exact sciences.
This transfer, which can be observed in the works of both Marx and Engels,
acquired an extremely peculiar character with their epigons, having become the
state philosophy of a large and strong nation, most closely related to the
International.

Thirdly, the situation was worsened by the fact that the authors of these
philosophical searches were people, either actually exercising dictatorial
power in an unprecedented depth and degree, and considering the philosophical
ideology of dialectical materialism as the basis of their political and
practical activity, or people, such as Marx and Engels, who are not subject to
free criticism in our country for the same reason.  Their conclusions are, in
fact, accepted as impeccable dogma, defended by the full mechanism of
government power.

The stagnation of philosophical thought here, and its transformation into
fruitless scholasticism and talmudism, opulently blooming against that
background, is a direct consequence of this state of affairs.

This, in essence, great historical phenomenon was prepared in our country by
deeply-rooted submissiveness---unchanged during all the transformations of the
form of government---to the state religion.  The official Orthodoxy in the
Tsardom of Russia, as well as in the Russian Empire, prepared the ground for
the official philosophy, which replaced it, and which has acquired the clear
form of official religion with all of the consequences from that.

154. This, however, is, historically and in essence, only the everyday side of
the matter.  The ideology and its associated belief at its foundation are far
more important.

Dialectical materialism, in sharp contrast to contemporary forms of philosophy,
is extremely distant from philosophical scepticism.  It is convinced that a
universal method rules---an infallible criterion of philosophical and
scientific truth.  This is the effect of the temperament of its founders Marx
and Engels, who succeeded, thanks to their joining the still alive at that time
Hegelian philosophy, to impart to their scientific achievements the vibrantly
active form of faith, and not only of a philosophical doctrine---to create a
political force, able to move the masses and vividly manifest itself in the
\rtitle{Communist Manifesto} of '48---in a brilliant and profound work,
reflecting the age of the middle of the last century, when the primacy of
philosophy over science dominated ideologically Euro-American civilization.

In contrast to other forms of materialism, with which it is in fundamental
disagreement, dialectical materialism is closely related in its genesis and in
the basis of its formulations with idealism in its Hegelian form.

It is far from clear, whether it is possible to regard it as free from the
influence of such history, and to attribute it completely to the philosophical
current of materialism.

As far as I know, this question is historiographically unresolved, and in the
manifestation which materialism has in our country, its idealistic basis is
strongly emphasized, whereas its materialistic one is an outer appearance.

But this is a debatable area, far from my interests, and from my knowledge, and
I would not concern myself with it, if the sharp distinction between the
philosophical current of materialism and dialectical materialism did not become
completely clear in our country in the aspect which most concerns the
naturalist and seriously affects scientific work in our country.

Materialistic philosophy was evidently distinct---and that is where its force
lied---from the other philosophical currents of modern times, in the fact that
it did not conflict with science, was completely based on its achievements, as
far as possible.  It generalized and developed them.  In essence, it continued
that great movement, which developed in the 17th--18th centuries on the basis
of the new science, the new philosophy, and the new ways of everyday life and
technologies, which were created at that time.

Materialism, in essence, was striving to become a scientific philosophy, or a
philosophy of science.  It did not succeed in practice, since in its logical
conclusions, being part of the philosophy of the Enlightenment from the end of
the 18th century, when it clearly occupied a place on the historical stage for
the first time, it quickly fell behind the science of the times.

But in the aspect concerned in this book, what is important is not the success,
or failure of materialism in its historical manifestation during the age of its
flourishing at the end of the 18th century, and in the 1860s, but the
foundation of its ideology, which always recognized the primacy of science
above philosophy.  It considered everything proven by science as obligatory for
itself.

The dialectical materialism, created by Marx and Engels, did not accept that,
and, in that, sharply distinguished itself from all forms of philosophical
materialism, and, from that standpoint, did not differ at all from idealistic
Hegelianism.

For that very reason, it is also clearly distinct from philosophical
scepticism, which accepts the realistic worldview, as it is manifested
scientifically, as the only possibility, and does not recognize, in comparison,
either religious, or philosophical views on an equal basis.  Philosophical
scepticism, in contrast to philosophical materialism, does not recognize the
scientific view of reality as its complete view, taking into account the
increase of scientific knowledge, and the imperfections of human reason.  But
for it the scientific achievements at a given historical moment, and at a given
form of the human brain have the character of the most precise achievement of
reality.  Dialectical materialism does not proceed from scientific data, is not
limited to their boundaries, is not based on them, but is striving to change
and develop them, adapting them to its views, which have as a basis the laws of
Hegelian dialectics.  It seems to me that this dialectics is so closely related
to the whole philosophy of Hegel that through it foreign, from the standpoint
of materialism, formulations enter into the spiritual environment of
materialism---mystical, distorting to it, such as, for example, the
manifestation of dialectics in nature, or in the present case, speaking
scientifically, in the biosphere.

The introduction of the dialectics of nature in the philosophical purview of
our country, in its official philosophy, during our time of great increase, and
significance of science---is a remarkable historical phenomenon.

This has been the form of the post-mortem influence of the works of Marx and
Engels, based on faith---officially---but not expressed philosophically, or
scientifically, etc.\ [by them].


155. Effectiveness, i.e. the equal significance of methodological thought and
the instructions of the philosophers-dialecticians for current scientific work,
is strongly underscored in our philosophical literature, and is introduced into
science through the agency of government power.

The philosophers-dialecticians are convinced that they can aid current
scientific work with their dialectical method.

They believe in its significance for science, but the manifestation of that
belief in reality contradicts it.

It appears to me that this is a misunderstanding.  No philosophy has played, or
plays, such a role in the history of thought.  No philosopher can instruct the
scientist in the pathway to take in the methodology of scientific work,
especially in our times.  The philosopher is not capable of precisely
encompassing the complex problems, whose solutions stand today before the
naturalist in one's current work.  The methods of scientific work in the area
of experimental sciences and descriptive natural sciences, and the methods of
philosophical work, even in the area of dialectical thought, are expressly
different.  It seems to me, the two lie in different domains of thought, as far
as we are dealing with concrete natural phenomena, i.e. with empirically
established facts, and empirical generalizations built upon scientific facts.
It seems to me that the issue here is so clear that no argument is necessary.
Our philosophers-dialecticians must not interfere with this area of scientific
knowledge for their own benefit.  Here, also, their attempt is doomed to
failure from early on.  Here they are fighting with science on its native
terrain.

Science lived through a similar interference of religious thought and religious
constructs, erroneous at their roots, during the age of the Renaissance, during
the 17th--19th centuries.  Though the fight here is not yet over, hardly
anybody would deny that victory has remained on the side of science, that the
majority of religious constructs of that type remained in the past, or are
being reconstructed in their essence, reinterpreted, are shifting from the area
of reality into that of personal belief and interpretation.  The historical
experience was not taken into account by the official philosophers of our
country, and they, in their squareness and insufficient scientific literacy,
entered into a sharp conflict with scientific thought and work, which are
correctly placed ideologically high in our country---on an equal level with
dialectical materialism---at the foundation of our system of government.

The weakness of placing ``dialectical materialism'' at such a height,
unavoidably impacts its real power in nation building, does not correspond to
reality, and unavoidably proves to be transient.

Conflicts with the actual needs of life are beginning, which must unavoidably
have those consequences, which came into being \dots\ supreme
\dots\footnoteEd{Illegible in the manuscript.} in the old Christian nations.


156. I have collided with this kind of circumstances in my scientific work many
times, and have even mentioned the struggle of my predecessors in scientific
knowledge from the past century in public statements.

In 1934 little-educated philosophers, heading the planning of scientific work
of the former Geological Committee\footnoteRus{Геологический комитет},
erroneously attempted to prove, by means of dialectical materialism, that the
determination of geological age by means of radioactivity is based on erroneous
theses---dialectically unproven.  They thought that the facts and empirical
generalizations that radiologists relied upon were dialectically impossible.
They were joined by a few geologists, occupying themselves with philosophy, and
heading the scientific leadership of the Committee.  They held up my work by
one-two years, because the Radium Institute\footnoteRus{Радиевый институт},
which I headed, was completely unable to get in touch with the work of the
Committee geologists, and to put the investigations on a solid basis.  In the
end, after an uncareful statement at the public session of the Committee by the
Vice Scientific Director\footnoteRus{заместителя директора по научной части}
professor M.  M.  Tetyaev\footnoteRus{М. М. Тетяев}, a prominent geologist,
publicly indicating the incompatibility between dialectical materialism and the
conclusions of radiologists, it was possible to achieve a, now public,
discussion on this subject.  It was possible to do so, because the whole
radiological work of the Committee was under attack by his statement.  I was
able to intervene in my role as an Acting Chairman\footnoteRus{и.~о.\
председател} of the Committee on Geological Time\footnoteRus{Комитета по
геологическому времени}, having been elected at the Soviet Union Radiological
Conference\footnoteRus{Всесоюзной Радиологической конференцией}, and to acquire
a public debate of this question.  This took place under my chairmanship at the
premises of the Geological Commitee, where I placed the condition that we, as
insufficiently competent in dialectical philosophy, would only address the
scientific side of the phenomenon.  The striking ignorance of the basic facts
and achievements in the area of radiogeology of all philosophers and many
geologists became undeniably clear to all at that session, attended by a few
hundred geologists and philosophers.  We were able to freely develop our work
to a large degree thanks to the fact that the philosophical leaders of the
Geological Committee soon proved to be heretics according to the official
interpretation of dialectical materialism, and were excluded from the
Committee.  However, they still did harm---weakened our scientific work by a
few years.

The phenomenon which was manifested here---errors in the interpretation of
dialectical materialism by official representatives of the philosophy---is an
everyday and widespread phenomenon of our life.  There are a few philosophers,
whom it didn't suit to retract the philosophical theses set forth by them,
which has been explained by an unconscious mistake, or a conscious one, by a
hidden departure from the official philosophy, or, even, by a conscious
interference with the government.  The wide manifestation of this phenomenon,
totalling hundreds of our philosophers-dialecticians, indicates the clear to
every scientist difficulty in the application of the dialectical method in the
current scientific environment.  For, as is clear from §153, there has been not
one prominent thinker from among the founders of dialectical materialism
throughout the historical course of its development, who has given a complete
treatment of this philosophy, thought through to the end.  It has been created
by them in the dust of fights and polemics, from case to case.

None of them has made a complete presentation, and the attempts by less
prominent thinkers, unavoidably proved to be ephemeral.  Errors were found in
them, they were revoked from circulation, one was to never refer to them.  That
continued tens of times, and there remained no presentation, which could be
considered firm.  The present official presentation of both dialectical
materialism, and of the history of the Communist Party, whose ideology this is,
is dated 1936--1937, and there is no certainty than in a year or two they would
not require new reworking.

I have had the occasion to, also, encounter other manifestations of this
scientific environment.  Inexplicably, the Kant-Laplace hypothesis and the
acceptance of the possibility of abiogenesis were connected to dialectical
materialism, and their negation was considered unacceptable from a dialectical
standpoint.  Such a presentation met censorial difficulties.  Already in 1936
in my report \rtitle{On the Problems of Biogeochemistry}, I ran into objections
of that kind at the session of the Academy.  And I was able to establish the
presently unscientific character of the Kant-Laplace hypothesis, and its
incompatibility with radiogeological data the next year in my official speech
at the International Geological Congress\footnoteRus{Международном
геологическом конгрессе} to the tacit agreement of our geologists, including
those considering themselves dialecticians.

In this case the question is not about the interference of dialectical
materialism with the scientific work of the naturalist in the manner indicated
earlier.

Principally, the naturalist cannot deny the correctness and usefulness of the
interference of philosophers in one's scientific work in many cases, when what
is being dealt with are scientific theories, hypotheses, generalizations of a
non-empirical character, cosmogonic constructs.  Here the naturalist
unavoidably treads upon philosophical terrain.

Even here scientific thought finds itself in a condition, which interferes with
its correct scientific work, in our country.  In this case, our scientific
thought conflicts with an obligatory philosophical dogma, with a definite
philosophy, which, as we have seen, has no firm presentation.  This dogma, with
the lack of free scientific and philosophical investigation in our country,
with the extreme centralization of advance censorship, and all means of
dissemination of scientific knowledge---by way of printed or spoken word---in
the hands of government power, is accepted as obligatory for all, and is
introduced in popular life through the full power of government.

\begin{flushright}
								   1936--1938.
\end{flushright}



\backmatter

% Add the bibliography to the table of contents:
\cleardoublepage
\phantomsection
\addcontentsline{toc}{part}{Bibliography}

\onecolumn
\printbibliography

\end{document}
